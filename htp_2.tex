\documentclass{article}
\usepackage{mathtools}
\usepackage{xcolor}
\usepackage{listings}
\usepackage{amssymb}
\usepackage{tikz}
\usepackage{soul}
\usetikzlibrary{shapes,backgrounds}
\renewcommand{\baselinestretch}{1.5}
\lstset{
  frame=none,
  xleftmargin=2pt,
  stepnumber=1,
  numbers=left,
  numbersep=5pt,
  numberstyle=\ttfamily\tiny\color[gray]{0.3},
  belowcaptionskip=\bigskipamount,
  captionpos=b,
  escapeinside={*'}{'*},
  language=haskell,
  tabsize=2,
  emphstyle={\bf},
  commentstyle=\it,
  stringstyle=\mdseries\rmfamily,
  showspaces=false,
  keywordstyle=\bfseries\rmfamily,
  columns=flexible,
  basicstyle=\small\sffamily,
  showstringspaces=false,
  morecomment=[l]\%,
}
\begin{document}
\topskip0pt
\vspace*{\fill}
\centerline{\sc \large Solutions of the exercises for "How to prove it" book }
\centerline{by drets}
\centerline{\textit{(may contain various errors)}}
\vspace*{\fill}
%
\pagebreak
\centerline{\sc \large 2. Quantificational Logic}
\vspace{50pt}

\textbf{2.1. Quantifiers}

\vspace{40pt}

$$\forall x (\neg L(x, j) \to L(s, x))$$

\centerline{
  \begin{tabular}{c c c}
  L(x, j) & L(s, x) & $\neg L(x, j) \to L(s, x)$ \\
  \hline
  F & F & F \\
  F & T & T \\
  T & F & T \\
  T & T & T \\
  \end{tabular}}
\vspace{30pt}

Exercises:
\vspace{30pt}

1. Analyze the logical forms of the following statements.

\hspace{12pt}(a) Anyone who has forgiven at least one person is a saint.

\hspace{12pt}(b) Nobody in the calculus class is smarter than everybody in the discrete
math class.

\hspace{12pt}(c) Everyone likes Mary, except Mary herself.

\hspace{12pt}(d) Jane saw a police officer, and Roger saw one too.

\hspace{12pt}(e) Jane saw a police officer, and Roger saw him too.
\vspace{30pt}

(a) $\forall x$ (x has forgiven at least one person is a saint)

$\exists y F(x, y)$

F(x, y) stand for "x has forgiven y"

$\forall x (\exists y F(x, y) \to S(x))$

S(x) stand for "x is a saint"
\vspace{20pt}

(b) $\neg \exists x [C(x) \land \forall y (D(y) \to S(x, y))]$

S(x, y) is "x is smarter y"

C(x) is "x is in calculus class"

D(x) is "x is in discrete class"
\vspace{20pt}

(c) $\forall x (\neg L(m, m) \to L(x, m))$

m is "Mary"

L(x, y) is "x likes y"
\vspace{20pt}

(d) $\exists x (P(x) \land S(j, x)) \land \exists y (P(y) \land S(r, y))$

S(x, y) is "x saw y"

j is Jane

r is Roger

P(x) is "x is a police officer"

\vspace{20pt}

(e) $\exists x (P(x) \land S(j, x) \land S(r, x))$

S(x, y) is "x saw y"

j is Jane

r is Roger

P(x) is "x is a police officer"

\vspace{30pt}

2. Analyze the logical forms of the following statements.

\hspace{12pt}(a) Anyone who has bought a Rolls Royce with cash must have a rich
uncle.

\hspace{12pt}(b) If anyone in the dorm has the measles, then everyone who has a friend
in the dorm will have to be quarantined.

\hspace{12pt}(c) If nobody failed the test, then everybody who got an A will tutor
someone who got a D.

\hspace{12pt}(d) If anyone can do it, Jones can.

\hspace{12pt}(e) If Jones can do it, anyone can
\vspace{30pt}

(a) $\forall x (C(x) \to \exists y (R(y) \land U(y, x)))$

C(x) is "x has bought a Rolls Royce with cash"

R(x) is "x is rich"

U(x, y) is "x is uncle of y"

\vspace{20pt}

(b) $\exists x [(D(x) \land M(x)) \to \forall y (F(x, y) \land Q(y))]$

$\exists x [(D(x) \land M(x)) \to \exists z \forall y (D(z) \land F(z, y) \land Q(y))]$

M(x) is "x has the measles"

D(x) is "x is in the dorm"

F(x, y) is "x is a friend of y"

Q(x) is "x will have to be quarantined"

\vspace{20pt}

(c) $\neg \exists x (F(x) \to \forall y \exists z(A(y) \land D(z) \land T(y, z)))$

T(x, y) is "x will tutor y"

F(x) is "x failed the test"

A(x) is "x got an A"

D(x) is "x got a D"

\vspace{20pt}

(d) $\exists x (D(x) \to D(j))$

D(x) is "x can do it"

j is Jones

\vspace{20pt}

(e) $\forall x (D(j) \to D(x))$

D(x) is "x can do it"

j is Jones

\vspace{30pt}

3. Analyze the logical forms of the following statements. The universe of
discourse is $\mathbb{R}$. What are the free variables in each statement?

\hspace{12pt}(a) Every number that is larger than $x$ is larger than $y$.

\hspace{12pt}(b) For every number $a$, the equation $ax^2 + 4x - 2 = 0$ has at least one
solution iff $a \geq −2$

\hspace{12pt}(c) All solutions of the inequality $x^3 - 3x < 3$ are smaller than $10$.

\hspace{12pt}(d) If there is a number $x$ such that $x^2 + 5x = w$ and there is a number $y$
such that $4 - y^2 = w$, then $w$ is between -10 and 10.
\vspace{30pt}

(a) $\forall n (n > x \to n > y)$

x and y are free variables.

\vspace{20pt}

(b) $\forall a \exists x (a > 2 \leftrightarrow ax^2 + 4x - 2 = 0)$

no free variables.

\vspace{20pt}

(c) $\forall x (x^3 - 3x < 3 \to x < 10)$

no free variables.

\vspace{20pt}

(d) $\forall w [(\exists x (x^2 + 5x = w) \land \exists y (4 - y^2 = w) \to (-10 < w < 10)]$

no free variables.

\vspace{30pt}

4. Translate the following statements into idiomatic English.

\hspace{12pt}(a) $\forall x[(H(x) \land \neg \exists y M(x, y)) \to U(x)]$, where H(x) means "x is a man,"
M(x, y) means "x is married to y," and U(x) means "x is unhappy."

\hspace{12pt}(b) $\exists z(P(z, x) \land S(z, y) \land W(y))$, where P(z, x) means "z is a parent of
x," S(z, y)  means "z and y are siblings," and W(y) means "y is a
woman."
\vspace{30pt}

(a) All unmarried men are unhappy
\vspace{20pt}

(b) y is a sister of one of x's parents.

\vspace{30pt}

5. Translate the following statements into idiomatic mathematical English.

\hspace{12pt}(a) $\forall x[(P(x) \land \neg (x = 2)) \to O(x)]$, where P(x) means "x is a prime
number" and O(x) means "x is odd."

\hspace{12pt}(b) $\exists x[P(x) \land \forall y(P(y) \to y \leq x)]$, where P(x) means "x is a perfect
number."
\vspace{30pt}

(a) All x which are prime numbers and not equal to 2 should be odd.
\vspace{20pt}

(b) There is at least one perfect number $x$ such that all perfect numbers are less or equal to $x$.

\vspace{30pt}

6. Are these statements true or false? The universe of discourse is the set of
all people, and P(x, y) means "x is a parent of y."

\hspace{12pt}(a) $\exists x \forall y P(x, y)$.

\hspace{12pt}(b) $\forall x \exists y P(x, y)$.

\hspace{12pt}(c) $\neg \exists x \exists y P(x, y)$.

\hspace{12pt}(d) $\exists x \neg \exists y P(x, y)$.

\hspace{12pt}(e) $\exists x \exists y \neg P(x, y)$.

\vspace{30pt}

(a) false

(there exists person x such that x is a parent of all people)
\vspace{30pt}

(b) false

(all people have child)
\vspace{30pt}

(c) false

(nobody has a child)
\vspace{30pt}

(d) true

(there exists a person without kids)
\vspace{30pt}

(e) true

(there exist person x and person y such that x is not a parent of y)

\vspace{30pt}

7. Are these statements true or false? The universe of discourse is $\mathbb{N}$.

\hspace{12pt}(a) $\forall x \exists y (2x - y = 0)$.

\hspace{12pt}(b) $\exists y \forall x (2x - y = 0)$.

\hspace{12pt}(c) $\forall x \exists y (x - 2y = 0)$.

\hspace{12pt}(d) $\forall x (x < 10 \to \forall y(y < x \to y < 9))$.

\hspace{12pt}(e) $\exists y \exists z (y + z = 100)$.

\hspace{12pt}(f) $\forall x \exists y (y > x \land \exists z (y + z = 100))$.
\vspace{30pt}

(a) true

\vspace{20pt}

(b) false

\vspace{20pt}

(c) false

\vspace{20pt}

(d) true

\vspace{20pt}

(e) true

\vspace{20pt}

(f) false

\vspace{30pt}

8. Same as exercise 7 but with $\mathbb{R}$ as the universe of discourse.
\vspace{30pt}

(a) true

\vspace{20pt}

(b) false

\vspace{20pt}

(c) true

\vspace{20pt}

(d) false

\vspace{20pt}

(e) true

\vspace{20pt}

(f) true

\vspace{30pt}

9. Same as exercise 7 but with $\mathbb{Z}$ as the universe of discourse
\vspace{30pt}

(a) true

\vspace{20pt}

(b) false

\vspace{20pt}

(c) false

\vspace{20pt}

(d) true

\vspace{20pt}

(e) true

\vspace{20pt}

(f) true

\vspace{30pt}

\textbf{2.2. Equivalences Involving Quantifiers}













































\end{document}
