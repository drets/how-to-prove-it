\documentclass{article}
\usepackage{mathtools}
\usepackage{xcolor}
\usepackage{listings}
\usepackage{amssymb}
\usepackage{tikz}
\usepackage{soul}
\usetikzlibrary{shapes,backgrounds}
\renewcommand{\baselinestretch}{1.5}
\lstset{
  frame=none,
  xleftmargin=2pt,
  stepnumber=1,
  numbers=left,
  numbersep=5pt,
  numberstyle=\ttfamily\tiny\color[gray]{0.3},
  belowcaptionskip=\bigskipamount,
  captionpos=b,
  escapeinside={*'}{'*},
  language=haskell,
  tabsize=2,
  emphstyle={\bf},
  commentstyle=\it,
  stringstyle=\mdseries\rmfamily,
  showspaces=false,
  keywordstyle=\bfseries\rmfamily,
  columns=flexible,
  basicstyle=\small\sffamily,
  showstringspaces=false,
  morecomment=[l]\%,
}
\begin{document}
\topskip0pt
\vspace*{\fill}
\centerline{\sc \large Solutions of the exercises for "How to prove it" book }
\centerline{by drets}
\centerline{\textit{(may contain various errors)}}
\vspace*{\fill}
%
\pagebreak
\centerline{\sc \large 2. Quantificational Logic}
\vspace{50pt}

\textbf{2.1. Quantifiers}

\vspace{40pt}

$$\forall x (\neg L(x, j) \to L(s, x))$$

\centerline{
  \begin{tabular}{c c c}
  L(x, j) & L(s, x) & $\neg L(x, j) \to L(s, x)$ \\
  \hline
  F & F & F \\
  F & T & T \\
  T & F & T \\
  T & T & T \\
  \end{tabular}}
\vspace{30pt}

Exercises:
\vspace{30pt}

1. Analyze the logical forms of the following statements.

\hspace{12pt}(a) Anyone who has forgiven at least one person is a saint.

\hspace{12pt}(b) Nobody in the calculus class is smarter than everybody in the discrete
math class.

\hspace{12pt}(c) Everyone likes Mary, except Mary herself.

\hspace{12pt}(d) Jane saw a police officer, and Roger saw one too.

\hspace{12pt}(e) Jane saw a police officer, and Roger saw him too.
\vspace{30pt}

(a) $\forall x$ (x has forgiven at least one person is a saint)

$\exists y F(x, y)$

F(x, y) stand for "x has forgiven y"

$\forall x (\exists y F(x, y) \to S(x))$

S(x) stand for "x is a saint"
\vspace{20pt}

(b) $\neg \exists x [C(x) \land \forall y (D(y) \to S(x, y))]$

S(x, y) is "x is smarter y"

C(x) is "x is in calculus class"

D(x) is "x is in discrete class"
\vspace{20pt}

(c) $\forall x (\neg L(m, m) \to L(x, m))$

m is "Mary"

L(x, y) is "x likes y"
\vspace{20pt}

(d) $\exists x (P(x) \land S(j, x)) \land \exists y (P(y) \land S(r, y))$

S(x, y) is "x saw y"

j is Jane

r is Roger

P(x) is "x is a police officer"

\vspace{20pt}

(e) $\exists x (P(x) \land S(j, x) \land S(r, x))$

S(x, y) is "x saw y"

j is Jane

r is Roger

P(x) is "x is a police officer"

\vspace{30pt}

2. Analyze the logical forms of the following statements.

\hspace{12pt}(a) Anyone who has bought a Rolls Royce with cash must have a rich
uncle.

\hspace{12pt}(b) If anyone in the dorm has the measles, then everyone who has a friend
in the dorm will have to be quarantined.

\hspace{12pt}(c) If nobody failed the test, then everybody who got an A will tutor
someone who got a D.

\hspace{12pt}(d) If anyone can do it, Jones can.

\hspace{12pt}(e) If Jones can do it, anyone can
\vspace{30pt}

(a) $\forall x (C(x) \to \exists y (R(y) \land U(y, x)))$

C(x) is "x has bought a Rolls Royce with cash"

R(x) is "x is rich"

U(x, y) is "x is uncle of y"

\vspace{20pt}

(b) $\exists x [(D(x) \land M(x)) \to \forall y (F(x, y) \land Q(y))]$

$\exists x [(D(x) \land M(x)) \to \exists z \forall y (D(z) \land F(z, y) \land Q(y))]$

M(x) is "x has the measles"

D(x) is "x is in the dorm"

F(x, y) is "x is a friend of y"

Q(x) is "x will have to be quarantined"

\vspace{20pt}

(c) $\neg \exists x (F(x) \to \forall y \exists z(A(y) \land D(z) \land T(y, z)))$

T(x, y) is "x will tutor y"

F(x) is "x failed the test"

A(x) is "x got an A"

D(x) is "x got a D"

\vspace{20pt}

(d) $\exists x (D(x) \to D(j))$

D(x) is "x can do it"

j is Jones

\vspace{20pt}

(e) $\forall x (D(j) \to D(x))$

D(x) is "x can do it"

j is Jones

\vspace{30pt}

3. Analyze the logical forms of the following statements. The universe of
discourse is $\mathbb{R}$. What are the free variables in each statement?

\hspace{12pt}(a) Every number that is larger than $x$ is larger than $y$.

\hspace{12pt}(b) For every number $a$, the equation $ax^2 + 4x - 2 = 0$ has at least one
solution iff $a \geq −2$

\hspace{12pt}(c) All solutions of the inequality $x^3 - 3x < 3$ are smaller than $10$.

\hspace{12pt}(d) If there is a number $x$ such that $x^2 + 5x = w$ and there is a number $y$
such that $4 - y^2 = w$, then $w$ is between -10 and 10.
\vspace{30pt}

(a) $\forall n (n > x \to n > y)$

x and y are free variables.

\vspace{20pt}

(b) $\forall a \exists x (a > 2 \leftrightarrow ax^2 + 4x - 2 = 0)$

no free variables.

\vspace{20pt}

(c) $\forall x (x^3 - 3x < 3 \to x < 10)$

no free variables.

\vspace{20pt}

(d) $\forall w [(\exists x (x^2 + 5x = w) \land \exists y (4 - y^2 = w) \to (-10 < w < 10)]$

no free variables.

\vspace{30pt}

4. Translate the following statements into idiomatic English.

\hspace{12pt}(a) $\forall x[(H(x) \land \neg \exists y M(x, y)) \to U(x)]$, where H(x) means "x is a man,"
M(x, y) means "x is married to y," and U(x) means "x is unhappy."

\hspace{12pt}(b) $\exists z(P(z, x) \land S(z, y) \land W(y))$, where P(z, x) means "z is a parent of
x," S(z, y)  means "z and y are siblings," and W(y) means "y is a
woman."
\vspace{30pt}

(a) All unmarried men are unhappy
\vspace{20pt}

(b) y is a sister of one of x's parents.

\vspace{30pt}

5. Translate the following statements into idiomatic mathematical English.

\hspace{12pt}(a) $\forall x[(P(x) \land \neg (x = 2)) \to O(x)]$, where P(x) means "x is a prime
number" and O(x) means "x is odd."

\hspace{12pt}(b) $\exists x[P(x) \land \forall y(P(y) \to y \leq x)]$, where P(x) means "x is a perfect
number."
\vspace{30pt}

(a) All x which are prime numbers and not equal to 2 should be odd.
\vspace{20pt}

(b) There is at least one perfect number $x$ such that all perfect numbers are less or equal to $x$.

\vspace{30pt}

6. Are these statements true or false? The universe of discourse is the set of
all people, and P(x, y) means "x is a parent of y."

\hspace{12pt}(a) $\exists x \forall y P(x, y)$.

\hspace{12pt}(b) $\forall x \exists y P(x, y)$.

\hspace{12pt}(c) $\neg \exists x \exists y P(x, y)$.

\hspace{12pt}(d) $\exists x \neg \exists y P(x, y)$.

\hspace{12pt}(e) $\exists x \exists y \neg P(x, y)$.

\vspace{30pt}

(a) false

(there exists person x such that x is a parent of all people)
\vspace{30pt}

(b) false

(all people have child)
\vspace{30pt}

(c) false

(nobody has a child)
\vspace{30pt}

(d) true

(there exists a person without kids)
\vspace{30pt}

(e) true

(there exist person x and person y such that x is not a parent of y)

\vspace{30pt}

7. Are these statements true or false? The universe of discourse is $\mathbb{N}$.

\hspace{12pt}(a) $\forall x \exists y (2x - y = 0)$.

\hspace{12pt}(b) $\exists y \forall x (2x - y = 0)$.

\hspace{12pt}(c) $\forall x \exists y (x - 2y = 0)$.

\hspace{12pt}(d) $\forall x (x < 10 \to \forall y(y < x \to y < 9))$.

\hspace{12pt}(e) $\exists y \exists z (y + z = 100)$.

\hspace{12pt}(f) $\forall x \exists y (y > x \land \exists z (y + z = 100))$.
\vspace{30pt}

(a) true

\vspace{20pt}

(b) false

\vspace{20pt}

(c) false

\vspace{20pt}

(d) true

\vspace{20pt}

(e) true

\vspace{20pt}

(f) false

\vspace{30pt}

8. Same as exercise 7 but with $\mathbb{R}$ as the universe of discourse.
\vspace{30pt}

(a) true

\vspace{20pt}

(b) false

\vspace{20pt}

(c) true

\vspace{20pt}

(d) false

\vspace{20pt}

(e) true

\vspace{20pt}

(f) true

\vspace{30pt}

9. Same as exercise 7 but with $\mathbb{Z}$ as the universe of discourse
\vspace{30pt}

(a) true

\vspace{20pt}

(b) false

\vspace{20pt}

(c) false

\vspace{20pt}

(d) true

\vspace{20pt}

(e) true

\vspace{20pt}

(f) true

\vspace{30pt}

\textbf{2.2. Equivalences Involving Quantifiers}

Quantifier Negation laws:

$\neg \exists x P(x)$ is equivalent to $\forall x \neg P(x)$

$\neg \forall x P(x)$ is equivalent to $\exists x \neg P(x)$
\vspace{30pt}

$\exists !x P(x) = \exists x (P(x) \land \neg \exists y (P(y) \land x \neq y))$

Abbreviations:

$\exists x \in A P(x) = \exists x (x \in A \land P(x))$

$\forall x \in A P(x) = \forall x (x \in A \to P(x))$

Universal quantifier distibutes over conjunction:

$\forall x (E(x) \land T(x)) = \forall x E(x) \land \forall x T(x)$

\vspace{50pt}

Exercises:

1. Negate these statements and then reexpress the results as equivalent positive statements. (See Example 2.2.1.)

\hspace{12pt}(a) Everyone who is majoring in math has a friend who needs help with his homework.

\hspace{12pt}(b) Everyone has a roommate who dislikes everyone.

\hspace{12pt}(c) $A \cup B \subseteq C \setminus D$.

\hspace{12pt}(d) $\exists x \forall y [y > x \to \exists z (z^2 + 5z = y)]$.
\vspace{30pt}

(a) $\forall x \exists y (M(x) \land F(x, y) \land H(y) \land x \neq y)$

M(x) is "x is majoring in math"

F(x, y) is "x has a friend y"

H(x) is "x needs help with his homework"

Negating:

$\neg \forall x \exists y (M(x) \land F(x, y) \land H(y) \land x \neq y)$

Quantifier negation law: $\exists x \neg \exists y (M(x) \land F(x, y) \land H(y) \land x \neq y)$

Quantifier negation law: $\exists x \forall y \neg (M(x) \land F(x,y) \land H(y) \land x \neq y)$

DeMorgan's law: $\exists x \forall y (\neg (M(x) \land F(x, y)) \lor \neg (H(y) \land x \neq y))$

Conditional law: $\exists x \forall y ((M(x) \land F(x, y)) \to \neg (H(y) \land x \neq y))$

Someone who is majoring in math doesn't have friends who are needs help with their homeworks.

\vspace{20pt}

(b) $\forall x \exists y (R(x, y) \land \forall z (\neg L(y, z)))$

R(x, y) is "x a roommate y"

L(x, y) is "x likes y"

Negating:

$\neg \forall x \exists y (R(x, y) \land \forall z (\neg L(y, z))$

Quantifier negation law: $\exists x \neg \exists y (R(x, y) \land \forall z (\neg L(y, z)))$

Quantifier negation law: $\exists x \forall y \neg (R(x, y) \land \forall z (\neg L(y, z)))$

DeMorgan's law: $\exists x \forall y (\neg R(x, y) \lor \neg \forall z (\neg L(y, z))$

Conditional law: $\exists x \forall y (R(x, y) \to \neg \forall z (\neg L(y, z)))$

$\exists x \forall y (R(x, y) \to \exists z (L(y, z)))$

There is someone all of whose roommates like at least one person.

\vspace{20pt}

(c) $\forall x (x \in (A \cup B) \to x \in (C \setminus D))$

$\forall x (\neg (x \in A \lor x \in B) \lor (x \in C \land x \notin D))$

Nagating:

$\neg \forall x (\neg (x \in A \lor x \in B) \lor (x \in C \land x \notin D))$

Quantifier negation law: $\exists x \neg (\neg (x \in A \lor x \in B) \lor (x \in C \land x \notin D))$

DeMorgan's law: $\exists x [(x \in A \lor x \in B) \land (x \notin C \lor x \in D)]$

\vspace{20pt}

(d) $\exists x \forall y [y > x \to \exists z (z^2 + 5z = y)]$

Negating:

$\neg \exists x \forall y [y > x \to \exists z (z^2 + 5z = y)]$

$\forall x \neg \forall y [y > x \to \exists z (z^2 + 5z = y)]$ (quantifier negation law)

$\forall x \exists y \neg [y > x \to \exists z (z^2 + 5z = y)]$ (quantifier negation law)

$\forall x \exists y \neg [\neg (y > x) \lor \exists z (z^2 + 5z = y)]$ (conditional law)

$\forall x \exists y [(y > x) \land \neg \exists z (z^2 + 5z = y)]$ (DeMorgan's law)

$\forall x \exists y [(y > x) \land \forall z \neg (z^2 + 5z = y)]$ (quantifier negation law)

$\forall x \exists y [y > x \land \forall z (z^2 + 5z \neq y)]$

\vspace{30pt}

2. Negate these statements and then reexpress the results as equivalent positive statements. (See Example 2.2.1.)

\hspace{12pt}(a) There is someone in the freshman class who doesn't have a roommate.

\hspace{12pt}(b) Everyone likes someone, but no one likes everyone.

\hspace{12pt}(c) $\forall a \in A \exists b \in B(a \in C \leftrightarrow b \in C)$.

\hspace{12pt}(d) $\forall y > 0 \exists x(ax^2 + bx + c = y)$.
\vspace{30pt}

(a) $\exists x [F(x) \land \neg \exists y(R(x, y))]$

F(x) is "x is freshman"

R(x, y) is "x has a roommate y"

Negating:

$\neg \exists x [F(x) \land \neg \exists y(R(x, y))]$

$\forall x \neg [F(x) \land \neg \exists y(R(x, y))]$ (quantifier negation law)

$\forall x [\neg F(x) \lor \exists y(R(x,y))]$ (DeMorgan's law)

$\forall x [F(x) \to \exists y(R(x, y))]$ (conditional law)

All freshman have a roommate

\vspace{20pt}

(b) $\forall x \exists y (L(x,y)) \land \neg \exists z \forall w (L(z, w))$

L(x, y) is "x likes y"

Negating:

$\neg \forall x \exists y (L(x, y)) \lor \neg \neg \exists z \forall w (L(z, w))$

$\exists x \forall y \neg L(x, y) \lor \exists z \forall w L(z, w)$

There exist someone who doesn't like everybody or someone who likes everybody.

\vspace{20pt}

(c) $\neg \forall a \in A \exists b \in B(a \in C \leftrightarrow b \in C)$

$\exists a \in A \forall b \in B \neg (a \in C \leftrightarrow b \in C)$

$\exists a \in A \forall b \in B \neg ((a \in C \to b \in C) \land (b \in C \to a \in C))$

$\exists a \in A \forall b \in B (\neg (a \in C \to b \in C) \lor \neg (b \in C \to a \in C))$

$\exists a \in A \forall b \in B (\neg (\neg (a \in C) \lor b \in C) \lor \neg (\neg (b \in C) \lor a \in C))$

$\exists a \in A \forall b \in B ((a \in C) \land \neg (b \in C)) \lor ((b \in C) \land \neg (a \in C))$

$\exists a \forall b [(a \in B \land b \in B) \land (a \in C \land b \notin C) \lor (b \in C \land a \notin C)]$

\vspace{20pt}

(d) $\neg \forall y > 0 \exists x(ax^2 + bx + c = y)$

$\exists y > 0 \neg \exists x (ax^2 + bx + c = y)$

$\exists y > 0 \forall x \neg (ax^2 + bx + c = y)$

$\exists y > 0 \forall x (ax^2 + bx + c \neq y)$

$\exists y \forall x (y > 0 \land ax^2 + bx + c \neq y)$

\vspace{30pt}

3. Are these statements true or false? The universe of discourse is $\mathbb{N}$.

\hspace{12pt}(a) $\forall x (x < 7 \to \exists a \exists b \exists c(a^2 + b^2 + c^2 = x))$.

\hspace{12pt}(b) $\exists ! x ((x - 4)^2 = 9)$.

\hspace{12pt}(c) $\exists ! x ((x - 4)^2 = 25)$.

\hspace{12pt}(d) $\exists x \exists y ((x - 4)^2 = 25 \land (y - 4)^2 = 25)$.
\vspace{30pt}

(a) true

$0 \leq x \leq 6$

\vspace{20pt}

(b) false

x = 7

x = 1

\vspace{20pt}

(c) false

x = 9

x = -1

\vspace{20pt}

(d) true

\vspace{30pt}

4. Show that the second quantifier negation law, which says that $\neg \forall x P(x)$
is equivalent to $\exists x \neg P(x)$, can be derived from the first, which says that
$\neg \exists x P(x)$ is equivalent to $\forall x \neg P(x)$. (Hint: Use the double negation law.)
\vspace{30pt}

Using $\neg \exists x P(x) =  \forall x \neg P(x)$

prove $\neg \forall x P(x) = \exists x \neg P(x)$

$\neg \neg \forall x P(x) = \neg \exists x \neg P(x)$

$\forall x P(x) = \forall x \neg \neg P(x)$

$\forall x P(x) = \forall x P(x)$

or

$\neg \exists x \neg P(x) =  \forall x \neg \neg P(x)$

$\neg \exists x \neg P(x) = \forall x P(x)$

$\exists x \neg P(x) = \neg \forall x P(x)$

\vspace{30pt}

5. Show that $\neg \exists x \in A P(x)$ is equivalent to $\forall x \in A \neg P(x)$.
\vspace{30pt}

$\neg \exists x \in A P(x)$

$\neg \exists x (x \in A \land P(x))$ (expanding abbreviation)

$\forall x \neg (x \in A \land P(x))$ (quantifier negation law)

$\forall x (\neg x \in A \lor \neg P(x))$ (DeMorgan's law)

$\forall x (x \in A \to \neg P(x))$ (conditional law)

$\forall x \in A \neg P(x)$ (abbreviation)

\vspace{30pt}

6. Show that the existential quantifier distributes over disjunction. In other
words, show that $\exists x (P(x) \lor Q(x))$ is equivalent to $\exists x P(x) \lor \exists x Q(x)$.
(Hint: Use the fact, discussed in this section, that the universal quantifier distributes over conjunction.)
\vspace{30pt}

$\exists x (P(x) \lor Q(x))$

$\neg \forall x \neg (P(x) \lor Q(x))$

$\neg \forall x (\neg P(x) \land \neg Q(x))$

$\neg (\forall x \neg P(x) \land \forall x \neg Q(x))$

$\neg \forall x \neg P(x) \lor \neg \forall x \neg Q(x)$

$\exists x \neg \neg P(x) \lor \exists x \neg \neg Q(x)$

$\exists x P(x) \lor \exists x Q(x)$

\vspace{30pt}

7. Show that $\exists x (P(x) \to Q(x))$ is equivalent to $\forall x P(x) \to \exists x Q(x)$.
\vspace{30pt}

$\exists x (P(x) \to Q(x))$

$\neg \neg \exists x (P(x) \to Q(x))$ (double negation law)

$\neg \forall x \neg (P(x) \to Q(x))$ (quantifier negation law)

$\neg \forall x \neg (\neg P(x) \lor Q(x))$ (conditional law)

$\neg \forall x (P(x) \land \neg Q(x))$ (DeMorgan's law)

$\neg (\forall x P(x) \land \forall x \neg Q(x))$ (universal quantifier distribution law)

$\neg \forall x P(x) \lor \neg \forall x \neg Q(x)$ (DeMorgan's law)

$\neg \forall x P(x) \lor \exists x \neg \neg Q(x)$ (quantifier negation law)

$\neg \forall x P(x) \lor \exists x Q(x)$ (double negation law)

$\forall x P(x) \to \exists x Q(x)$ (conditional law)

\vspace{30pt}

8. Show that $(\forall x \in A \, P(x)) \land (\forall x \in B \, P(x))$ is equivalent to
$\forall x \in (A \cup B) \, P(x)$. (Hint: Start by writing out the meanings of the bounded
quantifiers in terms of unbounded quantifiers.)
\vspace{30pt}

$(\forall x (x \in A \to P(x))) \land (\forall x (x \in B \to P(x)))$ (expanding abbreviation)

$\forall x ((x \in A \to P(x)) \land (x \in B \to P(x)))$ (universal quantifiers distribution law)

$\forall x ((\neg (x \in A) \lor P(x)) \land (\neg (x \in B) \lor P(x)))$ (conditional law)

$\forall x ((\neg (x \in A) \land \neg (x \in B)) \lor P(x))$ (distribution law)

$\forall x (\neg (x \in A \lor x \in B) \lor P(x))$ (DeMorgan's law)

$\forall x ((x \in A \lor x \in B) \to P(x))$ (conditional law)

$\forall x (x \in (A \cup B) \to P(x))$

$\forall x \in (A \cup B) \, P(x)$

\vspace{30pt}

9. Is $\forall x (P(x) \lor Q(x))$ equivalent to $\forall x P(x) \lor \forall x Q(x)$?
Explain. (Hint: Try assigning meanings to P(x) and Q(x).)
\vspace{30pt}

$\forall x (P(x) \lor Q(x))$

P(x) "x is tall"

Q(x) "x is smart"

All people are either tall or smart.
\vspace{20pt}

$\forall x P(x) \lor \forall x Q(x)$

All people are tall or all people are smart.

$\forall x (P(x) \land Q(x)) \neq \forall x P(x) \lor \forall x Q(x)$

\vspace{30pt}

10. (a) Show that $\exists x \in A \, P(x) \lor \exists x \in B \, P(x)$ is equivalent to $\exists x \in (A \cup B) \, P(x)$.

\hspace{12pt}(b) Is $\exists x \in A \, P(x) \land \exists x \in B \, P(x)$ equivalent to $\exists x \in (A \cap B) \, P(x)$?
Explain.
\vspace{30pt}

(a) $\exists x \in A \, P(x) \lor \exists x \in B \, P(x)$

$\exists x [(x \in A) \land P(x)) \lor \exists ((x \in B) \land P(x)]$

$\exists x [((x \in A) \land P(x)) \lor ((x \in B) \land P(x))]$

$\exists x [((x \in A) \lor (x \in B)) \land P(x)]$

$\exists x (x \in (A \cup B) \land P(x))$

$\exists x \in (A \cup B) \, P(x)$

\vspace{20pt}

(b) $\exists x \in A \, P(x) \land \exists x \in B \, P(x)$

$\exists x ((x \in A) \land P(x)) \land \exists x ((x \in B) \land P(x))$
\vspace{20pt}

$\exists x \in (A \cap B) P(x)$

$\exists x (x \in (A \cap B) \land P(x))$

$\exists x ((x \in A) \land (x \in B) \land P(x))$
\vspace{20pt}

A is \{1,2,3\}

B is \{4,5,7\}

P(x) is "$x < 6$"

$\exists x ((x \in A) \land P(x)) \land \exists x ((x \in B) \land P(x))$

$\exists x ((x \in \{1,2,3\}) \land x < 6) \land \exists x ((x \in \{4,5,7\}) \land x < 6)$

$true \land true = true$
\vspace{20pt}

$\exists x ((x \in \{1,2,3\} \land x \in \{4,5,7\} \land x < 6))$

$\exists x (x \in \varnothing \land x < 6)$ is false

So $\exists x \in A \, P(x) \land \exists x \in B \, P(x)$ is not equivalent to $\exists x \in (A \cap B) \, P(x)$.

\vspace{30pt}

11. Show that the statements $A \subseteq B$ and $A \setminus B = \varnothing$ are equivalent by writing
each in logical symbols and then showing that the resulting formulas are equivalent.
\vspace{30pt}

$A \subseteq B$ means $\forall x (x \in A \to x \in B)$

$A \setminus B = \varnothing$ means $\neg \exists x (x \in A \land x \notin B)$

$\forall x \neg (x \in A \land x \notin B)$

$\forall x (\neg x \in A \lor x \in B)$

$\forall x (x \in A \to x \in B)$

$A \subseteq B$

\vspace{30pt}

12. Let T(x, y) mean "x is a teacher of y."
What do the following statements mean?
Under what circumstances would each one be true?
Are any of them equivalent to each other?

\hspace{12pt}(a) $\exists !y T(x, y)$.

\hspace{12pt}(b) $\exists x \exists ! y T(x, y)$.

\hspace{12pt}(c) $\exists ! x \exists y T(x, y)$.

\hspace{12pt}(d) $\exists y \exists ! x T(x, y)$.

\hspace{12pt}(e) $\exists ! x \exists ! y T(x, y)$.

\hspace{12pt}(f) $\exists x \exists y [T (x, y) \land \neg \exists u \exists v (T(u, v) \land (u \neq  x \lor v \neq y))]$.
\vspace{20pt}

(a) There is exactly one student who has $x$ teacher.

Statement will be true if we choose instead of $x$ a teacher who has exactly one student.
\vspace{20pt}

(b) There is at least one teacher who has exactly one student.

Statement will be true if universe of discourse has at least one teacher who has exactly one student.
\vspace{20pt}

(c) There is exactly one teacher who has at least one student.

Statement will be true if universe of discourse has exactly one teacher who has at least one student.
\vspace{20pt}

(d) There is exactly one teacher who has at least one student

$\exists y \exists ! x T(x, y)$ is equivalent to $\exists ! x \exists ! y T(x, y)$

Statement will be true if universe of discourse has exactly one teacher who has at least one student.
\vspace{20pt}

(e) There is exactly one teacher who has exactly one student.

Statement will be true if universe of discourse has exactly one teacher who has exactly one student.
\vspace{20pt}

(f) There is exactly one teacher who has exactly one student.

$\exists x \exists y [T (x, y) \land \neg \exists u \exists v (T(u, v) \land (u \neq  x \lor v \neq y))]$ is equivalent to

$\exists ! x \exists ! y T(x, y)$

Statement will be true if universe of discourse has exactly one teacher who has exactly one student.
\vspace{20pt}


(c) equivalent to (d)

(e) equivalent to (f)

\vspace{50pt}


\textbf{2.3. More Operations on Sets}










































\end{document}
