\documentclass{article}
\usepackage{mathtools}
\usepackage{xcolor}
\usepackage{listings}
\usepackage{amssymb}
\renewcommand{\baselinestretch}{1.5}
\lstset{
  frame=none,
  xleftmargin=2pt,
  stepnumber=1,
  numbers=left,
  numbersep=5pt,
  numberstyle=\ttfamily\tiny\color[gray]{0.3},
  belowcaptionskip=\bigskipamount,
  captionpos=b,
  escapeinside={*'}{'*},
  language=haskell,
  tabsize=2,
  emphstyle={\bf},
  commentstyle=\it,
  stringstyle=\mdseries\rmfamily,
  showspaces=false,
  keywordstyle=\bfseries\rmfamily,
  columns=flexible,
  basicstyle=\small\sffamily,
  showstringspaces=false,
  morecomment=[l]\%,
}
\begin{document}
\topskip0pt
\vspace*{\fill}
\centerline{\sc \large Solutions for the exercises from "How to prove it" book }
\centerline{\textit{may contain various errors}}
\vspace*{\fill}
%
\pagebreak
\centerline{\sc \large 0. Inroduction}
\vspace{50pt}

1. (a) Factor $2^{15} - 1 = 32,767$ into a product of two smaller positive integers.

\hspace{12pt}(b) Find an integer $x$ such that $1 < x < 2^{32767} − 1$ and $2^{32767} - 1$ is divisible by x.
\vspace{20pt}


(a) $n = 15$. Since $3 * 5 = 12$, we could use the values $a = 3$ and $b = 5$.
$x = 2^b - 1 = 2^5 - 1 = 31$ and $y = 1 + 2^b + 2^{2*b} + \dotso + 2^{(a-1)b} = 1 + 2^5 + 2^{10} = 1 + 32 + 1024 = 1057$,
so $1057 * 31 = 32,767$
\vspace{10pt}

(b) $n=32767$. Since $7 * 4681 = 32767$, we could use the values $a = 4681$ and $b = 7$.
$x = 2^7 - 1 = 127$
\vspace{40pt}

2. Make some conjectures about the values of $n$ for which $3*n - 1$ is prime or
the values of n for which $3*n - 2*n$ is prime. (You might start by making a
table similar to Figure 1.)
\vspace{20pt}

The following simple Haskell program:
\lstinputlisting[language=Haskell]{0-2.hs}
gives the output:

\begin{verbatim}
"2. 3^n - 1: not prime | 3^n - 2^n: prime"
"3. 3^n - 1: not prime | 3^n - 2^n: prime"
"4. 3^n - 1: not prime | 3^n - 2^n: not prime"
"5. 3^n - 1: not prime | 3^n - 2^n: prime"
"6. 3^n - 1: not prime | 3^n - 2^n: not prime"
"7. 3^n - 1: not prime | 3^n - 2^n: not prime"
"8. 3^n - 1: not prime | 3^n - 2^n: not prime"
"9. 3^n - 1: not prime | 3^n - 2^n: not prime"
"10. 3^n - 1: not prime | 3^n - 2^n: not prime"
"11. 3^n - 1: not prime | 3^n - 2^n: not prime"
"12. 3^n - 1: not prime | 3^n - 2^n: not prime"
"13. 3^n - 1: not prime | 3^n - 2^n: not prime"
"14. 3^n - 1: not prime | 3^n - 2^n: not prime"
"15. 3^n - 1: not prime | 3^n - 2^n: not prime"
"16. 3^n - 1: not prime | 3^n - 2^n: not prime"
"17. 3^n - 1: not prime | 3^n - 2^n: prime"
"18. 3^n - 1: not prime | 3^n - 2^n: not prime"
"19. 3^n - 1: not prime | 3^n - 2^n: not prime"
"20. 3^n - 1: not prime | 3^n - 2^n: not prime"
"21. 3^n - 1: not prime | 3^n - 2^n: not prime"
"22. 3^n - 1: not prime | 3^n - 2^n: not prime"
"23. 3^n - 1: not prime | 3^n - 2^n: not prime"
"24. 3^n - 1: not prime | 3^n - 2^n: not prime"
"25. 3^n - 1: not prime | 3^n - 2^n: not prime"
"26. 3^n - 1: not prime | 3^n - 2^n: not prime"
"27. 3^n - 1: not prime | 3^n - 2^n: not prime"
"28. 3^n - 1: not prime | 3^n - 2^n: not prime"
"29. 3^n - 1: not prime | 3^n - 2^n: prime"
"30. 3^n - 1: not prime | 3^n - 2^n: not prime"
"31. 3^n - 1: not prime | 3^n - 2^n: prime"
"32. 3^n - 1: not prime | 3^n - 2^n: not prime"
"33. 3^n - 1: not prime | 3^n - 2^n: not prime"
"34. 3^n - 1: not prime | 3^n - 2^n: not prime"
"35. 3^n - 1: not prime | 3^n - 2^n: not prime"
"36. 3^n - 1: not prime | 3^n - 2^n: not prime"
"37. 3^n - 1: not prime | 3^n - 2^n: not prime"
"38. 3^n - 1: not prime | 3^n - 2^n: not prime"
"39. 3^n - 1: not prime | 3^n - 2^n: not prime"
"40. 3^n - 1: not prime | 3^n - 2^n: not prime"
"41. 3^n - 1: not prime | 3^n - 2^n: not prime"
"42. 3^n - 1: not prime | 3^n - 2^n: not prime"
"43. 3^n - 1: not prime | 3^n - 2^n: not prime"
"44. 3^n - 1: not prime | 3^n - 2^n: not prime"
"45. 3^n - 1: not prime | 3^n - 2^n: not prime"
"46. 3^n - 1: not prime | 3^n - 2^n: not prime"
"47. 3^n - 1: not prime | 3^n - 2^n: not prime"
"48. 3^n - 1: not prime | 3^n - 2^n: not prime"
"49. 3^n - 1: not prime | 3^n - 2^n: not prime"
"50. 3^n - 1: not prime | 3^n - 2^n: not prime"
\end{verbatim}
Conjecture 1: $3^n - 1$ doesn't contain prime numbers.
\vspace{40pt}

3. The proof of Theorem 3 gives a method for finding a prime number different
from any in a given list of prime numbers.

(a) Use this method to find a prime different from $2$, $3$, $5$, and $7$.

(b) Use this method to find a prime different from $2$, $5$, and $11$.
\vspace{20pt}

(a) $2*3*5*7+1 = 211$
\vspace{10pt}

(b) $$1*2+1=3$$
    $$11*2+1=23$$
\vspace{40pt}

4. Find five consecutive integers that are not prime.
\vspace{20pt}

$764, 765, 766, 767, 768$
\vspace{40pt}

5. Use the table in Figure 1 and the discussion on p. 5 to find two more perfect
numbers.
\vspace{20pt}

1) $n = 5$, $2^n-1 = 2^5-1=31$ (prime); Using the formula for perfect number $2^{n-1}*(2^n-1) = 2^4*(2^5-1) = 16*31 = 496$
\vspace{10pt}

2) $n = 7$, $2^7-1= 128 - 1 = 127$ (prime); Calculating prime: $2^6*(2^7-1) = 64*127 = 8128$
\vspace{40pt}

6. The sequence $3, 5, 7$ is a list of three prime numbers such that each pair of
adjacent numbers in the list differ by two. Are there any more such “triplet
primes”?
\vspace{20pt}

\lstinputlisting[language=Haskell]{0-6.hs}

Using the program above I couldn't find another “triplet”. I alse tried the case when each pair of adjacent numbers in the list differ by three.
\pagebreak

\centerline{\sc \large 1. Sentential Logic}
\vspace{50pt}

\textbf{1.1. Deductive Reasoning and Logical Connectives}
\vspace{40pt}

1. Analyze the logical forms of the following statements:

\hspace{12pt}(a) We'll have either a reading assignment or homework problems, but we
won't have both homework problems and a test.

\hspace{12pt}(b) You won't go skiing, or you will and there won't be any snow.

\hspace{12pt}(c) $\sqrt{7} \nleq 2$.
\vspace{20pt}

(a) Let P be "we have reading assignment" and Q be "we have homework problems", then

$$(P \lor Q) \lor \neg (P \land Q)$$
\vspace{10pt}

(b) $$\neg Q \lor (Q \land \neg R)$$
\vspace{10pt}

(c) $$\neg [(\sqrt{7} < 2) \lor (\sqrt{7} = 2)]$$
\vspace{40pt}

2. Analyze the logical forms of the following statements:

\hspace{12pt}(a) Either John and Bill are both telling the truth, or neither of them is.

\hspace{12pt}(b) I’ll have either fish or chicken, but I won’t have both fish and mashed
potatoes.

\hspace{12pt}(c) 3 is a common divisor of 6, 9, and 15
\vspace{20pt}

(a) let A be "John tells the truth" and B be "Bill tells the truth"
$$(A \land B) \lor \neg (A \land B)$$
\vspace{10pt}

(b) $$(A \lor B) \land \neg (A \land C)$$
\vspace{10pt}

(c) $$A \land B \land C$$
\vspace{40pt}

3. Analyze the logical forms of the following statements:

\hspace{12pt}(a) Alice and Bob are not both in the room.

\hspace{12pt}(b) Alice and Bob are both not in the room.

\hspace{12pt}(c) Either Alice or Bob is not in the room.

\hspace{12pt}(d) Neither Alice nor Bob is in the room.
\vspace{20pt}

(a) $$(\neg A \land B) \lor (A \land \neg B) \lor (\neg A \land \neg B)$$
\vspace{10pt}

(b) $$\neg (A \land B)$$
\vspace{10pt}

(c) $$(\neg A \land B) \lor (A \land \neg B)$$
\vspace{10pt}

(d) $$\neg A \land \neg B$$
\vspace{40pt}

4. Which of the following expressions are well-formed formulas?

\hspace{12pt}(a) $\neg (\neg P \lor \neg \neg R)$

\hspace{12pt}(b) $\neg (P, Q, \neg R)$

\hspace{12pt}(c) $P \land \neg P$

\hspace{12pt}(d) $(P \land Q)(P \lor R)$
\vspace{20pt}

\hspace{12pt}(c) is well-formed formula.
\vspace{40pt}

5. Let P stand for the statement "I will buy the pants" and S for the statement
"I will buy the shirt." What English sentences are represented by the following
expressions?

\hspace{12pt}(a) $\neg (P \land \neg S)$

\hspace{12pt}(b) $\neg P \land \neg S$.

\hspace{12pt}(c) $\neg P \lor \neg S$
\vspace{20pt}

(a) I won't buy the pants without the shirt.
\vspace{10pt}

(b) Neither I will buy the pants nor I will buy the shirt.
\vspace{10pt}

(c) Either I won't buy the pants or I won't buy the shirt.
\vspace{40pt}

6. Let S stand for the statement "Steve is happy" and G for "George is happy."
What English sentences are represented by the following expressions?

\hspace{12pt}(a) $(S \lor G) \land (\neg S \lor \neg G)$.

\hspace{12pt}(b) $[S \lor (G \land \neg S)] \lor \neg G$

\hspace{12pt}(c) $S \lor [G \land (\neg S \lor \neg G)]$
\vspace{20pt}

(a) Steve or George are happy, but either Steve is not happy or George.

(b) Steve is happy or George is happy, but Steve not; or George is not happy.

(c) Either Steve is happy or George is happy, but Steve is not happy or George.
\vspace{40pt}

7. Identify the premises and conclusions of the following deductive arguments
and analyze their logical forms. Do you think the reasoning is valid?
(Although you will have only your intuition to guide you in answering
this last question, in the next section we will develop some techniques for
determining the validity of arguments.)

\hspace{12pt}(a) Jane and Pete won’t both win the math prize. Pete will win either
the math prize or the chemistry prize. Jane will win the math prize.
Therefore, Pete will win the chemistry prize.

\hspace{12pt}(b) The main course will be either beef or fish. The vegetable will be either
peas or corn. We will not have both fish as a main course and corn as a
vegetable. Therefore, we will not have both beef as a main course and
peas as a vegetable.

\hspace{12pt}(c) Either John or Bill is telling the truth. Either Sam or Bill is lying.
Therefore, either John is telling the truth or Sam is lying.

\hspace{12pt}(d) Either sales will go up and the boss will be happy, or expenses will go up and the boss won’t be happy. Therefore, sales and expenses will not
both go up.
\vspace{20pt}

(a) $$(\neg J \land P) \lor (J \land \neg P)$$
$$P \lor C$$
$$J$$
$$C$$

Valid

\vspace{20pt}

(b) $$(B \land \neg F) \lor (\neg B \land F)$$
$$(P \land \neg C) \lor (\neg P \land C)$$
$$\neg (F \land C)$$
$$\neg (B \land P)$$
Invalid

\vspace{20pt}

(c) $$(J \land \neg B) \lor (\neg J \land B)$$
$$\neg S \lor \neg B$$
$$J \lor \neg S$$
Valid

\vspace{20pt}

(d) $$(S \land H) \lor (E \land \neg H)$$
$$\neg (S \land E)$$

Valid

\vspace{50pt}

\textbf{1.2. Truth Tables}

\vspace{40pt}

\centerline{
  \begin{tabular}{c c l c c}
  S & L & $(\neg S \land L) \lor S$ & S & $\neg L$ \\
  \hline
  F & F & T F F F \textbf{F} F & F & T \\
  F & T & T F T T \textbf{T} F & F & F \\
  T & F & F T F F \textbf{T} T & T & T \\
  T & T & F T F T \textbf{T} T & T & F \\
  \end{tabular}}
\vspace{20pt}

\centerline{
  \begin{tabular}{c c l l l}
  P & Q & $\neg (P \land Q)$ & $\neg P \land \neg Q $ & $\neg P \lor \neg Q$ \\
  \hline
  F & F & \textbf{T} F F F & T F \textbf{T} T F & T F \textbf{T} T F \\
  F & T & \textbf{T} F F T & T F \textbf{F} F T & T F \textbf{T} F T \\
  T & F & \textbf{T} T F F & F T \textbf{F} T F & F T \textbf{T} T F \\
  T & T & \textbf{F} T T T & F T \textbf{F} F T & F T \textbf{F} F T \\
  \end{tabular}}
\vspace{20pt}

\textbf{DeMorgan's laws}
\vspace{20pt}

$\neg (P \land Q)$ is equivalent to $\neg P \lor \neg Q$

\centerline{
  \begin{tabular}{c c l l}
  P & Q & $\neg (P \land Q)$ & $\neg P \lor \neg Q$ \\
  \hline
  F & F & \textbf{T} F F F & T F \textbf{T} T F \\
  F & T & \textbf{T} F F T & T F \textbf{T} F T \\
  T & F & \textbf{T} T F F & F T \textbf{T} T F \\
  T & T & \textbf{F} T T T & F T \textbf{F} F T \\
  \end{tabular}}
\vspace{10pt}

Let P stand for "Alice is smart" and P stand for "Bob is smart"

Alice and Bob aren't both smart

Either Alice or Bob aren't smart.
\vspace{20pt}

$\neg (P \lor Q)$ is equivalent to $\neg P \land \neg Q$

\centerline{
  \begin{tabular}{c c l l }
  P & Q & $\neg (P \lor Q)$ & $\neg P \land \neg Q $ \\
  \hline
  F & F & \textbf{T} F F F & T F \textbf{T} T F \\
  F & T & \textbf{F} F T T & T F \textbf{F} F T \\
  T & F & \textbf{F} T T F & F T \textbf{F} T F \\
  T & T & \textbf{F} T T T & F T \textbf{F} F T \\
  \end{tabular}}
\vspace{10pt}

Let P stand for "Alice is smart" and P stand for "Bob is smart"

Both Alice or Bob aren't smart.

Alice isn't smart and Bob isn't smart.
\vspace{20pt}

\textbf{Commutative laws}
\vspace{20pt}

$P \land Q$ is equivalent to $Q \land P$

\centerline{
  \begin{tabular}{c c l l }
  P & Q & $P \land Q$ & $Q \land P$ \\
  \hline
  F & F & F \textbf{F} F & F \textbf{F} F \\
  F & T & F \textbf{F} T & T \textbf{F} F \\
  T & F & T \textbf{F} F & F \textbf{F} T \\
  T & T & T \textbf{T} T & T \textbf{T} T \\
  \end{tabular}}
\vspace{10pt}

Alice is smart and Bob is smart

Bob is smart and Alice is smart
\vspace{20pt}

$P \lor Q$ is equivalent to $Q \lor P$

\centerline{
  \begin{tabular}{c c l l }
  P & Q & $P \lor Q$ & $Q \lor P$ \\
  \hline
  F & F & F \textbf{F} F & F \textbf{F} F \\
  F & T & F \textbf{T} T & T \textbf{T} F \\
  T & F & T \textbf{T} F & F \textbf{T} T \\
  T & T & T \textbf{T} T & T \textbf{T} T \\
  \end{tabular}}
\vspace{10pt}

Alice is smart or Bob is smart

Bob is smart or Alice is smart
\vspace{20pt}

\textbf{Associative laws}
\vspace{20pt}

$P \land (Q \land R)$ is equivalent to $(P \land Q) \land R$

\centerline{
  \begin{tabular}{c c c l l}
  P & Q & R & $P \land (Q \land R)$ & $(P \land Q) \land R$ \\
  \hline
  F & F & F & F \textbf{F} F F F & F F F \textbf{F} F \\
  F & T & F & F \textbf{F} T F F & F F T \textbf{F} F \\
  F & F & T & F \textbf{F} F F T & F F F \textbf{F} T \\
  F & T & T & F \textbf{F} T T T & F F T \textbf{F} T \\
  T & F & F & T \textbf{F} F F F & T F F \textbf{F} F \\
  T & T & F & T \textbf{F} T F F & T T T \textbf{F} F \\
  T & F & T & T \textbf{F} F F T & T F F \textbf{F} T \\
  T & T & T & T \textbf{T} T T T & T T T \textbf{T} T \\
  \end{tabular}}
\vspace{10pt}

P: Alice is smart

Q: Bob is smart

R: Eve is smart

Bob and Eve are both smart and Alice is smart

Alice and Bob are both smart and Eve is smart
\vspace{20pt}

$P \lor (Q \lor R)$ is equivalent to $(P \lor Q) \lor R$

\centerline{
  \begin{tabular}{c c c l l}
  P & Q & R & $P \lor (Q \lor R)$ & $(P \lor Q) \lor R$ \\
  \hline
  F & F & F & F \textbf{F} F F F & F F F \textbf{F} F \\
  F & T & F & F \textbf{T} T T F & F T T \textbf{T} F \\
  F & F & T & F \textbf{T} F T T & F F F \textbf{T} T \\
  F & T & T & F \textbf{T} T T T & F T T \textbf{T} T \\
  T & F & F & T \textbf{T} F F F & T T F \textbf{T} F \\
  T & T & F & T \textbf{T} T T F & T T T \textbf{T} F \\
  T & F & T & T \textbf{T} F T T & T T F \textbf{T} T \\
  T & T & T & T \textbf{T} T T T & T T T \textbf{T} T \\
  \end{tabular}}
\vspace{10pt}

P: Alice is smart

Q: Bob is smart

R: Eve is smart

Bob or Eve are smart, or Alice is smart

Either Alice or Bob are smart, or Eve is smart
\vspace{20pt}

\textbf{Idemponent laws}
\vspace{20pt}

$P \land P$ is equivalent to P

\centerline{
  \begin{tabular}{c l}
  P & $P \land P$ \\
  \hline
  T & T \textbf{T} T \\
  F & F \textbf{F} F \\
  \end{tabular}}
\vspace{10pt}

P: Alice is smart

Alice is smart and Alice is smart.

Alice is smart.
\vspace{20pt}

$P \lor P$ is equivalent to P

\centerline{
  \begin{tabular}{c l}
  P & $P \lor P$ \\
  \hline
  T & T \textbf{T} T \\
  F & F \textbf{F} F \\
  \end{tabular}}
\vspace{10pt}

P: Alice is smart

Alice is smart or Alice is smart.

Alice is smart.
\vspace{20pt}

\textbf{Distributive laws}
\vspace{20pt}

$P \land (Q \lor R)$ is equivalent to $(P \land Q) \lor (P \land R)$

\centerline{
  \begin{tabular}{c c c l l}
  P & Q & R & $P \land (Q \lor R)$ & $(P \land Q) \lor (P \land R)$ \\
  \hline
  F & F & F & F \textbf{F} F F F & F F F \textbf{F} F F F \\
  F & T & F & F \textbf{F} T T F & F F T \textbf{F} F F F \\
  F & F & T & F \textbf{F} F T T & F F F \textbf{F} F F T \\
  F & T & T & F \textbf{F} T T T & F F T \textbf{F} F F T \\
  T & F & F & T \textbf{F} F F F & T F F \textbf{F} T F F \\
  T & T & F & T \textbf{T} T T F & T T T \textbf{T} T F F \\
  T & F & T & T \textbf{T} F T T & T F F \textbf{T} T T T \\
  T & T & T & T \textbf{T} T T T & T T T \textbf{T} T T T \\
  \end{tabular}}
\vspace{10pt}

P: Alice is smart

Q: Bob is smart

R: Eve is smart

Bob and Eve are both smart, and Alice is smart.

Either Alice and Bob are both smart or Alice and Eve are both smart.
\vspace{20pt}

$P \lor (Q \land R)$ is equivalent to $(P \lor Q) \land (P \lor R)$

\centerline{
  \begin{tabular}{c c c l l}
  P & Q & R & $P \lor (Q \land R)$ & $(P \lor Q) \land (P \lor R)$ \\
  \hline
  F & F & F & F \textbf{F} F F F & F F F \textbf{F} F F F \\
  F & T & F & F \textbf{F} T F F & F T T \textbf{F} F F F \\
  F & F & T & F \textbf{F} F F T & F F F \textbf{F} F T T \\
  F & T & T & F \textbf{T} T T T & F T T \textbf{T} F T T \\
  T & F & F & T \textbf{T} F F F & T T F \textbf{T} T T F \\
  T & T & F & T \textbf{T} T F F & T T T \textbf{T} T T F \\
  T & F & T & T \textbf{T} F F T & T T F \textbf{T} T T T \\
  T & T & T & T \textbf{T} T T T & T T T \textbf{T} T T T \\
  \end{tabular}}
\vspace{10pt}

P: Alice is smart

Q: Bob is smart

R: Eve is smart

Bob and Eve are both smart or Alice is smart.

Either Alice or Bob are smart and either Alice and Eve are smart.
\vspace{20pt}

\textbf{Absorption laws}
\vspace{20pt}

$P \lor (P \land Q)$ is equivalent to P

\centerline{
  \begin{tabular}{c c l}
  P & Q & $P \lor (P \land Q)$ \\
  \hline
  F & F & F \textbf{F} F F F \\
  F & T & F \textbf{F} F F T \\
  T & F & T \textbf{T} T F F \\
  T & T & T \textbf{T} T T T \\
  \end{tabular}}
\vspace{10pt}

P: Alice is smart

Q: Bob is smart

Alice is smart or Alice and Bob are both smart.

Alice is smart.
\vspace{20pt}

$P \land (P \lor Q)$ is equivalent to P

\centerline{
  \begin{tabular}{c c l}
  P & Q & $P \land (P \lor Q)$ \\
  \hline
  F & F & F \textbf{F} F F F \\
  F & T & F \textbf{F} F T T \\
  T & F & T \textbf{T} T T F \\
  T & T & T \textbf{T} T T T \\
  \end{tabular}}
\vspace{10pt}

P: Alice is smart

Q: Bob is smart

Alice is smart and either Alice or Bob are smart.

Alice is smart.
\vspace{20pt}

\textbf{Double Negation laws}
\vspace{20pt}

$\neg \neg P$ is equivalent to P

\centerline{
  \begin{tabular}{c l}
  P & $\neg \neg P$ \\
  \hline
  T & \textbf{T} F T \\
  \end{tabular}}
\vspace{10pt}

P: Alice is smart

Alice isn't stupid.
\vspace{40pt}

1. Make truth tables for the following formulas:

\hspace{12pt}(a) $\neg P \lor Q$.

\hspace{12pt}(b) $(S \lor G) \land (\neg S \lor \neg G)$.
\vspace{20pt}

(a)

\centerline{
  \begin{tabular}{c c l}
  P & Q & $\neg P \lor Q$ \\
  \hline
  F & F & T F \textbf{T} F \\
  F & T & T F \textbf{T} T \\
  T & F & F T \textbf{F} F \\
  T & T & F T \textbf{T} T \\
  \end{tabular}}
\vspace{10pt}

(b)

\centerline{
  \begin{tabular}{c c l}
  S & G & $(S \lor G) \land (\neg S \lor \neg G)$ \\
  \hline
  F & F & F F F \textbf{F} T F T T F \\
  F & T & F T T \textbf{T} T F T F T \\
  T & F & T T F \textbf{T} F T T T F \\
  T & T & T T T \textbf{F} F T F F T \\
  \end{tabular}}
\vspace{30pt}

2. Make truth tables for the following formulas:

\hspace{12pt}(a) $\neg [P \land (Q \lor \neg P)]$

\hspace{12pt}(b) $(P \lor Q) \land (\neg P \lor R)$
\vspace{20pt}

(a)

\centerline{
  \begin{tabular}{c c l}
  P & Q & $\neg [P \land (Q \lor \neg P)]$ \\
  \hline
  F & F & \textbf{T} F F F T T F \\
  F & T & \textbf{T} F F T T T F \\
  T & F & \textbf{T} T F F F F T \\
  T & T & \textbf{F} T T T T F T \\
  \end{tabular}}
\vspace{10pt}

(b)

\centerline{
  \begin{tabular}{c c c l l c}
  P & Q & R & $(P \lor Q)$ & $(\neg P \lor R)$ &  $(P \lor Q) \land (\neg P \lor R)$ \\
  \hline
  F & F & F & F \textbf{F} F & T F \textbf{T} F & F \\
  F & T & F & F \textbf{T} T & T F \textbf{T} F & T \\
  F & F & T & F \textbf{F} F & T F \textbf{T} T & F \\
  F & T & T & F \textbf{T} T & T F \textbf{T} T & T \\
  T & F & F & T \textbf{T} F & F T \textbf{F} F & F \\
  T & T & F & T \textbf{T} T & F T \textbf{F} F & F \\
  T & F & T & T \textbf{T} F & F T \textbf{T} T & T \\
  T & T & T & T \textbf{T} T & F T \textbf{T} T & T \\
  \end{tabular}}
\vspace{30pt}

3. In this exercise we will use the symbol $+$ to mean exclusive or. In other
words, P $+$ Q means "P or Q, but not both."

\hspace{12pt}(a) Make a truth table for P $+$ Q.

\hspace{12pt}(b) Find a formula using only the connectives $\land$, $\lor$, and $\neg$ that is equivalent
to P $+$ Q. Justify your answer with a truth table.
\vspace{20pt}

(a)

\centerline{
  \begin{tabular}{c c l}
  P & Q & $+$ \\
  \hline
  F & F & F \\
  F & T & T \\
  T & F & T \\
  T & T & F \\
  \end{tabular}}
\vspace{10pt}

(b) $$P \land Q \lor \neg P$$

\centerline{
  \begin{tabular}{c c l l c}
  P & Q & $\neg P \land \neg Q$ & $P \land Q$ & $\neg [(\neg P \land \neg Q) \lor (P \land Q)]$\\
  \hline
  F & F & T F \textbf{T} T F & F \textbf{F} F & F \\
  F & T & T F \textbf{F} F T & F \textbf{F} T & T \\
  T & F & F T \textbf{F} T F & T \textbf{F} F & T \\
  T & T & F T \textbf{F} F T & T \textbf{T} T & F \\
  \end{tabular}}
\vspace{40pt}

4. Find a formula using only the connectives $\land$ and $\neg$ that is equivalent to
$P \lor Q$. Justify your answer with a truth table.
\vspace{30pt}

\centerline{
  \begin{tabular}{c c l l}
  P & Q & $P \lor Q$ & $\neg (\neg P \land \neg Q)$ \\
  \hline
  F & F & F & \textbf{F} T F T T F\\
  F & T & T & \textbf{T} T F F F T\\
  T & F & T & \textbf{T} F T F T F\\
  T & T & T & \textbf{T} F T F F T\\
  \end{tabular}}
\vspace{40pt}

5. Some mathematicians use the symbol $\downarrow$ to mean nor.

In other words, P $\downarrow$ Q means "neither P nor Q."

\hspace{12pt}(a) Make a truth table for P $\downarrow$ Q.

\hspace{12pt}(b) Find a formula using only the connectives $\land$, $\lor$, and $\neg$ that is equivalent
to P $\downarrow$ Q.

\hspace{12pt}(c) Find formulas using only the connective $\downarrow$ that are equivalent to $\neg P$,
$P \lor Q$, and $P \land Q$.
\vspace{20pt}

(a)

\centerline{
  \begin{tabular}{c c l}
  P & Q & $\downarrow$ \\
  \hline
  F & F & T \\
  F & T & F \\
  T & F & F \\
  T & T & F \\
  \end{tabular}}
\vspace{10pt}

(b)

\centerline{
  \begin{tabular}{c c l}
  P & Q & $\neg P \land \neg Q$ \\
  \hline
  F & F & T F \textbf{T} T F \\
  F & T & T F \textbf{F} F T \\
  T & F & F T \textbf{F} T F \\
  T & T & F T \textbf{F} F T \\
  \end{tabular}}
\vspace{10pt}

(c)

\centerline{
  \begin{tabular}{c c c}
  P & $\neg P$ & $P \downarrow P$ \\
  \hline
  T & F & F \\
  F & T & T \\
  \end{tabular}}
\vspace{10pt}

\centerline{
  \begin{tabular}{c c c c}
  P & Q & $P \lor Q$ & $(P \downarrow Q) \downarrow (P \downarrow Q)$ \\
  \hline
  F & F & F & F \\
  F & T & T & T \\
  T & F & T & T \\
  T & T & T & T \\
  \end{tabular}}
\vspace{10pt}

\centerline{
  \begin{tabular}{c c c c}
  P & Q & $P \land Q$ & $(P \downarrow P) \downarrow (Q \downarrow Q)$ \\
  \hline
  F & F & F & F \\
  F & T & F & F \\
  T & F & F & F \\
  T & T & T & T \\
  \end{tabular}}
\vspace{40pt}

6. Some mathematicians write $P | Q$ to mean "P and Q are not both true."
(This connective is called nand, and is used in the study of circuits in
computer science.)

\hspace{12pt}(a) Make a truth table for $P | Q$.

\hspace{12pt}(b) Find a formula using only the connectives $\land$, $\lor$, and $\neg$ that is equivalent
to $P | Q$.

\hspace{12pt}(c) Find formulas using only the connective $|$ that are equivalent to $\neg P$,
$P \lor Q$, and $P \land Q$.
\vspace{20pt}

(a)

\centerline{
  \begin{tabular}{c c c}
  P & Q & $P | Q$ \\
  \hline
  F & F & T \\
  F & T & T \\
  T & F & T \\
  T & T & F \\
  \end{tabular}}
\vspace{10pt}

(b)

\centerline{
  \begin{tabular}{c c c}
  P & Q & $\neg (P \land Q$) \\
  \hline
  F & F & T \\
  F & T & T \\
  T & F & T \\
  T & T & F \\
  \end{tabular}}
\vspace{10pt}

(c)

\centerline{
  \begin{tabular}{c c c}
  P & $\neg P$ & $P | P$ \\
  \hline
  T & F & F \\
  F & T & T \\
  \end{tabular}}
\vspace{10pt}

\centerline{
  \begin{tabular}{c c c l}
  P & Q & $P \land Q$ & $(P | Q) | (P | Q)$ \\
  \hline
  F & F & F & F T F \textbf{F} F T F \\
  F & T & F & F T T \textbf{F} F T T \\
  T & F & F & T T F \textbf{F} T T F \\
  T & T & T & T F T \textbf{T} T F T \\
  \end{tabular}}
\vspace{10pt}

\centerline{
  \begin{tabular}{c c c c}
  P & Q & $P \lor Q$ & $(P | P) | (Q | Q)$ \\
  \hline
  F & F & F & F T F \textbf{F} F T F \\
  F & T & T & F T F \textbf{T} T F T \\
  T & F & T & T F T \textbf{T} F T F \\
  T & T & T & T F T \textbf{T} T F T \\
  \end{tabular}}
\vspace{40pt}

7. Use truth tables to determine whether or not the arguments in exercise 7 of Section 1.1 are valid.
\vspace{20pt}

(a) $$\neg (J \land P)$$
$$P \lor C$$
$$J$$
$$\therefore C$$

\centerline{
  \begin{tabular}{c c c l c c c}
  J & P & C & $\neg (J \land P)$ & $P \lor C$ &  $J$ & $C$ \\
  \hline
  F & F & F & \textbf{T} F F F & F \textbf{F} F & F & F \\
  F & T & F & \textbf{T} F F T & T \textbf{T} F & F & F \\
  F & F & T & \textbf{T} F F F & F \textbf{T} T & F & T \\
  F & T & T & \textbf{T} F F T & T \textbf{T} T & F & T \\
  T & F & F & \textbf{T} T F F & F \textbf{F} F & T & F \\
  T & T & F & \textbf{F} T T T & T \textbf{T} F & T & F \\
  T & F & T & \textbf{T} T F F & F \textbf{T} T & T & T \\
  T & T & T & \textbf{F} T T T & T \textbf{T} T & T & T \\
  \end{tabular}}
\vspace{10pt}

All premises are true only in 7 row and conclusion is true as well. Therefore, the argument is valid.
\vspace{10pt}

(c) $$\neg (B \land F)$$
$$\neg (P \land C)$$
$$\neg F \land \neg C$$
$$\therefore \neg B \land \neg P$$


\centerline{
  \begin{tabular}{c c c c l l l l}
  B & F & P & C & $\neg (B \land F)$ & $\neg (P \land C)$ &  $\neg F \land \neg C$ & $\neg B \land \neg P$ \\
  \hline
  F & F & F & F & \textbf{T} F F F & \textbf{T} F F F & T F \textbf{T} T F & T F \textbf{T} T F \\
  F & F & F & T & \textbf{T} F F F & \textbf{T} F F T & T F \textbf{F} F T & T F \textbf{T} T F \\
  F & F & T & F & \textbf{T} F F F & \textbf{T} T F F & T F \textbf{T} T F & T F \textbf{F} F T \\
  F & F & T & T & \textbf{T} F F F & \textbf{F} T T T & T F \textbf{F} F T & T F \textbf{F} F T \\
  F & T & F & F & \textbf{T} F F T & \textbf{T} F F F & F T \textbf{F} T F & T F \textbf{T} T F \\
  F & T & F & T & \textbf{T} F F T & \textbf{T} F F T & F T \textbf{F} F T & T F \textbf{T} T F \\
  F & T & T & F & \textbf{T} F F T & \textbf{T} T F F & F T \textbf{F} T F & T F \textbf{F} F T \\
  F & T & T & T & \textbf{T} F F T & \textbf{F} T T T & F T \textbf{F} F T & T F \textbf{F} F T \\
  T & F & F & F & \textbf{T} T F F & \textbf{T} F F F & T F \textbf{T} T F & F T \textbf{F} T F \\
  T & F & F & T & \textbf{T} T F F & \textbf{T} F F T & T F \textbf{F} F T & F T \textbf{F} T F \\
  T & F & T & F & \textbf{T} T F F & \textbf{T} T F F & T F \textbf{T} T F & F T \textbf{F} F T \\
  T & F & T & T & \textbf{T} T F F & \textbf{F} T T T & T F \textbf{F} F T & F T \textbf{F} F T \\
  T & T & F & F & \textbf{F} T T T & \textbf{T} F F F & F T \textbf{F} T F & F T \textbf{F} T F \\
  T & T & F & T & \textbf{F} T T T & \textbf{T} F F T & F T \textbf{F} F T & F T \textbf{F} T F \\
  T & T & T & F & \textbf{F} T T T & \textbf{T} T F F & F T \textbf{F} T F & F T \textbf{F} F T \\
  T & T & T & T & \textbf{F} T T T & \textbf{F} T T T & F T \textbf{F} F T & F T \textbf{T} F T \\
  \end{tabular}}
\vspace{10pt}

3rd row: $T T T \therefore F$. Therefore, the argument is invalid.
\vspace{10pt}

(c) $$J \lor B$$
$$\neg (S \land B)$$
$$J \lor \neg S$$

\centerline{
  \begin{tabular}{c c c l l l}
  J & B & S & $J \land B$ & $\neg (S \land B)$ &  $J \lor \neg S$ \\
  \hline
  F & F & F & F \textbf{F} F & \textbf{T} F F F & F \textbf{T} T F \\
  F & T & F & F \textbf{F} T & \textbf{T} F F T & F \textbf{T} T F \\
  F & F & T & F \textbf{F} F & \textbf{T} T F F & F \textbf{F} F T \\
  F & T & T & F \textbf{F} T & \textbf{F} T T T & F \textbf{F} F T \\
  T & F & F & T \textbf{F} F & \textbf{T} F F F & T \textbf{T} T F \\
  T & T & F & T \textbf{T} T & \textbf{T} F F T & T \textbf{T} T F \\
  T & F & T & T \textbf{F} F & \textbf{T} T F F & T \textbf{T} F T \\
  T & T & T & T \textbf{T} T & \textbf{F} T T T & T \textbf{T} F T \\
  \end{tabular}}
\vspace{10pt}

All premises are true only in 6 row and conclusion is true as well. Therefore, the argument is valid.

(d) $$(S \land H) \lor (E \land \neg H)$$
$$\neg (S \land E)$$

\centerline{
  \begin{tabular}{c c c l l}
  S & H & E & $(S \land H) \lor (E \land \neg H)$ & $\neg (S \land E)$ \\
  \hline
  F & F & F & F F F \textbf{F} F F T F & \textbf{T} F F F \\
  F & T & F & F F T \textbf{F} F F F T & \textbf{T} F F F \\
  F & F & T & F F F \textbf{T} T T T F & \textbf{T} F F T \\
  F & T & T & F F T \textbf{F} T F F T & \textbf{T} F F T \\
  T & F & F & T F F \textbf{F} F F T F & \textbf{T} T F F \\
  T & T & F & T T T \textbf{T} F F F T & \textbf{T} T F F \\
  T & F & T & T F F \textbf{T} T T T F & \textbf{F} T T T \\
  T & T & T & T T T \textbf{T} T F F T & \textbf{F} T T T \\
  \end{tabular}}
\vspace{10pt}

7th row: $T \therefore F$. Therefore, the argument is invalid.
\vspace{30pt}

8. Use truth tables to determine which of the following formulas are equivalent
to each other:

\hspace{12pt}(a) $(P \land Q) \lor (\neg P \land \neg Q)$

\hspace{12pt}(b) $\neg P \lor Q$

\hspace{12pt}(c) $(P \lor \neg Q) \land (Q \lor \neg P)$

\hspace{12pt}(d) $\neg (P \lor Q)$

\hspace{12pt}(e) $(Q \land P) \lor \neg P$
\vspace{20pt}

(a)

\centerline{
  \begin{tabular}{c c l}
  P & Q & $(P \land Q) \lor (\neg P \land \neg Q)$ \\
  \hline
  F & F & F \textbf{F} F \textbf{T} T F \textbf{T} T F \\
  F & T & F \textbf{F} T \textbf{F} T F \textbf{F} F T \\
  T & F & T \textbf{F} F \textbf{F} F T \textbf{F} T F \\
  T & T & T \textbf{T} T \textbf{T} F T \textbf{F} F T \\
  \end{tabular}}
\vspace{10pt}

(b)

\centerline{
  \begin{tabular}{c c l}
  P & Q & $\neg P \lor Q$ \\
  \hline
  F & F & T F \textbf{T} F \\
  F & T & T F \textbf{T} T \\
  T & F & F T \textbf{F} F \\
  T & T & F T \textbf{T} T \\
  \end{tabular}}
\vspace{10pt}

(c)

\centerline{
  \begin{tabular}{c c l}
  P & Q & $(P \lor \neg Q) \land (Q \lor \neg P)$ \\
  \hline
  F & F & F \textbf{T} T F \textbf{T} F \textbf{T} T F \\
  F & T & F \textbf{F} F T \textbf{F} T \textbf{T} T F \\
  T & F & T \textbf{T} T F \textbf{F} F \textbf{F} F T \\
  T & T & T \textbf{T} F T \textbf{T} T \textbf{T} F T \\
  \end{tabular}}
\vspace{10pt}

(d)

\centerline{
  \begin{tabular}{c c l}
  P & Q & $(Q \land P) \lor \neg P$ \\
  \hline
  F & F & F \textbf{F} F \textbf{T} T F \\
  F & T & T \textbf{F} F \textbf{T} T F \\
  T & F & F \textbf{F} T \textbf{F} F T \\
  T & T & T \textbf{T} T \textbf{T} F T \\
  \end{tabular}}
\vspace{10pt}


(a) is equivalent to (d) and (b) is equivalent to (d).
\vspace{30pt}

9.  Use truth tables to determine which of these statements are tautologies,
which are contradictions, and which are neither:

\hspace{12pt}(a) $(P \lor Q) \land (\neg P \lor \neg Q)$

\hspace{12pt}(b) $(P \lor Q) \land (\neg P \land \neg Q)$

\hspace{12pt}(c) $(P \lor Q) \lor (\neg P \lor \neg Q)$

\hspace{12pt}(d) $[P \land (Q \lor \neg R)] \lor (\neg P \lor R)$
\vspace{20pt}

(a)

\centerline{
  \begin{tabular}{c c l}
  P & Q & $(P \lor Q) \land (\neg P \lor \neg Q)$ \\
  \hline
  F & F & F \textbf{F} F \textbf{F} T F \textbf{T} T F \\
  F & T & F \textbf{T} T \textbf{T} T F \textbf{T} F T \\
  T & F & T \textbf{T} F \textbf{T} F T \textbf{T} T F \\
  T & T & T \textbf{T} T \textbf{F} F T \textbf{F} F T \\
  \end{tabular}}
\vspace{10pt}

(b)

\centerline{
  \begin{tabular}{c c l}
  P & Q & $(P \lor Q) \land (\neg P \land \neg Q)$ \\
  \hline
  F & F & F \textbf{F} F \textbf{F} T F \textbf{T} T F \\
  F & T & F \textbf{T} T \textbf{F} T F \textbf{F} F T \\
  T & F & T \textbf{T} F \textbf{F} F T \textbf{F} T F \\
  T & T & T \textbf{T} T \textbf{F} F T \textbf{F} F T \\
  \end{tabular}}
\vspace{10pt}

(c)

\centerline{
  \begin{tabular}{c c l}
  P & Q & $(P \lor Q) \lor (\neg P \lor \neg Q)$ \\
  \hline
  F & F & F \textbf{F} F \textbf{T} T F \textbf{T} T F \\
  F & T & F \textbf{T} T \textbf{T} T F \textbf{T} F T \\
  T & F & T \textbf{T} F \textbf{T} F T \textbf{T} T F \\
  T & T & T \textbf{T} T \textbf{T} F T \textbf{F} F T \\
  \end{tabular}}
\vspace{10pt}

(d)

\centerline{
  \begin{tabular}{c c c l}
  P & Q & R & $[P \land (Q \lor \neg R)] \lor (\neg P \lor R)$ \\
  \hline
  F & F & F & F \textbf{F} F T T F \textbf{T} T F \textbf{T} F \\
  F & T & F & F \textbf{F} T T T F \textbf{T} T F \textbf{T} F \\
  F & F & T & F \textbf{F} F F F T \textbf{T} T F \textbf{T} T \\
  F & T & T & F \textbf{F} T T F T \textbf{T} T F \textbf{T} T \\
  T & F & F & T \textbf{T} F T T F \textbf{T} F T \textbf{F} F \\
  T & T & F & T \textbf{T} T T T F \textbf{T} F T \textbf{F} F \\
  T & F & T & T \textbf{F} F F F T \textbf{T} F T \textbf{T} T \\
  T & T & T & T \textbf{T} T T F T \textbf{T} F T \textbf{T} T \\
  \end{tabular}}
\vspace{10pt}

(b) is contradiction and (c), (d) are tautologies.

\vspace{30pt}

10. Use truth tables to check these laws:

\hspace{12pt}(a) The second DeMorgan’s law. (The first was checked in the text.)

\hspace{12pt}(b) The distributive laws.

Done, at the beginning of chapter.

\vspace{30pt}

11. Use the laws stated in the text to find simpler formulas equivalent to these
formulas. (See Examples 1.2.5 and 1.2.7.)

\hspace{12pt}(a) $\neg(\neg P \land \neg Q)$

\hspace{12pt}(b) $(P \land Q) \lor (P \land \neg Q)$

\hspace{12pt}(c) $\neg (P \land \neg Q) \lor (\neg P \land Q)$
\vspace{20pt}

(a) $$\neg(\neg P \land \neg Q)$$
DeMorgan's law: $$\neg \neg P \lor \neg \neg Q$$
Double Negation law: $$P \lor Q$$
\vspace{10pt}

(b) $$(P \land Q) \lor (P \land \neg Q)$$
Distributive law: $$P \land (Q \lor \neg Q)$$
\centerline{$P \land$ (a tautology)}

Tautology law: $$P$$
\vspace{10pt}

(c) $$\neg (P \land \neg Q) \lor (\neg P \land Q)$$
DeMorgan's law: $$(\neg P \lor \neg \neg Q) \lor (\neg P \land Q)$$
Double Negation law: $$(\neg P \lor Q) \lor (\neg P \land Q)$$
$$\neg P \lor Q \lor (\neg P \land Q)$$
Commutative law: $$\neg P \lor Q \lor (Q \land \neg P)$$
Absoption law: $$\neg P \lor Q$$

\vspace{30pt}

12. Use the laws stated in the text to find simpler formulas equivalent to these
formulas. (See Examples 1.2.5 and 1.2.7.)

\hspace{12pt}(a) $\neg (\neg P \lor Q) \lor (P \land \neg R)$

\hspace{12pt}(b) $\neg (\neg P \land Q) \lor (P \land \neg R)$

\hspace{12pt}(c) $(P \land R) \lor [\neg R \land (P \lor Q)]$
\vspace{10pt}

(a) $$\neg (\neg P \lor Q) \lor (P \land \neg R)$$
DeMorgan's law: $$(P \land \neg Q) \lor (P \land \neg R)$$
Distributive law: $$P \land (\neg Q \lor \neg R)$$
DeMorgan's law: $$P \land \neg (Q \land R)$$
\vspace{10pt}

(b) $$\neg (\neg P \land Q) \lor (P \land \neg R)$$
DeMorgan's law: $$P \lor \neg Q \lor (P \land \neg R)$$
Commutative law: $$\neg Q \lor P \lor (P \land \neg R)$$
Absoption law: $$\neg Q \lor P$$
\vspace{10pt}

(c) $$(P \land R) \lor [\neg R \land (P \lor Q)]$$
Distributive law: $$(P \land R) \lor [(\neg R \land P) \lor (\neg R \land Q)]$$
Associative law and Commutative law: $$(P \land R) \lor (P \land \neg R) \lor (Q \land \neg R)$$
Distributive law: $$P \land (R \lor \neg R) \lor (Q \land \neg R)$$
Tautology law: $$P \lor (Q \land \neg R)$$
\vspace{30pt}

13. Use the first DeMorgan’s law and the double negation law to derive the
second DeMorgan’s law.

1st: $$\neg (P \land Q) = \neg P \lor \neg Q$$

2st: $$\neg (P \lor Q) = \neg P \land \neg Q$$
Add negation 2 both parts: $$\neg \neg (P \lor Q) = \neg (\neg P \land \neg Q)$$
Use double negation law: $$P \lor Q = \neg (\neg P \land \neg Q)$$
Use 1st DeMorgan's law: $$P \lor Q = \neg \neg P \lor \neg \neg Q$$
Use double negation law: $$P \lor Q = P \lor Q$$


\vspace{30pt}

14. Note that the associative laws say only that parentheses are unnecessary
when combining three statements with $\land$ or $\lor$. In fact, these laws can be
used to justify leaving parentheses out when more than three statements
are combined. Use associative laws to show that $[P \land (Q \land R)] \land S$ is
equivalent to $(P \land Q) \land (R \land S)$

$[P \land (Q \land R)] \land S = [(P \land Q) \land R] \land S = (P \land Q) \land (R \land S)$

\vspace{30pt}

15. How many lines will there be in the truth table for a statement containing
n letters?

$2^n$ lines

\vspace{30pt}

16. Find a formula involving the connectives $\land$, $\lor$, and $\neg$ that has the following
truth table:

\centerline{
  \begin{tabular}{c c l}
  P & Q & $ P \lor \neg Q$ \\
  \hline
  F & F & T \\
  F & T & F \\
  T & F & T \\
  T & T & T \\
  \end{tabular}}
\vspace{30pt}

17. Find a formula involving the connectives $\land$, $\lor$, and $\neg$ that has the following
truth table:

\centerline{
  \begin{tabular}{c c l}
  P & Q & $\neg (P \land Q) \land (P \lor Q)$ \\
  \hline
  F & F & F \\
  F & T & T \\
  T & F & T \\
  T & T & F \\
  \end{tabular}}
\vspace{30pt}

18. Suppose the conclusion of an argument is a tautology. What can you
conclude about the validity of the argument? What if the conclusion is
a contradiction? What if one of the premises is either a tautology or a
contradiction?
\vspace{20pt}

If a conclusion is a tautology then the argument is always valid.
If a conclusion is a contradiction then the argument is always invalid.
If one of the premises is a tautology then this premise may be ignored.
If one of the premises is a contradiction then the argument is invalid.
\vspace{50pt}

\textbf{1.3. Variables and Sets}
\vspace{40pt}

1. Analyze the logical forms of the following statements:

\hspace{12pt}(a) 3 is a common divisor of 6, 9, and 15. (Note: You did this in exercise
2 of Section 1.1, but you should be able to give a better answer now.)

\hspace{12pt}(b) x is divisible by both 2 and 3 but not 4.

\hspace{12pt}(c) x and y are natural numbers, and exactly one of them is prime.
\vspace{20pt}

(a) Let D(x) stand for "x is disible by 3" then
$$D(6) \land D(9) \land D(15)$$
\vspace{10pt}

(b) Let D(y, x) stand for "x is divisible by y" then
$$D(2, x) \land D(3, x) \land \neg D(4, x)$$
\vspace{10pt}

(c) Let N(x) stand for "x is natural" and P(x) stand for "x is prime".
\vspace{10pt}

$$N(x) \land N(y) \land [(P(x) \land \neg P(y)) \lor (\neg P(x) \land P(y))]$$
\vspace{30pt}

2.Analyze the logical forms of the following statements:

\hspace{12pt}(a) x and y are men, and either x is taller than y or y is taller than x.

\hspace{12pt}(b) Either x or y has brown eyes, and either x or y has red hair.

\hspace{12pt}(c) Either x or y has both brown eyes and red hair.
\vspace{20pt}

(a) Let P(x) stand for "x is a man" and H(x, y) is "x is taller than y" then

$$P(x) \land P(y) \land [H(x, y) \lor H(y, x)]$$
\vspace{10pt}

(b) Let P(x) stand for "x has brown eyes" and H(x) is "x has red hair" then

$$[P(x) \lor P(y)] \lor [H(x) \lor H(y)]$$
\vspace{10pt}

(c) P(x) stand for "x has brown eyes and red hair" then

$$P(x) \lor P(y)$$
\vspace{30pt}

3. Write definitions using elementhood tests for the following sets:

\hspace{12pt}(a) \{Mercury, Venus, Earth, Mars, Jupiter, Saturn, Uranus, Neptune,
Pluto\}.

\hspace{12pt}(b) \{Brown, Columbia, Cornell, Dartmouth, Harvard, Princeton, University
of Pennsylvania, Yale\}.

\hspace{12pt}(c) \{Alabama, Alaska, Arizona, ... , Wisconsin, Wyoming\}.

\hspace{12pt}(d) \{Alberta, British Columbia, Manitoba, New Brunswick, Newfoundland
and Labrador, Northwest Territories, Nova Scotia, Nunavut, Ontario,
Prince Edward Island, Quebec, Saskatchewan, Yukon\}.
\vspace{20pt}

(a) $P = \{ p \mid \text{p is a planet} \}$
\vspace{10pt}

(b) $U = \{ p \mid \text{p is an Ivy League school} \}$
\vspace{10pt}

(c) $U = \{ p \mid \text{p is a state in the USA} \}$
\vspace{10pt}

(c) $C = \{ p \mid \text{p is a province of territory in Canada} \}$
\vspace{30pt}

4. Write definitions using elementhood tests for the following sets:

\hspace{12pt}(a) \{1, 4, 9, 16, 25, 36, 49,...\}.

\hspace{12pt}(b) \{1, 2, 4, 8, 16, 32, 64,...\}.

\hspace{12pt}(c) \{10, 11, 12, 13, 14, 15, 16, 17, 18, 19\}.
\vspace{20pt}

(a) $$A = \{ x \in \mathbb{N} \mid x > 0 \}$$
$$B = \{ y \mid y = x^2, x \in A \}$$
\vspace{10pt}

(b) $$A = \{ y \mid y = 2^n, n \in \mathbb{N}\}$$
\vspace{10pt}

(c) $$A = \{x \in \mathbb{N} \mid x >= 10 \}$$
$$B = \{x \in A \mid x < 19 \}$$
\vspace{30pt}

5. Simplify the following statements. Which variables are free and which are
bound? If the statement has no free variables, say whether it is true or
false.

\hspace{12pt}(a) $-3 \in \{x \in \mathbb{R} \mid 13 - 2x > 1 \}$

\hspace{12pt}(b) $4 \in \{x \in \mathbb{R}^- \mid 13 - 2x > 1 \}$

\hspace{12pt}(c) $5 \notin \{x \in \mathbb{R} \mid 13 - 2x > c\}$
\vspace{20pt}

(a) $(-3 \in \mathbb{R}) \land ((13+6) > 1)$ No free variables; x - bound; True.
\vspace{10pt}

(b) $(4 \in \mathbb{R}) \land (4 < 0) \land ((13 - 8) > 1)$ No free variables; x - bound; False.
\vspace{10pt}

(c) $\neg [(5 \in \mathbb{R}) \land ((13 - 10) > c)]$ c - free variable; x - bound.
\vspace{30pt}

6. Simplify the following statements. Which variables are free and which are
bound? If the statement has no free variables, say whether it is true or
false.

\hspace{12pt}(a) $\omega \in \{x \in \mathbb{R} \mid 13 - 2x > c\}$

\hspace{12pt}(b) $4 \in \{x \in \mathbb{R} \mid 13 - 2x \in \{y \mid y \text{ is a prime number}\}\}$. (It might make
this statement easier to read if we let $P = \{y \mid y \text{ is a prime number}\}$;
using this notation, we could rewrite the statement as $4 \in \{x \in \mathbb{R} \mid 13 - 2x \in P\}$.)

\hspace{12pt}(c) $4 \in \{x \in \{y \mid y \text{ is a prime number}\} \mid 13 - 2x > 1\}$. (Using the same notation
as in part (b), we could write this as $4 \in \{x \in P \mid 13 - 2x > 1\}$.)
\vspace{20pt}

(a) $(\omega \in \mathbb{R}) \land (13 - 2\omega > c)$ x - bound; $\omega$ and c are free variables.
\vspace{10pt}

(b) $(4 \in \mathbb{R}) \land ((13 - 8) \text{ is prime})$ True; no free variables; x, y - bound variables.
\vspace{10pt}

(c) $(4 \text{ is prime}) \land (5 > 1)$ False; no free variables; x, y - bound variables.
\vspace{30pt}

7. What are the truth sets of the following statements? List a few elements of
the truth set if you can.

\hspace{12pt}(a) Elizabeth Taylor was once married to x.

\hspace{12pt}(b) x is a logical connective studied in Section 1.1.

\hspace{12pt}(c) x is the author of this book.
\vspace{20pt}

(a) $\{x \mid \text{Elizabeth Taylor was once married on x}\}$ = \{Conrad Hilton, Michael Wilding, Mike Todd ...\}
\vspace{10pt}

(b) $\{x \mid x \text{ is logical connective studied in Section 1.1}\} = \{\lor, \land \neg\}$

(c) $\{x \mid x \text{ is author of this book}\}$ = \{Daniel J. Velleman\}
\vspace{30pt}

8. What are the truth sets of the following statements? List a few elements of
the truth set if you can.

\hspace{12pt}(a) $x$ is a real number and $x^2 - 4x + 3 = 0$.

\hspace{12pt}(b) $x$ is a real number and $x^2 - 2x + 3 = 0$.

\hspace{12pt}(c) $x$ is a real number and $5 \in \{y \in \mathbb{R} \mid x^2 + y^2 < 50\}$.
\vspace{20pt}

(a)

Vieta's formulas

$$x * y = 3$$
$$x + y = -4/1$$

$$\therefore \{1, 3\}$$

\vspace{10pt}

(b)

$$D = b^2 - 4ac = (-2)^2 - 4*1*3 = 4 - 12 = -8$$

$$\therefore \emptyset$$

\vspace{10pt}

(c)

$$(5 \in \mathbb{R}) \land (x^2 + 25 < 50)$$
$$x^2 < 25$$
$$\therefore \{x \in \mathbb{R} \mid x^2 < 25\}$$
$$e.g \{-4.9, -3, 0, 3, 4.9, \dots\}$$
\vspace{50pt}

\textbf{1.4. Operations on Sets}
\vspace{40pt}























\end{document}
