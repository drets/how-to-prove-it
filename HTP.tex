\documentclass{article}
\usepackage{mathtools}
\usepackage{xcolor}
\usepackage{listings}
\renewcommand{\baselinestretch}{1.5}
\lstset{
  frame=none,
  xleftmargin=2pt,
  stepnumber=1,
  numbers=left,
  numbersep=5pt,
  numberstyle=\ttfamily\tiny\color[gray]{0.3},
  belowcaptionskip=\bigskipamount,
  captionpos=b,
  escapeinside={*'}{'*},
  language=haskell,
  tabsize=2,
  emphstyle={\bf},
  commentstyle=\it,
  stringstyle=\mdseries\rmfamily,
  showspaces=false,
  keywordstyle=\bfseries\rmfamily,
  columns=flexible,
  basicstyle=\small\sffamily,
  showstringspaces=false,
  morecomment=[l]\%,
}
\begin{document}
\topskip0pt
\vspace*{\fill}
\centerline{\sc \large Solutions for the exercises from "How to prove it" book }
\vspace*{\fill}
%
\pagebreak
\centerline{\sc \large 0. Inroduction}
\vspace{50pt}

1. (a) Factor $2^{15} - 1 = 32,767$ into a product of two smaller positive integers.

\hspace{12pt}(b) Find an integer $x$ such that $1 < x < 2^{32767} − 1$ and $2^{32767} - 1$ is divisible by x.
\vspace{20pt}


(a) $n = 15$. Since $3 * 5 = 12$, we could use the values $a = 3$ and $b = 5$.
$x = 2^b - 1 = 2^5 - 1 = 31$ and $y = 1 + 2^b + 2^{2*b} + \dotso + 2^{(a-1)b} = 1 + 2^5 + 2^{10} = 1 + 32 + 1024 = 1057$,
so $1057 * 31 = 32,767$
\vspace{10pt}

(b) $n=32767$. Since $7 * 4681 = 32767$, we could use the values $a = 4681$ and $b = 7$.
$x = 2^7 - 1 = 127$
\vspace{40pt}

2. Make some conjectures about the values of $n$ for which $3*n - 1$ is prime or
the values of n for which $3*n - 2*n$ is prime. (You might start by making a
table similar to Figure 1.)
\vspace{20pt}

The following simple Haskell program:
\lstinputlisting[language=Haskell]{0-2.hs}
gives the output:

\begin{verbatim}
"2. 3^n - 1: not prime | 3^n - 2^n: prime"
"3. 3^n - 1: not prime | 3^n - 2^n: prime"
"4. 3^n - 1: not prime | 3^n - 2^n: not prime"
"5. 3^n - 1: not prime | 3^n - 2^n: prime"
"6. 3^n - 1: not prime | 3^n - 2^n: not prime"
"7. 3^n - 1: not prime | 3^n - 2^n: not prime"
"8. 3^n - 1: not prime | 3^n - 2^n: not prime"
"9. 3^n - 1: not prime | 3^n - 2^n: not prime"
"10. 3^n - 1: not prime | 3^n - 2^n: not prime"
"11. 3^n - 1: not prime | 3^n - 2^n: not prime"
"12. 3^n - 1: not prime | 3^n - 2^n: not prime"
"13. 3^n - 1: not prime | 3^n - 2^n: not prime"
"14. 3^n - 1: not prime | 3^n - 2^n: not prime"
"15. 3^n - 1: not prime | 3^n - 2^n: not prime"
"16. 3^n - 1: not prime | 3^n - 2^n: not prime"
"17. 3^n - 1: not prime | 3^n - 2^n: prime"
"18. 3^n - 1: not prime | 3^n - 2^n: not prime"
"19. 3^n - 1: not prime | 3^n - 2^n: not prime"
"20. 3^n - 1: not prime | 3^n - 2^n: not prime"
"21. 3^n - 1: not prime | 3^n - 2^n: not prime"
"22. 3^n - 1: not prime | 3^n - 2^n: not prime"
"23. 3^n - 1: not prime | 3^n - 2^n: not prime"
"24. 3^n - 1: not prime | 3^n - 2^n: not prime"
"25. 3^n - 1: not prime | 3^n - 2^n: not prime"
"26. 3^n - 1: not prime | 3^n - 2^n: not prime"
"27. 3^n - 1: not prime | 3^n - 2^n: not prime"
"28. 3^n - 1: not prime | 3^n - 2^n: not prime"
"29. 3^n - 1: not prime | 3^n - 2^n: prime"
"30. 3^n - 1: not prime | 3^n - 2^n: not prime"
"31. 3^n - 1: not prime | 3^n - 2^n: prime"
"32. 3^n - 1: not prime | 3^n - 2^n: not prime"
"33. 3^n - 1: not prime | 3^n - 2^n: not prime"
"34. 3^n - 1: not prime | 3^n - 2^n: not prime"
"35. 3^n - 1: not prime | 3^n - 2^n: not prime"
"36. 3^n - 1: not prime | 3^n - 2^n: not prime"
"37. 3^n - 1: not prime | 3^n - 2^n: not prime"
"38. 3^n - 1: not prime | 3^n - 2^n: not prime"
"39. 3^n - 1: not prime | 3^n - 2^n: not prime"
"40. 3^n - 1: not prime | 3^n - 2^n: not prime"
"41. 3^n - 1: not prime | 3^n - 2^n: not prime"
"42. 3^n - 1: not prime | 3^n - 2^n: not prime"
"43. 3^n - 1: not prime | 3^n - 2^n: not prime"
"44. 3^n - 1: not prime | 3^n - 2^n: not prime"
"45. 3^n - 1: not prime | 3^n - 2^n: not prime"
"46. 3^n - 1: not prime | 3^n - 2^n: not prime"
"47. 3^n - 1: not prime | 3^n - 2^n: not prime"
"48. 3^n - 1: not prime | 3^n - 2^n: not prime"
"49. 3^n - 1: not prime | 3^n - 2^n: not prime"
"50. 3^n - 1: not prime | 3^n - 2^n: not prime"
\end{verbatim}
Conjecture 1: $3^n - 1$ doesn't contain prime numbers.
\vspace{40pt}

3. The proof of Theorem 3 gives a method for finding a prime number different
from any in a given list of prime numbers.

(a) Use this method to find a prime different from $2$, $3$, $5$, and $7$.

(b) Use this method to find a prime different from $2$, $5$, and $11$.
\vspace{20pt}

(a) $2*3*5*7+1 = 211$
\vspace{10pt}

(b) $$1*2+1=3$$
    $$11*2+1=23$$
\vspace{40pt}

4. Find five consecutive integers that are not prime.
\vspace{20pt}

$764, 765, 766, 767, 768$
\vspace{40pt}

5. Use the table in Figure 1 and the discussion on p. 5 to find two more perfect
numbers.
\vspace{20pt}

1) $n = 5$, $2^n-1 = 2^5-1=31$ (prime); Using the formula for perfect number $2^{n-1}*(2^n-1) = 2^4*(2^5-1) = 16*31 = 496$
\vspace{10pt}

2) $n = 7$, $2^7-1= 128 - 1 = 127$ (prime); Calculating prime: $2^6*(2^7-1) = 64*127 = 8128$
\vspace{40pt}

6. The sequence $3, 5, 7$ is a list of three prime numbers such that each pair of
adjacent numbers in the list differ by two. Are there any more such “triplet
primes”?
\vspace{20pt}

\lstinputlisting[language=Haskell]{0-6.hs}

Using the program above I couldn't find another “triplet”. I alse tried the case when each pair of adjacent numbers in the list differ by three.
\pagebreak

\centerline{\sc \large 1. Sentential Logic}
\vspace{50pt}



























\end{document}
