\documentclass{article}
\usepackage{mathtools}
\usepackage{xcolor}
\usepackage{listings}
\usepackage{amssymb}
\renewcommand{\baselinestretch}{1.5}
\lstset{
  frame=none,
  xleftmargin=2pt,
  stepnumber=1,
  numbers=left,
  numbersep=5pt,
  numberstyle=\ttfamily\tiny\color[gray]{0.3},
  belowcaptionskip=\bigskipamount,
  captionpos=b,
  escapeinside={*'}{'*},
  language=haskell,
  tabsize=2,
  emphstyle={\bf},
  commentstyle=\it,
  stringstyle=\mdseries\rmfamily,
  showspaces=false,
  keywordstyle=\bfseries\rmfamily,
  columns=flexible,
  basicstyle=\small\sffamily,
  showstringspaces=false,
  morecomment=[l]\%,
}
\begin{document}
\topskip0pt
\vspace*{\fill}
\centerline{\sc \large Solutions for the exercises from "How to prove it" book }
\vspace*{\fill}
%
\pagebreak
\centerline{\sc \large 0. Inroduction}
\vspace{50pt}

1. (a) Factor $2^{15} - 1 = 32,767$ into a product of two smaller positive integers.

\hspace{12pt}(b) Find an integer $x$ such that $1 < x < 2^{32767} − 1$ and $2^{32767} - 1$ is divisible by x.
\vspace{20pt}


(a) $n = 15$. Since $3 * 5 = 12$, we could use the values $a = 3$ and $b = 5$.
$x = 2^b - 1 = 2^5 - 1 = 31$ and $y = 1 + 2^b + 2^{2*b} + \dotso + 2^{(a-1)b} = 1 + 2^5 + 2^{10} = 1 + 32 + 1024 = 1057$,
so $1057 * 31 = 32,767$
\vspace{10pt}

(b) $n=32767$. Since $7 * 4681 = 32767$, we could use the values $a = 4681$ and $b = 7$.
$x = 2^7 - 1 = 127$
\vspace{40pt}

2. Make some conjectures about the values of $n$ for which $3*n - 1$ is prime or
the values of n for which $3*n - 2*n$ is prime. (You might start by making a
table similar to Figure 1.)
\vspace{20pt}

The following simple Haskell program:
\lstinputlisting[language=Haskell]{0-2.hs}
gives the output:

\begin{verbatim}
"2. 3^n - 1: not prime | 3^n - 2^n: prime"
"3. 3^n - 1: not prime | 3^n - 2^n: prime"
"4. 3^n - 1: not prime | 3^n - 2^n: not prime"
"5. 3^n - 1: not prime | 3^n - 2^n: prime"
"6. 3^n - 1: not prime | 3^n - 2^n: not prime"
"7. 3^n - 1: not prime | 3^n - 2^n: not prime"
"8. 3^n - 1: not prime | 3^n - 2^n: not prime"
"9. 3^n - 1: not prime | 3^n - 2^n: not prime"
"10. 3^n - 1: not prime | 3^n - 2^n: not prime"
"11. 3^n - 1: not prime | 3^n - 2^n: not prime"
"12. 3^n - 1: not prime | 3^n - 2^n: not prime"
"13. 3^n - 1: not prime | 3^n - 2^n: not prime"
"14. 3^n - 1: not prime | 3^n - 2^n: not prime"
"15. 3^n - 1: not prime | 3^n - 2^n: not prime"
"16. 3^n - 1: not prime | 3^n - 2^n: not prime"
"17. 3^n - 1: not prime | 3^n - 2^n: prime"
"18. 3^n - 1: not prime | 3^n - 2^n: not prime"
"19. 3^n - 1: not prime | 3^n - 2^n: not prime"
"20. 3^n - 1: not prime | 3^n - 2^n: not prime"
"21. 3^n - 1: not prime | 3^n - 2^n: not prime"
"22. 3^n - 1: not prime | 3^n - 2^n: not prime"
"23. 3^n - 1: not prime | 3^n - 2^n: not prime"
"24. 3^n - 1: not prime | 3^n - 2^n: not prime"
"25. 3^n - 1: not prime | 3^n - 2^n: not prime"
"26. 3^n - 1: not prime | 3^n - 2^n: not prime"
"27. 3^n - 1: not prime | 3^n - 2^n: not prime"
"28. 3^n - 1: not prime | 3^n - 2^n: not prime"
"29. 3^n - 1: not prime | 3^n - 2^n: prime"
"30. 3^n - 1: not prime | 3^n - 2^n: not prime"
"31. 3^n - 1: not prime | 3^n - 2^n: prime"
"32. 3^n - 1: not prime | 3^n - 2^n: not prime"
"33. 3^n - 1: not prime | 3^n - 2^n: not prime"
"34. 3^n - 1: not prime | 3^n - 2^n: not prime"
"35. 3^n - 1: not prime | 3^n - 2^n: not prime"
"36. 3^n - 1: not prime | 3^n - 2^n: not prime"
"37. 3^n - 1: not prime | 3^n - 2^n: not prime"
"38. 3^n - 1: not prime | 3^n - 2^n: not prime"
"39. 3^n - 1: not prime | 3^n - 2^n: not prime"
"40. 3^n - 1: not prime | 3^n - 2^n: not prime"
"41. 3^n - 1: not prime | 3^n - 2^n: not prime"
"42. 3^n - 1: not prime | 3^n - 2^n: not prime"
"43. 3^n - 1: not prime | 3^n - 2^n: not prime"
"44. 3^n - 1: not prime | 3^n - 2^n: not prime"
"45. 3^n - 1: not prime | 3^n - 2^n: not prime"
"46. 3^n - 1: not prime | 3^n - 2^n: not prime"
"47. 3^n - 1: not prime | 3^n - 2^n: not prime"
"48. 3^n - 1: not prime | 3^n - 2^n: not prime"
"49. 3^n - 1: not prime | 3^n - 2^n: not prime"
"50. 3^n - 1: not prime | 3^n - 2^n: not prime"
\end{verbatim}
Conjecture 1: $3^n - 1$ doesn't contain prime numbers.
\vspace{40pt}

3. The proof of Theorem 3 gives a method for finding a prime number different
from any in a given list of prime numbers.

(a) Use this method to find a prime different from $2$, $3$, $5$, and $7$.

(b) Use this method to find a prime different from $2$, $5$, and $11$.
\vspace{20pt}

(a) $2*3*5*7+1 = 211$
\vspace{10pt}

(b) $$1*2+1=3$$
    $$11*2+1=23$$
\vspace{40pt}

4. Find five consecutive integers that are not prime.
\vspace{20pt}

$764, 765, 766, 767, 768$
\vspace{40pt}

5. Use the table in Figure 1 and the discussion on p. 5 to find two more perfect
numbers.
\vspace{20pt}

1) $n = 5$, $2^n-1 = 2^5-1=31$ (prime); Using the formula for perfect number $2^{n-1}*(2^n-1) = 2^4*(2^5-1) = 16*31 = 496$
\vspace{10pt}

2) $n = 7$, $2^7-1= 128 - 1 = 127$ (prime); Calculating prime: $2^6*(2^7-1) = 64*127 = 8128$
\vspace{40pt}

6. The sequence $3, 5, 7$ is a list of three prime numbers such that each pair of
adjacent numbers in the list differ by two. Are there any more such “triplet
primes”?
\vspace{20pt}

\lstinputlisting[language=Haskell]{0-6.hs}

Using the program above I couldn't find another “triplet”. I alse tried the case when each pair of adjacent numbers in the list differ by three.
\pagebreak

\centerline{\sc \large 1. Sentential Logic}
\vspace{50pt}

\textbf{1.1. Deductive Reasoning and Logical Connectives}
\vspace{40pt}

1. Analyze the logical forms of the following statements:

\hspace{12pt}(a) We'll have either a reading assignment or homework problems, but we
won't have both homework problems and a test.

\hspace{12pt}(b) You won't go skiing, or you will and there won't be any snow.

\hspace{12pt}(c) $\sqrt{7} \nleq 2$.
\vspace{20pt}

(a) Let P be "we have reading assignment" and Q be "we have homework problems", then

$$(P \lor Q) \lor \neg (P \land Q)$$
\vspace{10pt}

(b) $$\neg Q \lor (Q \land \neg R)$$
\vspace{10pt}

(c) $$\neg [(\sqrt{7} < 2) \lor (\sqrt{7} = 2)]$$
\vspace{40pt}

2. Analyze the logical forms of the following statements:

\hspace{12pt}(a) Either John and Bill are both telling the truth, or neither of them is.

\hspace{12pt}(b) I’ll have either fish or chicken, but I won’t have both fish and mashed
potatoes.

\hspace{12pt}(c) 3 is a common divisor of 6, 9, and 15
\vspace{20pt}

(a) let A be "John tells the truth" and B be "Bill tells the truth"
$$(A \land B) \lor \neg (A \land B)$$
\vspace{10pt}

(b) $$(A \lor B) \land \neg (A \land C)$$
\vspace{10pt}

(c) $$A \land B \land C$$
\vspace{40pt}

3. Analyze the logical forms of the following statements:

\hspace{12pt}(a) Alice and Bob are not both in the room.

\hspace{12pt}(b) Alice and Bob are both not in the room.

\hspace{12pt}(c) Either Alice or Bob is not in the room.

\hspace{12pt}(d) Neither Alice nor Bob is in the room.
\vspace{20pt}

(a) $$(\neg A \land B) \lor (A \land \neg B) \lor (\neg A \land \neg B)$$
\vspace{10pt}

(b) $$\neg (A \land B)$$
\vspace{10pt}

(c) $$(\neg A \land B) \lor (A \land \neg B)$$
\vspace{10pt}

(d) $$\neg A \land \neg B$$
\vspace{40pt}

4. Which of the following expressions are well-formed formulas?

\hspace{12pt}(a) $\neg (\neg P \lor \neg \neg R)$

\hspace{12pt}(b) $\neg (P, Q, \neg R)$

\hspace{12pt}(c) $P \land \neg P$

\hspace{12pt}(d) $(P \land Q)(P \lor R)$
\vspace{20pt}

\hspace{12pt}(c) is well-formed formula.
\vspace{40pt}

5. Let P stand for the statement "I will buy the pants" and S for the statement
"I will buy the shirt." What English sentences are represented by the following
expressions?

\hspace{12pt}(a) $\neg (P \land \neg S)$

\hspace{12pt}(b) $\neg P \land \neg S$.

\hspace{12pt}(c) $\neg P \lor \neg S$
\vspace{20pt}

(a) I won't buy the pants without the shirt.
\vspace{10pt}

(b) Neither I will buy the pants nor I will buy the shirt.
\vspace{10pt}

(c) Either I won't buy the pants or I won't buy the shirt.
\vspace{40pt}

6. Let S stand for the statement "Steve is happy" and G for "George is happy."
What English sentences are represented by the following expressions?

\hspace{12pt}(a) $(S \lor G) \land (\neg S \lor \neg G)$.

\hspace{12pt}(b) $[S \lor (G \land \neg S)] \lor \neg G$

\hspace{12pt}(c) $S \lor [G \land (\neg S \lor \neg G)]$
\vspace{20pt}

(a) Steve or George are happy, but either Steve is not happy or George.

(b) Steve is happy or George is happy, but Steve not; or George is not happy.

(c) Either Steve is happy or George is happy, but Steve is not happy or George.
\vspace{40pt}

7. Identify the premises and conclusions of the following deductive arguments
and analyze their logical forms. Do you think the reasoning is valid?
(Although you will have only your intuition to guide you in answering
this last question, in the next section we will develop some techniques for
determining the validity of arguments.)

\hspace{12pt}(a) Jane and Pete won’t both win the math prize. Pete will win either
the math prize or the chemistry prize. Jane will win the math prize.
Therefore, Pete will win the chemistry prize.

\hspace{12pt}(b) The main course will be either beef or fish. The vegetable will be either
peas or corn. We will not have both fish as a main course and corn as a
vegetable. Therefore, we will not have both beef as a main course and
peas as a vegetable.

\hspace{12pt}(c) Either John or Bill is telling the truth. Either Sam or Bill is lying.
Therefore, either John is telling the truth or Sam is lying.

\hspace{12pt}(d) Either sales will go up and the boss will be happy, or expenses will go
up and the boss won’t be happy. Therefore, sales and expenses will not
both go up.
\vspace{20pt}

(a) $$(\neg J \land P) \lor (J \land \neg P)$$
$$P \lor C$$
$$J$$
$$C$$

Valid

\vspace{20pt}

(b) $$(B \land \neg F) \lor (\neg B \land F)$$
$$(P \land \neg C) \lor (\neg P \land C)$$
$$\neg (F \land C)$$
$$\neg (B \land P)$$
Invalid

\vspace{20pt}

(c) $$(J \land \neg B) \lor (\neg J \land B)$$
$$\neg S \lor \neg B$$
$$J \lor \neg S$$
Valid

\vspace{20pt}

(d) $$(S \land H) \lor (E \land \neg H)$$
$$\neg (S \land E)$$

Valid

\vspace{50pt}

\textbf{1.2. Truth Tables}

\vspace{40pt}

























\end{document}
