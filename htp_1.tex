\documentclass{article}
\usepackage{mathtools}
\usepackage{xcolor}
\usepackage{listings}
\usepackage{amssymb}
\usepackage{tikz}
\usepackage{soul}
\usetikzlibrary{shapes,backgrounds}
\renewcommand{\baselinestretch}{1.5}
\lstset{
  frame=none,
  xleftmargin=2pt,
  stepnumber=1,
  numbers=left,
  numbersep=5pt,
  numberstyle=\ttfamily\tiny\color[gray]{0.3},
  belowcaptionskip=\bigskipamount,
  captionpos=b,
  escapeinside={*'}{'*},
  language=haskell,
  tabsize=2,
  emphstyle={\bf},
  commentstyle=\it,
  stringstyle=\mdseries\rmfamily,
  showspaces=false,
  keywordstyle=\bfseries\rmfamily,
  columns=flexible,
  basicstyle=\small\sffamily,
  showstringspaces=false,
  morecomment=[l]\%,
}
\begin{document}
\topskip0pt
\vspace*{\fill}
\centerline{\sc \large Solutions of the exercises for "How to prove it" book }
\centerline{by drets}
\centerline{\textit{(may contain various errors)}}
\vspace*{\fill}
%
\pagebreak
\centerline{\sc \large 0. Inroduction}
\vspace{50pt}

1. (a) Factor $2^{15} - 1 = 32,767$ into a product of two smaller positive integers.

\hspace{12pt}(b) Find an integer $x$ such that $1 < x < 2^{32767} − 1$ and $2^{32767} - 1$ is divisible by x.
\vspace{20pt}


(a) $n = 15$. Since $3 * 5 = 12$, we could use the values $a = 3$ and $b = 5$.
$x = 2^b - 1 = 2^5 - 1 = 31$ and $y = 1 + 2^b + 2^{2*b} + \dotso + 2^{(a-1)b} = 1 + 2^5 + 2^{10} = 1 + 32 + 1024 = 1057$,
so $1057 * 31 = 32,767$
\vspace{10pt}

(b) $n=32767$. Since $7 * 4681 = 32767$, we could use the values $a = 4681$ and $b = 7$.
$x = 2^7 - 1 = 127$
\vspace{40pt}

2. Make some conjectures about the values of $n$ for which $3*n - 1$ is prime or
the values of n for which $3*n - 2*n$ is prime. (You might start by making a
table similar to Figure 1.)
\vspace{20pt}

The following simple Haskell program:
\lstinputlisting[language=Haskell]{0-2.hs}
gives the output:

\begin{verbatim}
"2. 3^n - 1: not prime | 3^n - 2^n: prime"
"3. 3^n - 1: not prime | 3^n - 2^n: prime"
"4. 3^n - 1: not prime | 3^n - 2^n: not prime"
"5. 3^n - 1: not prime | 3^n - 2^n: prime"
"6. 3^n - 1: not prime | 3^n - 2^n: not prime"
"7. 3^n - 1: not prime | 3^n - 2^n: not prime"
"8. 3^n - 1: not prime | 3^n - 2^n: not prime"
"9. 3^n - 1: not prime | 3^n - 2^n: not prime"
"10. 3^n - 1: not prime | 3^n - 2^n: not prime"
"11. 3^n - 1: not prime | 3^n - 2^n: not prime"
"12. 3^n - 1: not prime | 3^n - 2^n: not prime"
"13. 3^n - 1: not prime | 3^n - 2^n: not prime"
"14. 3^n - 1: not prime | 3^n - 2^n: not prime"
"15. 3^n - 1: not prime | 3^n - 2^n: not prime"
"16. 3^n - 1: not prime | 3^n - 2^n: not prime"
"17. 3^n - 1: not prime | 3^n - 2^n: prime"
"18. 3^n - 1: not prime | 3^n - 2^n: not prime"
"19. 3^n - 1: not prime | 3^n - 2^n: not prime"
"20. 3^n - 1: not prime | 3^n - 2^n: not prime"
"21. 3^n - 1: not prime | 3^n - 2^n: not prime"
"22. 3^n - 1: not prime | 3^n - 2^n: not prime"
"23. 3^n - 1: not prime | 3^n - 2^n: not prime"
"24. 3^n - 1: not prime | 3^n - 2^n: not prime"
"25. 3^n - 1: not prime | 3^n - 2^n: not prime"
"26. 3^n - 1: not prime | 3^n - 2^n: not prime"
"27. 3^n - 1: not prime | 3^n - 2^n: not prime"
"28. 3^n - 1: not prime | 3^n - 2^n: not prime"
"29. 3^n - 1: not prime | 3^n - 2^n: prime"
"30. 3^n - 1: not prime | 3^n - 2^n: not prime"
"31. 3^n - 1: not prime | 3^n - 2^n: prime"
"32. 3^n - 1: not prime | 3^n - 2^n: not prime"
"33. 3^n - 1: not prime | 3^n - 2^n: not prime"
"34. 3^n - 1: not prime | 3^n - 2^n: not prime"
"35. 3^n - 1: not prime | 3^n - 2^n: not prime"
"36. 3^n - 1: not prime | 3^n - 2^n: not prime"
"37. 3^n - 1: not prime | 3^n - 2^n: not prime"
"38. 3^n - 1: not prime | 3^n - 2^n: not prime"
"39. 3^n - 1: not prime | 3^n - 2^n: not prime"
"40. 3^n - 1: not prime | 3^n - 2^n: not prime"
"41. 3^n - 1: not prime | 3^n - 2^n: not prime"
"42. 3^n - 1: not prime | 3^n - 2^n: not prime"
"43. 3^n - 1: not prime | 3^n - 2^n: not prime"
"44. 3^n - 1: not prime | 3^n - 2^n: not prime"
"45. 3^n - 1: not prime | 3^n - 2^n: not prime"
"46. 3^n - 1: not prime | 3^n - 2^n: not prime"
"47. 3^n - 1: not prime | 3^n - 2^n: not prime"
"48. 3^n - 1: not prime | 3^n - 2^n: not prime"
"49. 3^n - 1: not prime | 3^n - 2^n: not prime"
"50. 3^n - 1: not prime | 3^n - 2^n: not prime"
\end{verbatim}
Conjecture 1: $3^n - 1$ doesn't contain prime numbers.
\vspace{40pt}

3. The proof of Theorem 3 gives a method for finding a prime number different
from any in a given list of prime numbers.

(a) Use this method to find a prime different from $2$, $3$, $5$, and $7$.

(b) Use this method to find a prime different from $2$, $5$, and $11$.
\vspace{20pt}

(a) $2*3*5*7+1 = 211$
\vspace{10pt}

(b) $$1*2+1=3$$
    $$11*2+1=23$$
\vspace{40pt}

4. Find five consecutive integers that are not prime.
\vspace{20pt}

$764, 765, 766, 767, 768$
\vspace{40pt}

5. Use the table in Figure 1 and the discussion on p. 5 to find two more perfect
numbers.
\vspace{20pt}

1) $n = 5$, $2^n-1 = 2^5-1=31$ (prime); Using the formula for perfect number $2^{n-1}*(2^n-1) = 2^4*(2^5-1) = 16*31 = 496$
\vspace{10pt}

2) $n = 7$, $2^7-1= 128 - 1 = 127$ (prime); Calculating prime: $2^6*(2^7-1) = 64*127 = 8128$
\vspace{40pt}

6. The sequence $3, 5, 7$ is a list of three prime numbers such that each pair of
adjacent numbers in the list differ by two. Are there any more such “triplet
primes”?
\vspace{20pt}

\lstinputlisting[language=Haskell]{0-6.hs}

Using the program above I couldn't find another “triplet”. I alse tried the case when each pair of adjacent numbers in the list differ by three.
\pagebreak

\centerline{\sc \large 1. Sentential Logic}
\vspace{50pt}

\textbf{1.1. Deductive Reasoning and Logical Connectives}
\vspace{40pt}

1. Analyze the logical forms of the following statements:

\hspace{12pt}(a) We'll have either a reading assignment or homework problems, but we
won't have both homework problems and a test.

\hspace{12pt}(b) You won't go skiing, or you will and there won't be any snow.

\hspace{12pt}(c) $\sqrt{7} \nleq 2$.
\vspace{20pt}

(a) Let P be "we have reading assignment" and Q be "we have homework problems", then

$$(P \lor Q) \lor \neg (P \land Q)$$
\vspace{10pt}

(b) $$\neg Q \lor (Q \land \neg R)$$
\vspace{10pt}

(c) $$\neg [(\sqrt{7} < 2) \lor (\sqrt{7} = 2)]$$
\vspace{40pt}

2. Analyze the logical forms of the following statements:

\hspace{12pt}(a) Either John and Bill are both telling the truth, or neither of them is.

\hspace{12pt}(b) I’ll have either fish or chicken, but I won’t have both fish and mashed
potatoes.

\hspace{12pt}(c) 3 is a common divisor of 6, 9, and 15
\vspace{20pt}

(a) let A be "John tells the truth" and B be "Bill tells the truth"
$$(A \land B) \lor \neg (A \land B)$$
\vspace{10pt}

(b) $$(A \lor B) \land \neg (A \land C)$$
\vspace{10pt}

(c) $$A \land B \land C$$
\vspace{40pt}

3. Analyze the logical forms of the following statements:

\hspace{12pt}(a) Alice and Bob are not both in the room.

\hspace{12pt}(b) Alice and Bob are both not in the room.

\hspace{12pt}(c) Either Alice or Bob is not in the room.

\hspace{12pt}(d) Neither Alice nor Bob is in the room.
\vspace{20pt}

(a) $$(\neg A \land B) \lor (A \land \neg B) \lor (\neg A \land \neg B)$$
\vspace{10pt}

(b) $$\neg (A \land B)$$
\vspace{10pt}

(c) $$(\neg A \land B) \lor (A \land \neg B)$$
\vspace{10pt}

(d) $$\neg A \land \neg B$$
\vspace{40pt}

4. Which of the following expressions are well-formed formulas?

\hspace{12pt}(a) $\neg (\neg P \lor \neg \neg R)$

\hspace{12pt}(b) $\neg (P, Q, \neg R)$

\hspace{12pt}(c) $P \land \neg P$

\hspace{12pt}(d) $(P \land Q)(P \lor R)$
\vspace{20pt}

\hspace{12pt}(c) is well-formed formula.
\vspace{40pt}

5. Let P stand for the statement "I will buy the pants" and S for the statement
"I will buy the shirt." What English sentences are represented by the following
expressions?

\hspace{12pt}(a) $\neg (P \land \neg S)$

\hspace{12pt}(b) $\neg P \land \neg S$.

\hspace{12pt}(c) $\neg P \lor \neg S$
\vspace{20pt}

(a) I won't buy the pants without the shirt.
\vspace{10pt}

(b) Neither I will buy the pants nor I will buy the shirt.
\vspace{10pt}

(c) Either I won't buy the pants or I won't buy the shirt.
\vspace{40pt}

6. Let S stand for the statement "Steve is happy" and G for "George is happy."
What English sentences are represented by the following expressions?

\hspace{12pt}(a) $(S \lor G) \land (\neg S \lor \neg G)$.

\hspace{12pt}(b) $[S \lor (G \land \neg S)] \lor \neg G$

\hspace{12pt}(c) $S \lor [G \land (\neg S \lor \neg G)]$
\vspace{20pt}

(a) Steve or George are happy, but either Steve is not happy or George.

(b) Steve is happy or George is happy, but Steve not; or George is not happy.

(c) Either Steve is happy or George is happy, but Steve is not happy or George.
\vspace{40pt}

7. Identify the premises and conclusions of the following deductive arguments
and analyze their logical forms. Do you think the reasoning is valid?
(Although you will have only your intuition to guide you in answering
this last question, in the next section we will develop some techniques for
determining the validity of arguments.)

\hspace{12pt}(a) Jane and Pete won’t both win the math prize. Pete will win either
the math prize or the chemistry prize. Jane will win the math prize.
Therefore, Pete will win the chemistry prize.

\hspace{12pt}(b) The main course will be either beef or fish. The vegetable will be either
peas or corn. We will not have both fish as a main course and corn as a
vegetable. Therefore, we will not have both beef as a main course and
peas as a vegetable.

\hspace{12pt}(c) Either John or Bill is telling the truth. Either Sam or Bill is lying.
Therefore, either John is telling the truth or Sam is lying.

\hspace{12pt}(d) Either sales will go up and the boss will be happy, or expenses will go up and the boss won’t be happy. Therefore, sales and expenses will not
both go up.
\vspace{20pt}

(a) $$(\neg J \land P) \lor (J \land \neg P)$$
$$P \lor C$$
$$J$$
$$C$$

Valid

\vspace{20pt}

(b) $$(B \land \neg F) \lor (\neg B \land F)$$
$$(P \land \neg C) \lor (\neg P \land C)$$
$$\neg (F \land C)$$
$$\neg (B \land P)$$
Invalid

\vspace{20pt}

(c) $$(J \land \neg B) \lor (\neg J \land B)$$
$$\neg S \lor \neg B$$
$$J \lor \neg S$$
Valid

\vspace{20pt}

(d) $$(S \land H) \lor (E \land \neg H)$$
$$\neg (S \land E)$$

Valid

\vspace{50pt}

\textbf{1.2. Truth Tables}

\vspace{40pt}

\centerline{
  \begin{tabular}{c c l c c}
  S & L & $(\neg S \land L) \lor S$ & S & $\neg L$ \\
  \hline
  F & F & T F F F \textbf{F} F & F & T \\
  F & T & T F T T \textbf{T} F & F & F \\
  T & F & F T F F \textbf{T} T & T & T \\
  T & T & F T F T \textbf{T} T & T & F \\
  \end{tabular}}
\vspace{20pt}

\centerline{
  \begin{tabular}{c c l l l}
  P & Q & $\neg (P \land Q)$ & $\neg P \land \neg Q $ & $\neg P \lor \neg Q$ \\
  \hline
  F & F & \textbf{T} F F F & T F \textbf{T} T F & T F \textbf{T} T F \\
  F & T & \textbf{T} F F T & T F \textbf{F} F T & T F \textbf{T} F T \\
  T & F & \textbf{T} T F F & F T \textbf{F} T F & F T \textbf{T} T F \\
  T & T & \textbf{F} T T T & F T \textbf{F} F T & F T \textbf{F} F T \\
  \end{tabular}}
\vspace{20pt}

\textbf{DeMorgan's laws}
\vspace{20pt}

$\neg (P \land Q)$ is equivalent to $\neg P \lor \neg Q$

\centerline{
  \begin{tabular}{c c l l}
  P & Q & $\neg (P \land Q)$ & $\neg P \lor \neg Q$ \\
  \hline
  F & F & \textbf{T} F F F & T F \textbf{T} T F \\
  F & T & \textbf{T} F F T & T F \textbf{T} F T \\
  T & F & \textbf{T} T F F & F T \textbf{T} T F \\
  T & T & \textbf{F} T T T & F T \textbf{F} F T \\
  \end{tabular}}
\vspace{10pt}

Let P stand for "Alice is smart" and P stand for "Bob is smart"

Alice and Bob aren't both smart

Either Alice or Bob aren't smart.
\vspace{20pt}

$\neg (P \lor Q)$ is equivalent to $\neg P \land \neg Q$

\centerline{
  \begin{tabular}{c c l l }
  P & Q & $\neg (P \lor Q)$ & $\neg P \land \neg Q $ \\
  \hline
  F & F & \textbf{T} F F F & T F \textbf{T} T F \\
  F & T & \textbf{F} F T T & T F \textbf{F} F T \\
  T & F & \textbf{F} T T F & F T \textbf{F} T F \\
  T & T & \textbf{F} T T T & F T \textbf{F} F T \\
  \end{tabular}}
\vspace{10pt}

Let P stand for "Alice is smart" and P stand for "Bob is smart"

Both Alice or Bob aren't smart.

Alice isn't smart and Bob isn't smart.
\vspace{20pt}

\textbf{Commutative laws}
\vspace{20pt}

$P \land Q$ is equivalent to $Q \land P$

\centerline{
  \begin{tabular}{c c l l }
  P & Q & $P \land Q$ & $Q \land P$ \\
  \hline
  F & F & F \textbf{F} F & F \textbf{F} F \\
  F & T & F \textbf{F} T & T \textbf{F} F \\
  T & F & T \textbf{F} F & F \textbf{F} T \\
  T & T & T \textbf{T} T & T \textbf{T} T \\
  \end{tabular}}
\vspace{10pt}

Alice is smart and Bob is smart

Bob is smart and Alice is smart
\vspace{20pt}

$P \lor Q$ is equivalent to $Q \lor P$

\centerline{
  \begin{tabular}{c c l l }
  P & Q & $P \lor Q$ & $Q \lor P$ \\
  \hline
  F & F & F \textbf{F} F & F \textbf{F} F \\
  F & T & F \textbf{T} T & T \textbf{T} F \\
  T & F & T \textbf{T} F & F \textbf{T} T \\
  T & T & T \textbf{T} T & T \textbf{T} T \\
  \end{tabular}}
\vspace{10pt}

Alice is smart or Bob is smart

Bob is smart or Alice is smart
\vspace{20pt}

\textbf{Associative laws}
\vspace{20pt}

$P \land (Q \land R)$ is equivalent to $(P \land Q) \land R$

\centerline{
  \begin{tabular}{c c c l l}
  P & Q & R & $P \land (Q \land R)$ & $(P \land Q) \land R$ \\
  \hline
  F & F & F & F \textbf{F} F F F & F F F \textbf{F} F \\
  F & T & F & F \textbf{F} T F F & F F T \textbf{F} F \\
  F & F & T & F \textbf{F} F F T & F F F \textbf{F} T \\
  F & T & T & F \textbf{F} T T T & F F T \textbf{F} T \\
  T & F & F & T \textbf{F} F F F & T F F \textbf{F} F \\
  T & T & F & T \textbf{F} T F F & T T T \textbf{F} F \\
  T & F & T & T \textbf{F} F F T & T F F \textbf{F} T \\
  T & T & T & T \textbf{T} T T T & T T T \textbf{T} T \\
  \end{tabular}}
\vspace{10pt}

P: Alice is smart

Q: Bob is smart

R: Eve is smart

Bob and Eve are both smart and Alice is smart

Alice and Bob are both smart and Eve is smart
\vspace{20pt}

$P \lor (Q \lor R)$ is equivalent to $(P \lor Q) \lor R$

\centerline{
  \begin{tabular}{c c c l l}
  P & Q & R & $P \lor (Q \lor R)$ & $(P \lor Q) \lor R$ \\
  \hline
  F & F & F & F \textbf{F} F F F & F F F \textbf{F} F \\
  F & T & F & F \textbf{T} T T F & F T T \textbf{T} F \\
  F & F & T & F \textbf{T} F T T & F F F \textbf{T} T \\
  F & T & T & F \textbf{T} T T T & F T T \textbf{T} T \\
  T & F & F & T \textbf{T} F F F & T T F \textbf{T} F \\
  T & T & F & T \textbf{T} T T F & T T T \textbf{T} F \\
  T & F & T & T \textbf{T} F T T & T T F \textbf{T} T \\
  T & T & T & T \textbf{T} T T T & T T T \textbf{T} T \\
  \end{tabular}}
\vspace{10pt}

P: Alice is smart

Q: Bob is smart

R: Eve is smart

Bob or Eve are smart, or Alice is smart

Either Alice or Bob are smart, or Eve is smart
\vspace{20pt}

\textbf{Idemponent laws}
\vspace{20pt}

$P \land P$ is equivalent to P

\centerline{
  \begin{tabular}{c l}
  P & $P \land P$ \\
  \hline
  T & T \textbf{T} T \\
  F & F \textbf{F} F \\
  \end{tabular}}
\vspace{10pt}

P: Alice is smart

Alice is smart and Alice is smart.

Alice is smart.
\vspace{20pt}

$P \lor P$ is equivalent to P

\centerline{
  \begin{tabular}{c l}
  P & $P \lor P$ \\
  \hline
  T & T \textbf{T} T \\
  F & F \textbf{F} F \\
  \end{tabular}}
\vspace{10pt}

P: Alice is smart

Alice is smart or Alice is smart.

Alice is smart.
\vspace{20pt}

\textbf{Distributive laws}
\vspace{20pt}

$P \land (Q \lor R)$ is equivalent to $(P \land Q) \lor (P \land R)$

\centerline{
  \begin{tabular}{c c c l l}
  P & Q & R & $P \land (Q \lor R)$ & $(P \land Q) \lor (P \land R)$ \\
  \hline
  F & F & F & F \textbf{F} F F F & F F F \textbf{F} F F F \\
  F & T & F & F \textbf{F} T T F & F F T \textbf{F} F F F \\
  F & F & T & F \textbf{F} F T T & F F F \textbf{F} F F T \\
  F & T & T & F \textbf{F} T T T & F F T \textbf{F} F F T \\
  T & F & F & T \textbf{F} F F F & T F F \textbf{F} T F F \\
  T & T & F & T \textbf{T} T T F & T T T \textbf{T} T F F \\
  T & F & T & T \textbf{T} F T T & T F F \textbf{T} T T T \\
  T & T & T & T \textbf{T} T T T & T T T \textbf{T} T T T \\
  \end{tabular}}
\vspace{10pt}

P: Alice is smart

Q: Bob is smart

R: Eve is smart

Bob and Eve are both smart, and Alice is smart.

Either Alice and Bob are both smart or Alice and Eve are both smart.
\vspace{20pt}

$P \lor (Q \land R)$ is equivalent to $(P \lor Q) \land (P \lor R)$

\centerline{
  \begin{tabular}{c c c l l}
  P & Q & R & $P \lor (Q \land R)$ & $(P \lor Q) \land (P \lor R)$ \\
  \hline
  F & F & F & F \textbf{F} F F F & F F F \textbf{F} F F F \\
  F & T & F & F \textbf{F} T F F & F T T \textbf{F} F F F \\
  F & F & T & F \textbf{F} F F T & F F F \textbf{F} F T T \\
  F & T & T & F \textbf{T} T T T & F T T \textbf{T} F T T \\
  T & F & F & T \textbf{T} F F F & T T F \textbf{T} T T F \\
  T & T & F & T \textbf{T} T F F & T T T \textbf{T} T T F \\
  T & F & T & T \textbf{T} F F T & T T F \textbf{T} T T T \\
  T & T & T & T \textbf{T} T T T & T T T \textbf{T} T T T \\
  \end{tabular}}
\vspace{10pt}

P: Alice is smart

Q: Bob is smart

R: Eve is smart

Bob and Eve are both smart or Alice is smart.

Either Alice or Bob are smart and either Alice and Eve are smart.
\vspace{20pt}

\textbf{Absorption laws}
\vspace{20pt}

$P \lor (P \land Q)$ is equivalent to P

\centerline{
  \begin{tabular}{c c l}
  P & Q & $P \lor (P \land Q)$ \\
  \hline
  F & F & F \textbf{F} F F F \\
  F & T & F \textbf{F} F F T \\
  T & F & T \textbf{T} T F F \\
  T & T & T \textbf{T} T T T \\
  \end{tabular}}
\vspace{10pt}

P: Alice is smart

Q: Bob is smart

Alice is smart or Alice and Bob are both smart.

Alice is smart.
\vspace{20pt}

$P \land (P \lor Q)$ is equivalent to P

\centerline{
  \begin{tabular}{c c l}
  P & Q & $P \land (P \lor Q)$ \\
  \hline
  F & F & F \textbf{F} F F F \\
  F & T & F \textbf{F} F T T \\
  T & F & T \textbf{T} T T F \\
  T & T & T \textbf{T} T T T \\
  \end{tabular}}
\vspace{10pt}

P: Alice is smart

Q: Bob is smart

Alice is smart and either Alice or Bob are smart.

Alice is smart.
\vspace{20pt}

\textbf{Double Negation laws}
\vspace{20pt}

$\neg \neg P$ is equivalent to P

\centerline{
  \begin{tabular}{c l}
  P & $\neg \neg P$ \\
  \hline
  T & \textbf{T} F T \\
  \end{tabular}}
\vspace{10pt}

P: Alice is smart

Alice isn't stupid.
\vspace{40pt}

1. Make truth tables for the following formulas:

\hspace{12pt}(a) $\neg P \lor Q$.

\hspace{12pt}(b) $(S \lor G) \land (\neg S \lor \neg G)$.
\vspace{20pt}

(a)

\centerline{
  \begin{tabular}{c c l}
  P & Q & $\neg P \lor Q$ \\
  \hline
  F & F & T F \textbf{T} F \\
  F & T & T F \textbf{T} T \\
  T & F & F T \textbf{F} F \\
  T & T & F T \textbf{T} T \\
  \end{tabular}}
\vspace{10pt}

(b)

\centerline{
  \begin{tabular}{c c l}
  S & G & $(S \lor G) \land (\neg S \lor \neg G)$ \\
  \hline
  F & F & F F F \textbf{F} T F T T F \\
  F & T & F T T \textbf{T} T F T F T \\
  T & F & T T F \textbf{T} F T T T F \\
  T & T & T T T \textbf{F} F T F F T \\
  \end{tabular}}
\vspace{30pt}

2. Make truth tables for the following formulas:

\hspace{12pt}(a) $\neg [P \land (Q \lor \neg P)]$

\hspace{12pt}(b) $(P \lor Q) \land (\neg P \lor R)$
\vspace{20pt}

(a)

\centerline{
  \begin{tabular}{c c l}
  P & Q & $\neg [P \land (Q \lor \neg P)]$ \\
  \hline
  F & F & \textbf{T} F F F T T F \\
  F & T & \textbf{T} F F T T T F \\
  T & F & \textbf{T} T F F F F T \\
  T & T & \textbf{F} T T T T F T \\
  \end{tabular}}
\vspace{10pt}

(b)

\centerline{
  \begin{tabular}{c c c l l c}
  P & Q & R & $(P \lor Q)$ & $(\neg P \lor R)$ &  $(P \lor Q) \land (\neg P \lor R)$ \\
  \hline
  F & F & F & F \textbf{F} F & T F \textbf{T} F & F \\
  F & T & F & F \textbf{T} T & T F \textbf{T} F & T \\
  F & F & T & F \textbf{F} F & T F \textbf{T} T & F \\
  F & T & T & F \textbf{T} T & T F \textbf{T} T & T \\
  T & F & F & T \textbf{T} F & F T \textbf{F} F & F \\
  T & T & F & T \textbf{T} T & F T \textbf{F} F & F \\
  T & F & T & T \textbf{T} F & F T \textbf{T} T & T \\
  T & T & T & T \textbf{T} T & F T \textbf{T} T & T \\
  \end{tabular}}
\vspace{30pt}

3. In this exercise we will use the symbol $+$ to mean exclusive or. In other
words, P $+$ Q means "P or Q, but not both."

\hspace{12pt}(a) Make a truth table for P $+$ Q.

\hspace{12pt}(b) Find a formula using only the connectives $\land$, $\lor$, and $\neg$ that is equivalent
to P $+$ Q. Justify your answer with a truth table.
\vspace{20pt}

(a)

\centerline{
  \begin{tabular}{c c l}
  P & Q & $+$ \\
  \hline
  F & F & F \\
  F & T & T \\
  T & F & T \\
  T & T & F \\
  \end{tabular}}
\vspace{10pt}

(b) $$P \land Q \lor \neg P$$

\centerline{
  \begin{tabular}{c c l l c}
  P & Q & $\neg P \land \neg Q$ & $P \land Q$ & $\neg [(\neg P \land \neg Q) \lor (P \land Q)]$\\
  \hline
  F & F & T F \textbf{T} T F & F \textbf{F} F & F \\
  F & T & T F \textbf{F} F T & F \textbf{F} T & T \\
  T & F & F T \textbf{F} T F & T \textbf{F} F & T \\
  T & T & F T \textbf{F} F T & T \textbf{T} T & F \\
  \end{tabular}}
\vspace{40pt}

4. Find a formula using only the connectives $\land$ and $\neg$ that is equivalent to
$P \lor Q$. Justify your answer with a truth table.
\vspace{30pt}

\centerline{
  \begin{tabular}{c c l l}
  P & Q & $P \lor Q$ & $\neg (\neg P \land \neg Q)$ \\
  \hline
  F & F & F & \textbf{F} T F T T F\\
  F & T & T & \textbf{T} T F F F T\\
  T & F & T & \textbf{T} F T F T F\\
  T & T & T & \textbf{T} F T F F T\\
  \end{tabular}}
\vspace{40pt}

5. Some mathematicians use the symbol $\downarrow$ to mean nor.

In other words, P $\downarrow$ Q means "neither P nor Q."

\hspace{12pt}(a) Make a truth table for P $\downarrow$ Q.

\hspace{12pt}(b) Find a formula using only the connectives $\land$, $\lor$, and $\neg$ that is equivalent
to P $\downarrow$ Q.

\hspace{12pt}(c) Find formulas using only the connective $\downarrow$ that are equivalent to $\neg P$,
$P \lor Q$, and $P \land Q$.
\vspace{20pt}

(a)

\centerline{
  \begin{tabular}{c c l}
  P & Q & $\downarrow$ \\
  \hline
  F & F & T \\
  F & T & F \\
  T & F & F \\
  T & T & F \\
  \end{tabular}}
\vspace{10pt}

(b)

\centerline{
  \begin{tabular}{c c l}
  P & Q & $\neg P \land \neg Q$ \\
  \hline
  F & F & T F \textbf{T} T F \\
  F & T & T F \textbf{F} F T \\
  T & F & F T \textbf{F} T F \\
  T & T & F T \textbf{F} F T \\
  \end{tabular}}
\vspace{10pt}

(c)

\centerline{
  \begin{tabular}{c c c}
  P & $\neg P$ & $P \downarrow P$ \\
  \hline
  T & F & F \\
  F & T & T \\
  \end{tabular}}
\vspace{10pt}

\centerline{
  \begin{tabular}{c c c c}
  P & Q & $P \lor Q$ & $(P \downarrow Q) \downarrow (P \downarrow Q)$ \\
  \hline
  F & F & F & F \\
  F & T & T & T \\
  T & F & T & T \\
  T & T & T & T \\
  \end{tabular}}
\vspace{10pt}

\centerline{
  \begin{tabular}{c c c c}
  P & Q & $P \land Q$ & $(P \downarrow P) \downarrow (Q \downarrow Q)$ \\
  \hline
  F & F & F & F \\
  F & T & F & F \\
  T & F & F & F \\
  T & T & T & T \\
  \end{tabular}}
\vspace{40pt}

6. Some mathematicians write $P | Q$ to mean "P and Q are not both true."
(This connective is called nand, and is used in the study of circuits in
computer science.)

\hspace{12pt}(a) Make a truth table for $P | Q$.

\hspace{12pt}(b) Find a formula using only the connectives $\land$, $\lor$, and $\neg$ that is equivalent
to $P | Q$.

\hspace{12pt}(c) Find formulas using only the connective $|$ that are equivalent to $\neg P$,
$P \lor Q$, and $P \land Q$.
\vspace{20pt}

(a)

\centerline{
  \begin{tabular}{c c c}
  P & Q & $P | Q$ \\
  \hline
  F & F & T \\
  F & T & T \\
  T & F & T \\
  T & T & F \\
  \end{tabular}}
\vspace{10pt}

(b)

\centerline{
  \begin{tabular}{c c c}
  P & Q & $\neg (P \land Q$) \\
  \hline
  F & F & T \\
  F & T & T \\
  T & F & T \\
  T & T & F \\
  \end{tabular}}
\vspace{10pt}

(c)

\centerline{
  \begin{tabular}{c c c}
  P & $\neg P$ & $P | P$ \\
  \hline
  T & F & F \\
  F & T & T \\
  \end{tabular}}
\vspace{10pt}

\centerline{
  \begin{tabular}{c c c l}
  P & Q & $P \land Q$ & $(P | Q) | (P | Q)$ \\
  \hline
  F & F & F & F T F \textbf{F} F T F \\
  F & T & F & F T T \textbf{F} F T T \\
  T & F & F & T T F \textbf{F} T T F \\
  T & T & T & T F T \textbf{T} T F T \\
  \end{tabular}}
\vspace{10pt}

\centerline{
  \begin{tabular}{c c c c}
  P & Q & $P \lor Q$ & $(P | P) | (Q | Q)$ \\
  \hline
  F & F & F & F T F \textbf{F} F T F \\
  F & T & T & F T F \textbf{T} T F T \\
  T & F & T & T F T \textbf{T} F T F \\
  T & T & T & T F T \textbf{T} T F T \\
  \end{tabular}}
\vspace{40pt}

7. Use truth tables to determine whether or not the arguments in exercise 7 of Section 1.1 are valid.
\vspace{20pt}

(a) $$\neg (J \land P)$$
$$P \lor C$$
$$J$$
$$\therefore C$$

\centerline{
  \begin{tabular}{c c c l c c c}
  J & P & C & $\neg (J \land P)$ & $P \lor C$ &  $J$ & $C$ \\
  \hline
  F & F & F & \textbf{T} F F F & F \textbf{F} F & F & F \\
  F & T & F & \textbf{T} F F T & T \textbf{T} F & F & F \\
  F & F & T & \textbf{T} F F F & F \textbf{T} T & F & T \\
  F & T & T & \textbf{T} F F T & T \textbf{T} T & F & T \\
  T & F & F & \textbf{T} T F F & F \textbf{F} F & T & F \\
  T & T & F & \textbf{F} T T T & T \textbf{T} F & T & F \\
  T & F & T & \textbf{T} T F F & F \textbf{T} T & T & T \\
  T & T & T & \textbf{F} T T T & T \textbf{T} T & T & T \\
  \end{tabular}}
\vspace{10pt}

All premises are true only in 7 row and conclusion is true as well. Therefore, the argument is valid.
\vspace{10pt}

(c) $$\neg (B \land F)$$
$$\neg (P \land C)$$
$$\neg F \land \neg C$$
$$\therefore \neg B \land \neg P$$


\centerline{
  \begin{tabular}{c c c c l l l l}
  B & F & P & C & $\neg (B \land F)$ & $\neg (P \land C)$ &  $\neg F \land \neg C$ & $\neg B \land \neg P$ \\
  \hline
  F & F & F & F & \textbf{T} F F F & \textbf{T} F F F & T F \textbf{T} T F & T F \textbf{T} T F \\
  F & F & F & T & \textbf{T} F F F & \textbf{T} F F T & T F \textbf{F} F T & T F \textbf{T} T F \\
  F & F & T & F & \textbf{T} F F F & \textbf{T} T F F & T F \textbf{T} T F & T F \textbf{F} F T \\
  F & F & T & T & \textbf{T} F F F & \textbf{F} T T T & T F \textbf{F} F T & T F \textbf{F} F T \\
  F & T & F & F & \textbf{T} F F T & \textbf{T} F F F & F T \textbf{F} T F & T F \textbf{T} T F \\
  F & T & F & T & \textbf{T} F F T & \textbf{T} F F T & F T \textbf{F} F T & T F \textbf{T} T F \\
  F & T & T & F & \textbf{T} F F T & \textbf{T} T F F & F T \textbf{F} T F & T F \textbf{F} F T \\
  F & T & T & T & \textbf{T} F F T & \textbf{F} T T T & F T \textbf{F} F T & T F \textbf{F} F T \\
  T & F & F & F & \textbf{T} T F F & \textbf{T} F F F & T F \textbf{T} T F & F T \textbf{F} T F \\
  T & F & F & T & \textbf{T} T F F & \textbf{T} F F T & T F \textbf{F} F T & F T \textbf{F} T F \\
  T & F & T & F & \textbf{T} T F F & \textbf{T} T F F & T F \textbf{T} T F & F T \textbf{F} F T \\
  T & F & T & T & \textbf{T} T F F & \textbf{F} T T T & T F \textbf{F} F T & F T \textbf{F} F T \\
  T & T & F & F & \textbf{F} T T T & \textbf{T} F F F & F T \textbf{F} T F & F T \textbf{F} T F \\
  T & T & F & T & \textbf{F} T T T & \textbf{T} F F T & F T \textbf{F} F T & F T \textbf{F} T F \\
  T & T & T & F & \textbf{F} T T T & \textbf{T} T F F & F T \textbf{F} T F & F T \textbf{F} F T \\
  T & T & T & T & \textbf{F} T T T & \textbf{F} T T T & F T \textbf{F} F T & F T \textbf{T} F T \\
  \end{tabular}}
\vspace{10pt}

3rd row: $T T T \therefore F$. Therefore, the argument is invalid.
\vspace{10pt}

(c) $$J \lor B$$
$$\neg (S \land B)$$
$$J \lor \neg S$$

\centerline{
  \begin{tabular}{c c c l l l}
  J & B & S & $J \land B$ & $\neg (S \land B)$ &  $J \lor \neg S$ \\
  \hline
  F & F & F & F \textbf{F} F & \textbf{T} F F F & F \textbf{T} T F \\
  F & T & F & F \textbf{F} T & \textbf{T} F F T & F \textbf{T} T F \\
  F & F & T & F \textbf{F} F & \textbf{T} T F F & F \textbf{F} F T \\
  F & T & T & F \textbf{F} T & \textbf{F} T T T & F \textbf{F} F T \\
  T & F & F & T \textbf{F} F & \textbf{T} F F F & T \textbf{T} T F \\
  T & T & F & T \textbf{T} T & \textbf{T} F F T & T \textbf{T} T F \\
  T & F & T & T \textbf{F} F & \textbf{T} T F F & T \textbf{T} F T \\
  T & T & T & T \textbf{T} T & \textbf{F} T T T & T \textbf{T} F T \\
  \end{tabular}}
\vspace{10pt}

All premises are true only in 6 row and conclusion is true as well. Therefore, the argument is valid.

(d) $$(S \land H) \lor (E \land \neg H)$$
$$\neg (S \land E)$$

\centerline{
  \begin{tabular}{c c c l l}
  S & H & E & $(S \land H) \lor (E \land \neg H)$ & $\neg (S \land E)$ \\
  \hline
  F & F & F & F F F \textbf{F} F F T F & \textbf{T} F F F \\
  F & T & F & F F T \textbf{F} F F F T & \textbf{T} F F F \\
  F & F & T & F F F \textbf{T} T T T F & \textbf{T} F F T \\
  F & T & T & F F T \textbf{F} T F F T & \textbf{T} F F T \\
  T & F & F & T F F \textbf{F} F F T F & \textbf{T} T F F \\
  T & T & F & T T T \textbf{T} F F F T & \textbf{T} T F F \\
  T & F & T & T F F \textbf{T} T T T F & \textbf{F} T T T \\
  T & T & T & T T T \textbf{T} T F F T & \textbf{F} T T T \\
  \end{tabular}}
\vspace{10pt}

7th row: $T \therefore F$. Therefore, the argument is invalid.
\vspace{30pt}

8. Use truth tables to determine which of the following formulas are equivalent
to each other:

\hspace{12pt}(a) $(P \land Q) \lor (\neg P \land \neg Q)$

\hspace{12pt}(b) $\neg P \lor Q$

\hspace{12pt}(c) $(P \lor \neg Q) \land (Q \lor \neg P)$

\hspace{12pt}(d) $\neg (P \lor Q)$

\hspace{12pt}(e) $(Q \land P) \lor \neg P$
\vspace{20pt}

(a)

\centerline{
  \begin{tabular}{c c l}
  P & Q & $(P \land Q) \lor (\neg P \land \neg Q)$ \\
  \hline
  F & F & F \textbf{F} F \textbf{T} T F \textbf{T} T F \\
  F & T & F \textbf{F} T \textbf{F} T F \textbf{F} F T \\
  T & F & T \textbf{F} F \textbf{F} F T \textbf{F} T F \\
  T & T & T \textbf{T} T \textbf{T} F T \textbf{F} F T \\
  \end{tabular}}
\vspace{10pt}

(b)

\centerline{
  \begin{tabular}{c c l}
  P & Q & $\neg P \lor Q$ \\
  \hline
  F & F & T F \textbf{T} F \\
  F & T & T F \textbf{T} T \\
  T & F & F T \textbf{F} F \\
  T & T & F T \textbf{T} T \\
  \end{tabular}}
\vspace{10pt}

(c)

\centerline{
  \begin{tabular}{c c l}
  P & Q & $(P \lor \neg Q) \land (Q \lor \neg P)$ \\
  \hline
  F & F & F \textbf{T} T F \textbf{T} F \textbf{T} T F \\
  F & T & F \textbf{F} F T \textbf{F} T \textbf{T} T F \\
  T & F & T \textbf{T} T F \textbf{F} F \textbf{F} F T \\
  T & T & T \textbf{T} F T \textbf{T} T \textbf{T} F T \\
  \end{tabular}}
\vspace{10pt}

(d)

\centerline{
  \begin{tabular}{c c l}
  P & Q & $(Q \land P) \lor \neg P$ \\
  \hline
  F & F & F \textbf{F} F \textbf{T} T F \\
  F & T & T \textbf{F} F \textbf{T} T F \\
  T & F & F \textbf{F} T \textbf{F} F T \\
  T & T & T \textbf{T} T \textbf{T} F T \\
  \end{tabular}}
\vspace{10pt}


(a) is equivalent to (d) and (b) is equivalent to (d).
\vspace{30pt}

9.  Use truth tables to determine which of these statements are tautologies,
which are contradictions, and which are neither:

\hspace{12pt}(a) $(P \lor Q) \land (\neg P \lor \neg Q)$

\hspace{12pt}(b) $(P \lor Q) \land (\neg P \land \neg Q)$

\hspace{12pt}(c) $(P \lor Q) \lor (\neg P \lor \neg Q)$

\hspace{12pt}(d) $[P \land (Q \lor \neg R)] \lor (\neg P \lor R)$
\vspace{20pt}

(a)

\centerline{
  \begin{tabular}{c c l}
  P & Q & $(P \lor Q) \land (\neg P \lor \neg Q)$ \\
  \hline
  F & F & F \textbf{F} F \textbf{F} T F \textbf{T} T F \\
  F & T & F \textbf{T} T \textbf{T} T F \textbf{T} F T \\
  T & F & T \textbf{T} F \textbf{T} F T \textbf{T} T F \\
  T & T & T \textbf{T} T \textbf{F} F T \textbf{F} F T \\
  \end{tabular}}
\vspace{10pt}

(b)

\centerline{
  \begin{tabular}{c c l}
  P & Q & $(P \lor Q) \land (\neg P \land \neg Q)$ \\
  \hline
  F & F & F \textbf{F} F \textbf{F} T F \textbf{T} T F \\
  F & T & F \textbf{T} T \textbf{F} T F \textbf{F} F T \\
  T & F & T \textbf{T} F \textbf{F} F T \textbf{F} T F \\
  T & T & T \textbf{T} T \textbf{F} F T \textbf{F} F T \\
  \end{tabular}}
\vspace{10pt}

(c)

\centerline{
  \begin{tabular}{c c l}
  P & Q & $(P \lor Q) \lor (\neg P \lor \neg Q)$ \\
  \hline
  F & F & F \textbf{F} F \textbf{T} T F \textbf{T} T F \\
  F & T & F \textbf{T} T \textbf{T} T F \textbf{T} F T \\
  T & F & T \textbf{T} F \textbf{T} F T \textbf{T} T F \\
  T & T & T \textbf{T} T \textbf{T} F T \textbf{F} F T \\
  \end{tabular}}
\vspace{10pt}

(d)

\centerline{
  \begin{tabular}{c c c l}
  P & Q & R & $[P \land (Q \lor \neg R)] \lor (\neg P \lor R)$ \\
  \hline
  F & F & F & F \textbf{F} F T T F \textbf{T} T F \textbf{T} F \\
  F & T & F & F \textbf{F} T T T F \textbf{T} T F \textbf{T} F \\
  F & F & T & F \textbf{F} F F F T \textbf{T} T F \textbf{T} T \\
  F & T & T & F \textbf{F} T T F T \textbf{T} T F \textbf{T} T \\
  T & F & F & T \textbf{T} F T T F \textbf{T} F T \textbf{F} F \\
  T & T & F & T \textbf{T} T T T F \textbf{T} F T \textbf{F} F \\
  T & F & T & T \textbf{F} F F F T \textbf{T} F T \textbf{T} T \\
  T & T & T & T \textbf{T} T T F T \textbf{T} F T \textbf{T} T \\
  \end{tabular}}
\vspace{10pt}

(b) is contradiction and (c), (d) are tautologies.

\vspace{30pt}

10. Use truth tables to check these laws:

\hspace{12pt}(a) The second DeMorgan’s law. (The first was checked in the text.)

\hspace{12pt}(b) The distributive laws.

Done, at the beginning of chapter.

\vspace{30pt}

11. Use the laws stated in the text to find simpler formulas equivalent to these
formulas. (See Examples 1.2.5 and 1.2.7.)

\hspace{12pt}(a) $\neg(\neg P \land \neg Q)$

\hspace{12pt}(b) $(P \land Q) \lor (P \land \neg Q)$

\hspace{12pt}(c) $\neg (P \land \neg Q) \lor (\neg P \land Q)$
\vspace{20pt}

(a) $$\neg(\neg P \land \neg Q)$$
DeMorgan's law: $$\neg \neg P \lor \neg \neg Q$$
Double Negation law: $$P \lor Q$$
\vspace{10pt}

(b) $$(P \land Q) \lor (P \land \neg Q)$$
Distributive law: $$P \land (Q \lor \neg Q)$$
\centerline{$P \land$ (a tautology)}

Tautology law: $$P$$
\vspace{10pt}

(c) $$\neg (P \land \neg Q) \lor (\neg P \land Q)$$
DeMorgan's law: $$(\neg P \lor \neg \neg Q) \lor (\neg P \land Q)$$
Double Negation law: $$(\neg P \lor Q) \lor (\neg P \land Q)$$
$$\neg P \lor Q \lor (\neg P \land Q)$$
Commutative law: $$\neg P \lor Q \lor (Q \land \neg P)$$
Absoption law: $$\neg P \lor Q$$

\vspace{30pt}

12. Use the laws stated in the text to find simpler formulas equivalent to these
formulas. (See Examples 1.2.5 and 1.2.7.)

\hspace{12pt}(a) $\neg (\neg P \lor Q) \lor (P \land \neg R)$

\hspace{12pt}(b) $\neg (\neg P \land Q) \lor (P \land \neg R)$

\hspace{12pt}(c) $(P \land R) \lor [\neg R \land (P \lor Q)]$
\vspace{10pt}

(a) $$\neg (\neg P \lor Q) \lor (P \land \neg R)$$
DeMorgan's law: $$(P \land \neg Q) \lor (P \land \neg R)$$
Distributive law: $$P \land (\neg Q \lor \neg R)$$
DeMorgan's law: $$P \land \neg (Q \land R)$$
\vspace{10pt}

(b) $$\neg (\neg P \land Q) \lor (P \land \neg R)$$
DeMorgan's law: $$P \lor \neg Q \lor (P \land \neg R)$$
Commutative law: $$\neg Q \lor P \lor (P \land \neg R)$$
Absoption law: $$\neg Q \lor P$$
\vspace{10pt}

(c) $$(P \land R) \lor [\neg R \land (P \lor Q)]$$
Distributive law: $$(P \land R) \lor [(\neg R \land P) \lor (\neg R \land Q)]$$
Associative law and Commutative law: $$(P \land R) \lor (P \land \neg R) \lor (Q \land \neg R)$$
Distributive law: $$P \land (R \lor \neg R) \lor (Q \land \neg R)$$
Tautology law: $$P \lor (Q \land \neg R)$$
\vspace{30pt}

13. Use the first DeMorgan’s law and the double negation law to derive the
second DeMorgan’s law.

1st: $$\neg (P \land Q) = \neg P \lor \neg Q$$

2st: $$\neg (P \lor Q) = \neg P \land \neg Q$$
Add negation 2 both parts: $$\neg \neg (P \lor Q) = \neg (\neg P \land \neg Q)$$
Use double negation law: $$P \lor Q = \neg (\neg P \land \neg Q)$$
Use 1st DeMorgan's law: $$P \lor Q = \neg \neg P \lor \neg \neg Q$$
Use double negation law: $$P \lor Q = P \lor Q$$


\vspace{30pt}

14. Note that the associative laws say only that parentheses are unnecessary
when combining three statements with $\land$ or $\lor$. In fact, these laws can be
used to justify leaving parentheses out when more than three statements
are combined. Use associative laws to show that $[P \land (Q \land R)] \land S$ is
equivalent to $(P \land Q) \land (R \land S)$

$[P \land (Q \land R)] \land S = [(P \land Q) \land R] \land S = (P \land Q) \land (R \land S)$

\vspace{30pt}

15. How many lines will there be in the truth table for a statement containing
n letters?

$2^n$ lines

\vspace{30pt}

16. Find a formula involving the connectives $\land$, $\lor$, and $\neg$ that has the following
truth table:

\centerline{
  \begin{tabular}{c c l}
  P & Q & $ P \lor \neg Q$ \\
  \hline
  F & F & T \\
  F & T & F \\
  T & F & T \\
  T & T & T \\
  \end{tabular}}
\vspace{30pt}

17. Find a formula involving the connectives $\land$, $\lor$, and $\neg$ that has the following
truth table:

\centerline{
  \begin{tabular}{c c l}
  P & Q & $\neg (P \land Q) \land (P \lor Q)$ \\
  \hline
  F & F & F \\
  F & T & T \\
  T & F & T \\
  T & T & F \\
  \end{tabular}}
\vspace{30pt}

18. Suppose the conclusion of an argument is a tautology. What can you
conclude about the validity of the argument? What if the conclusion is
a contradiction? What if one of the premises is either a tautology or a
contradiction?
\vspace{20pt}

If a conclusion is a tautology then the argument is always valid.
If a conclusion is a contradiction then the argument is always invalid.
If one of the premises is a tautology then this premise may be ignored.
If one of the premises is a contradiction then the argument is invalid.
\vspace{50pt}

\textbf{1.3. Variables and Sets}
\vspace{40pt}

1. Analyze the logical forms of the following statements:

\hspace{12pt}(a) 3 is a common divisor of 6, 9, and 15. (Note: You did this in exercise
2 of Section 1.1, but you should be able to give a better answer now.)

\hspace{12pt}(b) x is divisible by both 2 and 3 but not 4.

\hspace{12pt}(c) x and y are natural numbers, and exactly one of them is prime.
\vspace{20pt}

(a) Let D(x) stand for "x is disible by 3" then
$$D(6) \land D(9) \land D(15)$$
\vspace{10pt}

(b) Let D(y, x) stand for "x is divisible by y" then
$$D(2, x) \land D(3, x) \land \neg D(4, x)$$
\vspace{10pt}

(c) Let N(x) stand for "x is natural" and P(x) stand for "x is prime".
\vspace{10pt}

$$N(x) \land N(y) \land [(P(x) \land \neg P(y)) \lor (\neg P(x) \land P(y))]$$
\vspace{30pt}

2.Analyze the logical forms of the following statements:

\hspace{12pt}(a) x and y are men, and either x is taller than y or y is taller than x.

\hspace{12pt}(b) Either x or y has brown eyes, and either x or y has red hair.

\hspace{12pt}(c) Either x or y has both brown eyes and red hair.
\vspace{20pt}

(a) Let P(x) stand for "x is a man" and H(x, y) is "x is taller than y" then

$$P(x) \land P(y) \land [H(x, y) \lor H(y, x)]$$
\vspace{10pt}

(b) Let P(x) stand for "x has brown eyes" and H(x) is "x has red hair" then

$$[P(x) \lor P(y)] \lor [H(x) \lor H(y)]$$
\vspace{10pt}

(c) P(x) stand for "x has brown eyes and red hair" then

$$P(x) \lor P(y)$$
\vspace{30pt}

3. Write definitions using elementhood tests for the following sets:

\hspace{12pt}(a) \{Mercury, Venus, Earth, Mars, Jupiter, Saturn, Uranus, Neptune,
Pluto\}.

\hspace{12pt}(b) \{Brown, Columbia, Cornell, Dartmouth, Harvard, Princeton, University
of Pennsylvania, Yale\}.

\hspace{12pt}(c) \{Alabama, Alaska, Arizona, ... , Wisconsin, Wyoming\}.

\hspace{12pt}(d) \{Alberta, British Columbia, Manitoba, New Brunswick, Newfoundland
and Labrador, Northwest Territories, Nova Scotia, Nunavut, Ontario,
Prince Edward Island, Quebec, Saskatchewan, Yukon\}.
\vspace{20pt}

(a) $P = \{ p \mid \text{p is a planet} \}$
\vspace{10pt}

(b) $U = \{ p \mid \text{p is an Ivy League school} \}$
\vspace{10pt}

(c) $U = \{ p \mid \text{p is a state in the USA} \}$
\vspace{10pt}

(c) $C = \{ p \mid \text{p is a province of territory in Canada} \}$
\vspace{30pt}

4. Write definitions using elementhood tests for the following sets:

\hspace{12pt}(a) \{1, 4, 9, 16, 25, 36, 49,...\}.

\hspace{12pt}(b) \{1, 2, 4, 8, 16, 32, 64,...\}.

\hspace{12pt}(c) \{10, 11, 12, 13, 14, 15, 16, 17, 18, 19\}.
\vspace{20pt}

(a) $$A = \{ x \in \mathbb{N} \mid x > 0 \}$$
$$B = \{ y \mid y = x^2, x \in A \}$$
\vspace{10pt}

(b) $$A = \{ y \mid y = 2^n, n \in \mathbb{N}\}$$
\vspace{10pt}

(c) $$A = \{x \in \mathbb{N} \mid x >= 10 \}$$
$$B = \{x \in A \mid x < 19 \}$$
\vspace{30pt}

5. Simplify the following statements. Which variables are free and which are
bound? If the statement has no free variables, say whether it is true or
false.

\hspace{12pt}(a) $-3 \in \{x \in \mathbb{R} \mid 13 - 2x > 1 \}$

\hspace{12pt}(b) $4 \in \{x \in \mathbb{R}^- \mid 13 - 2x > 1 \}$

\hspace{12pt}(c) $5 \notin \{x \in \mathbb{R} \mid 13 - 2x > c\}$
\vspace{20pt}

(a) $(-3 \in \mathbb{R}) \land ((13+6) > 1)$ No free variables; x - bound; True.
\vspace{10pt}

(b) $(4 \in \mathbb{R}) \land (4 < 0) \land ((13 - 8) > 1)$ No free variables; x - bound; False.
\vspace{10pt}

(c) $\neg [(5 \in \mathbb{R}) \land ((13 - 10) > c)]$ c - free variable; x - bound.
\vspace{30pt}

6. Simplify the following statements. Which variables are free and which are
bound? If the statement has no free variables, say whether it is true or
false.

\hspace{12pt}(a) $\omega \in \{x \in \mathbb{R} \mid 13 - 2x > c\}$

\hspace{12pt}(b) $4 \in \{x \in \mathbb{R} \mid 13 - 2x \in \{y \mid y \text{ is a prime number}\}\}$. (It might make
this statement easier to read if we let $P = \{y \mid y \text{ is a prime number}\}$;
using this notation, we could rewrite the statement as $4 \in \{x \in \mathbb{R} \mid 13 - 2x \in P\}$.)

\hspace{12pt}(c) $4 \in \{x \in \{y \mid y \text{ is a prime number}\} \mid 13 - 2x > 1\}$. (Using the same notation
as in part (b), we could write this as $4 \in \{x \in P \mid 13 - 2x > 1\}$.)
\vspace{20pt}

(a) $(\omega \in \mathbb{R}) \land (13 - 2\omega > c)$ x - bound; $\omega$ and c are free variables.
\vspace{10pt}

(b) $(4 \in \mathbb{R}) \land ((13 - 8) \text{ is prime})$ True; no free variables; x, y - bound variables.
\vspace{10pt}

(c) $(4 \text{ is prime}) \land (5 > 1)$ False; no free variables; x, y - bound variables.
\vspace{30pt}

7. What are the truth sets of the following statements? List a few elements of
the truth set if you can.

\hspace{12pt}(a) Elizabeth Taylor was once married to x.

\hspace{12pt}(b) x is a logical connective studied in Section 1.1.

\hspace{12pt}(c) x is the author of this book.
\vspace{20pt}

(a) $\{x \mid \text{Elizabeth Taylor was once married on x}\}$ = \{Conrad Hilton, Michael Wilding, Mike Todd ...\}
\vspace{10pt}

(b) $\{x \mid x \text{ is logical connective studied in Section 1.1}\} = \{\lor, \land \neg\}$

(c) $\{x \mid x \text{ is author of this book}\}$ = \{Daniel J. Velleman\}
\vspace{30pt}

8. What are the truth sets of the following statements? List a few elements of
the truth set if you can.

\hspace{12pt}(a) $x$ is a real number and $x^2 - 4x + 3 = 0$.

\hspace{12pt}(b) $x$ is a real number and $x^2 - 2x + 3 = 0$.

\hspace{12pt}(c) $x$ is a real number and $5 \in \{y \in \mathbb{R} \mid x^2 + y^2 < 50\}$.
\vspace{20pt}

(a)

Vieta's formulas

$$x * y = 3$$
$$x + y = -4/1$$

$$\therefore \{1, 3\}$$

\vspace{10pt}

(b)

$$D = b^2 - 4ac = (-2)^2 - 4*1*3 = 4 - 12 = -8$$

$$\therefore \emptyset$$

\vspace{10pt}

(c)

$$(5 \in \mathbb{R}) \land (x^2 + 25 < 50)$$
$$x^2 < 25$$
$$\therefore \{x \in \mathbb{R} \mid x^2 < 25\}$$
$$e.g \{-4.9, -3, 0, 3, 4.9, \dots\}$$
\vspace{50pt}

\textbf{1.4. Operations on Sets}
\vspace{30pt}

\def\firstcircle{(0,0) circle (1.5cm)}
\def\secondcircle{(45:2cm) circle (1.5cm)}
\def\thirdcircle{(0:2cm) circle (1.5cm)}

$$A \cup (B \cap C) = (A \cup B) \cap (A \cup C)$$

$$B \cap C$$

\centerline{\begin{tikzpicture}
    \begin{scope}
      \clip \secondcircle;
      \fill[cyan] \thirdcircle;
    \end{scope}
    \begin{scope}
      \clip \firstcircle;
    \end{scope}
    \draw \firstcircle node[text=black,above] {$A$};
    \draw \secondcircle node [text=black,below left] {$B$};
    \draw \thirdcircle node [text=black,below right] {$C$};
  \end{tikzpicture}}

$$A \cup (B \cap C)$$

\centerline{\begin{tikzpicture}
    \begin{scope}
      \clip \secondcircle;
      \fill[cyan] \thirdcircle;
    \end{scope}
    \begin{scope}
      \fill[cyan] \firstcircle;
    \end{scope}
    \draw \firstcircle node[text=black,above] {$A$};
    \draw \secondcircle node [text=black,below left] {$B$};
    \draw \thirdcircle node [text=black,below right] {$C$};
  \end{tikzpicture}}
\vspace{40pt}

$$A \cup B$$

\centerline{\begin{tikzpicture}
    \begin{scope}
      \fill[cyan] \firstcircle;
      \fill[cyan] \secondcircle;
    \end{scope}
    \draw \firstcircle node[text=black,above] {$A$};
    \draw \secondcircle node [text=black,below left] {$B$};
    \draw \thirdcircle node [text=black,below right] {$C$};
  \end{tikzpicture}}

$$A \cup C$$

\centerline{\begin{tikzpicture}
    \begin{scope}
      \fill[cyan] \firstcircle;
      \fill[cyan] \thirdcircle;
    \end{scope}
    \draw \firstcircle node[text=black,above] {$A$};
    \draw \secondcircle node [text=black,below left] {$B$};
    \draw \thirdcircle node [text=black,below right] {$C$};
  \end{tikzpicture}}

$$(A \cup B) \cap (A \cup C)$$

\centerline{\begin{tikzpicture}
    \begin{scope}
      \clip \secondcircle;
      \fill[cyan] \thirdcircle;
    \end{scope}
    \begin{scope}
      \fill[cyan] \firstcircle;
    \end{scope}
    \draw \firstcircle node[text=black,above] {$A$};
    \draw \secondcircle node [text=black,below left] {$B$};
    \draw \thirdcircle node [text=black,below right] {$C$};
  \end{tikzpicture}}
\vspace{40pt}

$$A \setminus (B \cap C) = (A \setminus B) \cup (A \setminus C)$$

$$B \cap C$$

\centerline{\begin{tikzpicture}
    \begin{scope}
      \clip \secondcircle;
      \fill[cyan] \thirdcircle;
    \end{scope}
    \begin{scope}
      \clip \firstcircle;
    \end{scope}
    \draw \firstcircle node[text=black,above] {$A$};
    \draw \secondcircle node [text=black,below left] {$B$};
    \draw \thirdcircle node [text=black,below right] {$C$};
  \end{tikzpicture}}

$$A$$

\centerline{\begin{tikzpicture}
    \begin{scope}
      \fill[cyan] \firstcircle;
    \end{scope}
    \begin{scope}
      \clip \firstcircle;
    \end{scope}
    \draw \firstcircle node[text=black,above] {$A$};
    \draw \secondcircle node [text=black,below left] {$B$};
    \draw \thirdcircle node [text=black,below right] {$C$};
  \end{tikzpicture}}

$$A \setminus (B \cap C)$$

\centerline{\begin{tikzpicture}
  \begin{scope}[shift={(6cm,0cm)}]
    \begin{scope}
      \fill[cyan] \firstcircle;
    \end{scope}
    \begin{scope}
      \clip \firstcircle;
      \clip \secondcircle;
      \fill[white] \thirdcircle;
    \end{scope}
    \draw \firstcircle node {$A$};
    \draw \secondcircle node {$B$};
    \draw \thirdcircle node {$C$};
  \end{scope}
\end{tikzpicture}}
\vspace{40pt}

$$(A \setminus B) \cup (A \setminus C)$$

$$A \setminus B$$

\centerline{\begin{tikzpicture}
    \begin{scope}[shift={(6cm,0cm)}]
      \begin{scope}[even odd rule]
        \clip \secondcircle (-3,-3) rectangle (3,3);
        \fill[cyan] \firstcircle;
      \end{scope}
      \draw \firstcircle node {$A$};
      \draw \secondcircle node {$B$};
      \draw \thirdcircle node {$C$};
    \end{scope}
  \end{tikzpicture}}

$$A \setminus C$$

\centerline{\begin{tikzpicture}
    \begin{scope}[shift={(6cm,0cm)}]
      \begin{scope}[even odd rule]
        \clip \thirdcircle (-3,-3) rectangle (3,3);
        \fill[cyan] \firstcircle;
      \end{scope}
      \draw \firstcircle node {$A$};
      \draw \secondcircle node {$B$};
      \draw \thirdcircle node {$C$};
    \end{scope}
  \end{tikzpicture}}

$$(A \setminus B) \cup (A \setminus C)$$


\centerline{\begin{tikzpicture}
  \begin{scope}[shift={(6cm,0cm)}]
    \begin{scope}
      \fill[cyan] \firstcircle;
    \end{scope}
    \begin{scope}
      \clip \firstcircle;
      \clip \secondcircle;
      \fill[white] \thirdcircle;
    \end{scope}
    \draw \firstcircle node {$A$};
    \draw \secondcircle node {$B$};
    \draw \thirdcircle node {$C$};
  \end{scope}
\end{tikzpicture}}
\vspace{40pt}

1. Let A = \{1, 3, 12, 35\}, B = \{3, 7, 12, 20\}, and C = $\{x \mid x \text{ is a prime
number}\}$. List the elements of the following sets. Are any of the sets
below disjoint from any of the others? Are any of the sets below subsets
of any others?

\hspace{12pt}(a) $A \cap B$.

\hspace{12pt}(b) $(A \cup B) \setminus C$.

\hspace{12pt}(c) $A \cup (B \setminus C)$.
\vspace{20pt}

(a) $P = A \cap B = \{1, 3, 12, 35\} \cap \{3, 7, 12, 20\} = \{3, 12\}$
\vspace{10pt}

(b) $$A \cup B = \{1, 3, 12, 35\} \cup \{3, 7, 12, 20\} = \{1, 3, 7, 12, 20, 35\}$$
$$Q = (A \cup B) \setminus C = \{1, 12, 20, 35\}$$
\vspace{10pt}

(c)  $$B \setminus C = \{12, 20\}$$
$$R = A \cup (B \setminus C) = \{1, 3, 12, 20, 35\}$$
\vspace{10pt}

No disjonts; $P \subseteq R$ and $Q \subseteq R$

2. Let A = \{United States, Germany, China, Australia\}, B = \{Germany,
France, India, Brazil\}, and $C = \{x \mid x \text { is a country in Europe}\}$. List the
elements of the following sets. Are any of the sets below disjoint from
any of the others? Are any of the sets below subsets of any others?

\hspace{12pt}(a) $A \cup B$.

\hspace{12pt}(b) $(A \cap B) \setminus C$.

\hspace{12pt}(c) $(B \cap C) \setminus A$.
\vspace{20pt}

(a) P = \{United States, Germany, China, Australia, France, India, Brazil\}
\vspace{10pt}

(b) $A \cap B$ = \{Germany\}

$Q = (A \cap B) \setminus C = \emptyset$
\vspace{10pt}

(c) $(B \cap C)$ = \{Germany, France\}

$R = (B \cap C) \setminus A$ = \{France\}.
\vspace{10pt}

$$Q \subseteq P \text{ and } Q \subseteq R$$
$$R \subseteq P$$
\vspace{30pt}

3. Verify that the Venn diagrams for $(A \cup B) \setminus (A \cap B)$ and $(A \setminus B) \cup
(B \setminus A)$ both look like Figure 5, as stated in this section.

$$(A \cup B) \setminus (A \cap B)$$

$$A \cup B$$

\centerline{\begin{tikzpicture}
    \begin{scope}
      \fill[cyan] \firstcircle;
      \fill[cyan] \secondcircle;
    \end{scope}
    \draw \firstcircle node[text=black,above] {$A$};
    \draw \secondcircle node [text=black,below left] {$B$};
  \end{tikzpicture}}

$$A \cap B$$

\centerline{\begin{tikzpicture}
    \begin{scope}
      \clip \firstcircle;
      \fill[cyan] \secondcircle;
    \end{scope}
    \begin{scope}
      \clip \firstcircle;
    \end{scope}
    \draw \firstcircle node[text=black,above] {$A$};
    \draw \secondcircle node [text=black,below left] {$B$};
  \end{tikzpicture}}

$$(A \cup B) \setminus (A \cap B)$$

\centerline{\begin{tikzpicture}
    \begin{scope}[shift={(6cm,0cm)}]
      \begin{scope}[even odd rule]
        \clip \secondcircle (-3,-3) rectangle (3,3);
        \fill[cyan] \firstcircle;
      \end{scope}
      \begin{scope}[even odd rule]
        \clip \firstcircle (-3,-3) rectangle (3,3);
        \fill[cyan] \secondcircle;
      \end{scope}
      \draw \firstcircle node {$A$};
      \draw \secondcircle node {$B$};
    \end{scope}
  \end{tikzpicture}}
\vspace{30pt}

$$(A \setminus B) \cup (B \setminus A)$$

$$A \setminus B$$

\centerline{\begin{tikzpicture}
    \begin{scope}[shift={(6cm,0cm)}]
      \begin{scope}[even odd rule]
        \clip \secondcircle (-3,-3) rectangle (3,3);
        \fill[cyan] \firstcircle;
      \end{scope}
      \draw \firstcircle node {$A$};
      \draw \secondcircle node {$B$};
    \end{scope}
  \end{tikzpicture}}

$$B \setminus A$$

\centerline{\begin{tikzpicture}
    \begin{scope}[shift={(6cm,0cm)}]
      \begin{scope}[even odd rule]
        \clip \firstcircle (-3,-3) rectangle (3,3);
        \fill[cyan] \secondcircle;
      \end{scope}
      \draw \firstcircle node {$A$};
      \draw \secondcircle node {$B$};
    \end{scope}
  \end{tikzpicture}}

$$(A \setminus B) \cup (B \setminus A)$$

\centerline{\begin{tikzpicture}
    \begin{scope}[shift={(6cm,0cm)}]
      \begin{scope}[even odd rule]
        \clip \secondcircle (-3,-3) rectangle (3,3);
        \fill[cyan] \firstcircle;
      \end{scope}
      \begin{scope}[even odd rule]
        \clip \firstcircle (-3,-3) rectangle (3,3);
        \fill[cyan] \secondcircle;
      \end{scope}
      \draw \firstcircle node {$A$};
      \draw \secondcircle node {$B$};
    \end{scope}
  \end{tikzpicture}}
\vspace{30pt}

4. Use Venn diagrams to verify the following identities:

\hspace{12pt}(a) $A \setminus (A \cap B) = A \setminus B$.

\hspace{12pt}(b) $A \cup (B \cap C) = (A \cup B) \cap (A \cup C)$.
\vspace{20pt}

(a) $$A \setminus (A \cap B) = A \setminus B$$

$$A \cap B$$

\centerline{\begin{tikzpicture}
    \begin{scope}
      \clip \firstcircle;
      \fill[cyan] \secondcircle;
    \end{scope}
    \begin{scope}
      \clip \firstcircle;
    \end{scope}
    \draw \firstcircle node[text=black,above] {$A$};
    \draw \secondcircle node [text=black,below left] {$B$};
  \end{tikzpicture}}

$$A$$

\centerline{\begin{tikzpicture}
    \begin{scope}
      \fill[cyan] \firstcircle;
    \end{scope}
    \begin{scope}
      \clip \firstcircle;
    \end{scope}
    \draw \firstcircle node[text=black,above] {$A$};
    \draw \secondcircle node [text=black,below left] {$B$};
  \end{tikzpicture}}

$$A \setminus (A \cap B) = A \setminus B$$

\centerline{\begin{tikzpicture}
    \begin{scope}[shift={(6cm,0cm)}]
      \begin{scope}[even odd rule]
        \clip \secondcircle (-3,-3) rectangle (3,3);
        \fill[cyan] \firstcircle;
      \end{scope}
      \draw \firstcircle node {$A$};
      \draw \secondcircle node {$B$};
    \end{scope}
  \end{tikzpicture}}




\vspace{10pt}

(b) $$A \cup (B \cap C) = (A \cup B) \cap (A \cup C)$$

\centerline{Done at the beginning of the section.}

\vspace{30pt}

5. Verify the identities in exercise 4 by writing out (using logical symbols)
what it means for an object x to be an element of each set and then using
logical equivalences.
\vspace{30pt}

(a) $$A \setminus (A \cap B) = A \setminus B$$

$$A \setminus (A \cap B) = x \in A \land \neg (x \in A \land x \in B)$$
DeMorgan's law: $$x \in A \land (x \notin A \lor x \notin B)$$
Distributive law: $$(x \in A \land x \notin A) \lor (x \in A \land x \notin B)$$
\centerline{(a contradiction) $\lor (x \in A \land x \notin B)$}
Contradiction law: $$x \in A \land x \notin B$$

$$A \setminus B = x \in A \land x \notin B$$
\vspace{10pt}

(b) $$A \cup (B \cap C) = (A \cup B) \cap (A \cup C)$$
$$x \in A \lor x \in (B \cap C)$$
$$x \in A \lor (x \in B \land x \in C)$$
Distributive law: $$(x \in A \lor x \in B) \land (x \in A \lor x \in C)$$

$$(A \cup B) \cap (A \cup C)$$
$$(x \in A \lor x \in B) \land (x \in A \lor x \in C)$$
\vspace{30pt}

6. Use Venn diagrams to verify the following identities:

\hspace{12pt}(a) $(A \cup B) \setminus C = (A \setminus C) \cup (B \setminus C)$.

\hspace{12pt}(b) $A \cup (B \setminus C) = (A \cup B) \setminus (C \setminus A)$.
\vspace{30pt}

(a) $$(A \cup B) \setminus C$$

$$(A \cup B)$$

\centerline{\begin{tikzpicture}
    \begin{scope}
      \fill[cyan] \firstcircle;
      \fill[cyan] \secondcircle;
    \end{scope}
    \draw \firstcircle node[text=black,above] {$A$};
    \draw \secondcircle node [text=black,below left] {$B$};
    \draw \thirdcircle node [text=black,below right] {$C$};
  \end{tikzpicture}}

$$(A \cup B) \setminus C$$

\centerline{\begin{tikzpicture}
    \begin{scope}[shift={(6cm,0cm)}]
      \begin{scope}[even odd rule]
        \clip \thirdcircle (-3,-3) rectangle (3,3);
        \fill[cyan] \firstcircle;
      \end{scope}
      \begin{scope}[even odd rule]
        \clip \thirdcircle (-3,-3) rectangle (3,3);
        \fill[cyan] \secondcircle;
      \end{scope}
      \draw \firstcircle node {$A$};
      \draw \secondcircle node {$B$};
      \draw \thirdcircle node {$C$};
    \end{scope}
  \end{tikzpicture}}
\vspace{10pt}

$$(A \setminus C) \cup (B \setminus C)$$

$$A \setminus C$$

\centerline{\begin{tikzpicture}
    \begin{scope}[shift={(6cm,0cm)}]
      \begin{scope}[even odd rule]
        \clip \thirdcircle (-3,-3) rectangle (3,3);
        \fill[cyan] \firstcircle;
      \end{scope}
      \draw \firstcircle node {$A$};
      \draw \secondcircle node {$B$};
      \draw \thirdcircle node {$C$};
    \end{scope}
  \end{tikzpicture}}

$$B \setminus C$$

\centerline{\begin{tikzpicture}
    \begin{scope}[shift={(6cm,0cm)}]
      \begin{scope}[even odd rule]
        \clip \thirdcircle (-3,-3) rectangle (3,3);
        \fill[cyan] \secondcircle;
      \end{scope}
      \draw \firstcircle node {$A$};
      \draw \secondcircle node {$B$};
      \draw \thirdcircle node {$C$};
    \end{scope}
  \end{tikzpicture}}

$$(A \setminus C) \cup (B \setminus C)$$

\centerline{\begin{tikzpicture}
    \begin{scope}[shift={(6cm,0cm)}]
      \begin{scope}[even odd rule]
        \clip \thirdcircle (-3,-3) rectangle (3,3);
        \fill[cyan] \firstcircle;
      \end{scope}
      \begin{scope}[even odd rule]
        \clip \thirdcircle (-3,-3) rectangle (3,3);
        \fill[cyan] \secondcircle;
      \end{scope}
      \draw \firstcircle node {$A$};
      \draw \secondcircle node {$B$};
      \draw \thirdcircle node {$C$};
    \end{scope}
  \end{tikzpicture}}
\vspace{20pt}

(b) $A \cup (B \setminus C) = (A \cup B) \setminus (C \setminus A)$

$A \cup (B \setminus C)$

$$B \setminus C$$

\centerline{\begin{tikzpicture}
    \begin{scope}[shift={(6cm,0cm)}]
      \begin{scope}[even odd rule]
        \clip \thirdcircle (-3,-3) rectangle (3,3);
        \fill[cyan] \secondcircle;
      \end{scope}
      \draw \firstcircle node {$A$};
      \draw \secondcircle node {$B$};
      \draw \thirdcircle node {$C$};
    \end{scope}
  \end{tikzpicture}}

$$A \cup (B \setminus C)$$

\centerline{\begin{tikzpicture}
    \begin{scope}[shift={(6cm,0cm)}]
      \begin{scope}
        \fill[cyan] \firstcircle;
      \end{scope}
      \begin{scope}[even odd rule]
        \clip \thirdcircle (-3,-3) rectangle (3,3);
        \fill[cyan] \secondcircle;
      \end{scope}
      \draw \firstcircle node {$A$};
      \draw \secondcircle node {$B$};
      \draw \thirdcircle node {$C$};
    \end{scope}
  \end{tikzpicture}}

$$(A \cup B) \setminus (C \setminus A)$$

$$A \cup B$$

\centerline{\begin{tikzpicture}
    \begin{scope}
      \fill[cyan] \firstcircle;
      \fill[cyan] \secondcircle;
    \end{scope}
    \draw \firstcircle node[text=black,above] {$A$};
    \draw \secondcircle node [text=black,below left] {$B$};
    \draw \thirdcircle node [text=black,below right] {$C$};
  \end{tikzpicture}}

$$C \setminus A$$

\centerline{\begin{tikzpicture}
    \begin{scope}[shift={(6cm,0cm)}]
      \begin{scope}[even odd rule]
        \clip \firstcircle (-5,-5) rectangle (5,5);
        \fill[cyan] \thirdcircle;
      \end{scope}
      \draw \firstcircle node {$A$};
      \draw \secondcircle node {$B$};
      \draw \thirdcircle node {$C$};
    \end{scope}
  \end{tikzpicture}}

$$(A \cup B) \setminus (C \setminus A)$$

\centerline{\begin{tikzpicture}
    \begin{scope}[shift={(6cm,0cm)}]
      \begin{scope}
        \fill[cyan] \firstcircle;
      \end{scope}
      \begin{scope}[even odd rule]
        \clip \thirdcircle (-3,-3) rectangle (3,3);
        \fill[cyan] \secondcircle;
      \end{scope}
      \draw \firstcircle node {$A$};
      \draw \secondcircle node {$B$};
      \draw \thirdcircle node {$C$};
    \end{scope}
  \end{tikzpicture}}
\vspace{30pt}

7. Verify the identities in exercise 6 by writing out (using logical symbols)
what it means for an object x to be an element of each set and then using
logical equivalences.

\hspace{12pt}(a) $(A \cup B) \setminus C = (A \setminus C) \cup (B \setminus C)$.

\hspace{12pt}(b) $A \cup (B \setminus C) = (A \cup B) \setminus (C \setminus A)$.
\vspace{30pt}

(a) $$(A \cup B) \setminus C$$
$$(x \in A \lor x \in B) \land \neg (x \in C)$$
Distributive law: $$(x \in A \land \neg (x \in C)) \lor (x \in B \land \neg (x \in C))$$
$$(x \in A \land x \notin C) \lor (x \in B \land x \notin C)$$

$$(A \setminus C) \cup (B \setminus C)$$
$$(x \in A \land x \notin C) \lor (x \in B \land x \notin C)$$
\vspace{10pt}

(b) $$A \cup (B \setminus C)$$
$$x \in A \lor (x \in B \land x \notin C)$$

$$(A \cup B) \setminus (C \setminus A)$$
$$(x \in A \lor x \in B) \land \neg (x \in C \land \neg (x \in A))$$
DeMorgan's law: $$(x \in A \lor x \in B) \land (x \notin C \lor x \in A)$$
Commutative law: $$(x \in A \lor x \in B) \land (x \in A \lor x \notin C)$$
Distributive law: $$x \in A \lor (x \in B \land x \notin C)$$
\vspace{30pt}

8. For each of the following sets, write out (using logical symbols) what it
means for an object x to be an element of the set. Then determine which
of these sets must be equal to each other by determining which statements
are equivalent.

\hspace{12pt}(a) $(A \setminus B) \setminus C$.

\hspace{12pt}(b) $A \setminus (B \setminus C)$.

\hspace{12pt}(c) $(A \setminus B) \cup (A \cap C)$.

\hspace{12pt}(d) $(A \setminus B) \cap (A \setminus C)$.

\hspace{12pt}(e) $A \setminus (B \cup C)$
\vspace{30pt}

(a) $$(x \in A \land x \notin B) \land x \notin C$$
$$x \in A \land x \notin B \land x \notin C$$
\vspace{10pt}

(b) $$x \in A \land \neg (x \in B \land \neg (x \in C))$$
$$x \in A \land (x \notin B \lor x \in C)$$
\vspace{10pt}

(c) $$(A \setminus B) \cup (A \cap C)$$
$$(x \in A \land x \notin B) \lor (x \in A \land x \in C)$$
$$x \in A \land (x \notin B \lor x \in C)$$
\vspace{10pt}

(d) $$(A \setminus B) \cap (A \setminus C)$$
$$(x \in A \land x \notin B) \land (x \in A \land x \notin C)$$
$$x \in A \land x \in A \land x \notin B \land x \notin C$$
$$x \in A \land x \notin B \land x \notin C$$
\vspace{10pt}

(e) $$A \setminus (B \cup C)$$
$$x \in A \land \neg (x \in B \lor x \in C)$$
$$x \in A \land x \notin B \land \notin C$$
\vspace{10pt}

Set (a) equals to set (d) and (e).

Set (b) equals to set (c).
\vspace{30pt}

9. It was shown in this section that for any sets A and B, $(A \cup B) \setminus B \subseteq A$.
Give an example of two sets A and B for which $(A \cup B) \setminus B \neq A$
\vspace{30pt}

$$A = \{1,2,3\}$$
$$B = \{3\}$$
$$A \cup B = \{1,2,3\}$$
$$(A \cup B) \setminus B = \{1, 2\}$$
$$\{1,2\} \neq \{1,2,3\}$$
\vspace{30pt}

10. It is claimed in this section that you cannot make a Venn diagram for four sets using overlapping circles.

\hspace{12pt}(a) What’s wrong with the following diagram? (Hint: Where’s the set $(A \cap D) \setminus (B \cup C)$?)

\def\firstcircle2{(0,0) circle (1.5cm)}
\def\secondcircle2{(0:1cm) circle (1.5cm)}
\def\thirdcircle2{(90:1cm) circle (1.5cm)}
\def\fourthcircle2{(45:1.4cm) circle (1.5cm)}


\centerline{$A$\hspace{50pt}$B$}

\centerline{\begin{tikzpicture}
    \draw \firstcircle2 node[text=black,below left] {};
    \draw \secondcircle2 node [text=black,above left] {};
    \draw \thirdcircle2 node [text=black,below right] {};
    \draw \fourthcircle2 node [text=black,above right] {};
  \end{tikzpicture}}

\centerline{$C$\hspace{50pt}$D$}

\hspace{12pt}(b) Can you make a Venn diagram for four sets using shapes other than
circles?
\vspace{30pt}

(a) $$A \cap D$$

\centerline{$A$\hspace{50pt}$B$}

\centerline{\begin{tikzpicture}
    \begin{scope}
      \clip \thirdcircle2;
      \fill[cyan] \secondcircle2;
    \end{scope}
    \draw \firstcircle2 node[text=black,below left] {};
    \draw \secondcircle2 node [text=black,above left] {};
    \draw \thirdcircle2 node [text=black,below right] {};
    \draw \fourthcircle2 node [text=black,above right] {};
  \end{tikzpicture}}

\centerline{$C$\hspace{50pt}$D$}
\vspace{30pt}
$$B \cup C$$

\centerline{$A$\hspace{50pt}$B$}

\centerline{\begin{tikzpicture}
    \begin{scope}
      \fill[cyan] \firstcircle2;
      \fill[cyan] \fourthcircle2;
    \end{scope}
    \draw \firstcircle2 node[text=black,below left] {};
    \draw \secondcircle2 node [text=black,above left] {};
    \draw \thirdcircle2 node [text=black,below right] {};
    \draw \fourthcircle2 node [text=black,above right] {};
  \end{tikzpicture}}

\centerline{$C$\hspace{50pt}$D$}

There's no region corresponding to set $(A \cap D) \setminus (B \cup C)$, but this set could have elements. (hmm)

\vspace{30pt}


(b)

\def\thirdellipse2{(90:1cm) ellipse (1.5cm and 2cm)}

\centerline{$A$\hspace{50pt}$B$}

\centerline{\begin{tikzpicture}
    \draw \firstcircle2 node[text=black,below left] {};
    \draw \secondcircle2 node [text=black,above left] {};
    \draw \thirdellipse2 node [text=black,below right] {};
    \draw \fourthcircle2 node [text=black,above right] {};
  \end{tikzpicture}}

\centerline{$C$\hspace{50pt}$D$}

\vspace{30pt}

11. (a) Make Venn diagrams for the sets $(A \cup B) \setminus C$ and $A \cup (B \setminus C)$. What
can you conclude about whether one of these sets is necessarily a
subset of the other?

\hspace{12pt}(b) Give an example of sets A, B, and C for which $(A \cup B) \setminus C \neq A \cup
(B \setminus C)$.

\vspace{30pt}

(a) $$Q = (A \cup B) \setminus C$$

\def\firstcircle{(0,0) circle (1.5cm)}
\def\secondcircle{(45:2cm) circle (1.5cm)}
\def\thirdcircle{(0:2cm) circle (1.5cm)}

\centerline{\begin{tikzpicture}
    \begin{scope}[shift={(6cm,0cm)}]
      \begin{scope}[even odd rule]
        \clip \thirdcircle (-3,-3) rectangle (3,3);
        \fill[cyan] \firstcircle;
      \end{scope}
      \begin{scope}[even odd rule]
        \clip \thirdcircle (-3,-3) rectangle (3,3);
        \fill[cyan] \secondcircle;
      \end{scope}
      \draw \firstcircle node {$A$};
      \draw \secondcircle node {$B$};
      \draw \thirdcircle node {$C$};
    \end{scope}
  \end{tikzpicture}}
\vspace{10pt}

$$A \cup (B \setminus C)$$

$$B \setminus C$$

\centerline{\begin{tikzpicture}
    \begin{scope}[shift={(6cm,0cm)}]
      \begin{scope}[even odd rule]
        \clip \thirdcircle (-3,-3) rectangle (3,3);
        \fill[cyan] \secondcircle;
      \end{scope}
      \draw \firstcircle node {$A$};
      \draw \secondcircle node {$B$};
      \draw \thirdcircle node {$C$};
    \end{scope}
  \end{tikzpicture}}

$$P = A \cup (B \setminus C)$$

\centerline{\begin{tikzpicture}
    \begin{scope}[shift={(6cm,0cm)}]
      \begin{scope}
        \fill[cyan] \firstcircle;
      \end{scope}
      \begin{scope}[even odd rule]
        \clip \thirdcircle (-3,-3) rectangle (3,3);
        \fill[cyan] \secondcircle;
      \end{scope}
      \draw \firstcircle node {$A$};
      \draw \secondcircle node {$B$};
      \draw \thirdcircle node {$C$};
    \end{scope}
  \end{tikzpicture}}

$$Q \subseteq P$$
\vspace{30pt}

(b) $$A = \{1,2,3\}$$
$$B = \{4, 5\}$$
$$C = \{1\}$$

$$(A \cup B) \setminus C$$
$$\{1,2,3,4,5\} \setminus \{1\}$$
$$Q = \{2,3,4,5\}$$

$$A \cup (B \setminus C)$$
$$(\{4,5\} \setminus {1}) \cup {1, 2, 3}$$
$$\{4,5\} \cup {1, 2, 3}$$
$$P = \{1,2,3,4,5\}$$

$$Q \neq P$$
\vspace{30pt}

12. Use Venn diagrams to show that the associative law holds for symmetric
difference; that is, for any sets A, B, and C, $A \triangle (B \triangle C) = (A \triangle B) \triangle C$.
\vspace{30pt}

$$A \triangle (B \triangle C)$$

$$P = B \triangle C$$

\centerline{\begin{tikzpicture}
    \begin{scope}[shift={(6cm,0cm)}]
      \begin{scope}[even odd rule]
        \clip \thirdcircle (-3,-3) rectangle (3,3);
        \fill[cyan] \secondcircle;
      \end{scope}
      \begin{scope}[even odd rule]
        \clip \secondcircle (-3,-3) rectangle (5,5);
        \fill[cyan] \thirdcircle;
      \end{scope}
      \draw \firstcircle node {$A$};
      \draw \secondcircle node {$B$};
      \draw \thirdcircle node {$C$};
    \end{scope}
  \end{tikzpicture}}

$$A \cup P$$

\centerline{\begin{tikzpicture}
    \begin{scope}[shift={(6cm,0cm)}]
      \begin{scope}[even odd rule]
        \fill[cyan] \firstcircle;
      \end{scope}
      \begin{scope}[even odd rule]
        \clip \thirdcircle (-3,-3) rectangle (3,3);
        \fill[cyan] \secondcircle;
      \end{scope}
      \begin{scope}[even odd rule]
        \clip \secondcircle (-3,-3) rectangle (5,5);
        \fill[cyan] \thirdcircle;
      \end{scope}
      \draw \firstcircle node {$A$};
      \draw \secondcircle node {$B$};
      \draw \thirdcircle node {$C$};
    \end{scope}
  \end{tikzpicture}}

$$A \cap P$$

\centerline{\begin{tikzpicture}
  \begin{scope}[shift={(6cm,0cm)}]
    \begin{scope}
      \clip \firstcircle;
      \fill[cyan] \secondcircle;
    \end{scope}
    \begin{scope}
      \clip \firstcircle;
      \fill[cyan] \thirdcircle;
    \end{scope}
    \begin{scope}
      \clip \firstcircle;
      \clip \secondcircle;
      \fill[white] \thirdcircle;
    \end{scope}
    \draw \firstcircle node {$A$};
    \draw \secondcircle node {$B$};
    \draw \thirdcircle node {$C$};
  \end{scope}
\end{tikzpicture}}

$$A \triangle (B \triangle C)$$

\centerline{\begin{tikzpicture}
    \begin{scope}[shift={(6cm,0cm)}]
      \begin{scope}[even odd rule]
        \clip \secondcircle (-3,-3) rectangle (3,3);
        \fill[cyan] \firstcircle;
      \end{scope}
      \begin{scope}[even odd rule]
        \clip \thirdcircle (-3,-3) rectangle (3,3);
        \fill[cyan] \secondcircle;
      \end{scope}
      \begin{scope}[even odd rule]
        \clip \firstcircle (-3,-3) rectangle (5,5);
        \fill[cyan] \thirdcircle;
      \end{scope}
      \begin{scope}
        \clip \secondcircle;
        \fill[white] \thirdcircle;
      \end{scope}
      \begin{scope}
        \clip \firstcircle;
        \fill[white] \secondcircle;
      \end{scope}
      \begin{scope}
        \clip \thirdcircle;
        \fill[white] \firstcircle;
      \end{scope}
      \begin{scope}
        \clip \firstcircle;
        \clip \secondcircle;
        \fill[cyan] \thirdcircle;
      \end{scope}
      \draw \firstcircle node {$A$};
      \draw \secondcircle node {$B$};
      \draw \thirdcircle node {$C$};
    \end{scope}
  \end{tikzpicture}}
\vspace{30pt}

$$(A \triangle B) \triangle C$$

$$Q = A \triangle B$$

\centerline{\begin{tikzpicture}
    \begin{scope}[shift={(6cm,0cm)}]
      \begin{scope}[even odd rule]
        \clip \firstcircle (-3,-3) rectangle (3,3);
        \fill[cyan] \secondcircle;
      \end{scope}
      \begin{scope}[even odd rule]
        \clip \secondcircle (-3,-3) rectangle (5,5);
        \fill[cyan] \firstcircle;
      \end{scope}
      \draw \firstcircle node {$A$};
      \draw \secondcircle node {$B$};
      \draw \thirdcircle node {$C$};
    \end{scope}
  \end{tikzpicture}}

$$Q \cup C$$

\centerline{\begin{tikzpicture}
    \begin{scope}[shift={(6cm,0cm)}]
      \begin{scope}[even odd rule]
        \fill[cyan] \thirdcircle;
      \end{scope}
      \begin{scope}[even odd rule]
        \clip \firstcircle (-3,-3) rectangle (3,3);
        \fill[cyan] \secondcircle;
      \end{scope}
      \begin{scope}[even odd rule]
        \clip \secondcircle (-3,-3) rectangle (5,5);
        \fill[cyan] \firstcircle;
      \end{scope}
      \draw \firstcircle node {$A$};
      \draw \secondcircle node {$B$};
      \draw \thirdcircle node {$C$};
    \end{scope}
  \end{tikzpicture}}

$$Q \cap C$$

\centerline{\begin{tikzpicture}
  \begin{scope}[shift={(6cm,0cm)}]
    \begin{scope}
      \clip \secondcircle;
      \fill[cyan] \thirdcircle;
    \end{scope}
    \begin{scope}
      \clip \firstcircle;
      \fill[cyan] \thirdcircle;
    \end{scope}
    \begin{scope}
      \clip \firstcircle;
      \clip \secondcircle;
      \fill[white] \thirdcircle;
    \end{scope}
    \draw \firstcircle node {$A$};
    \draw \secondcircle node {$B$};
    \draw \thirdcircle node {$C$};
  \end{scope}
\end{tikzpicture}}

$$(A \triangle B) \triangle C$$

\centerline{\begin{tikzpicture}
    \begin{scope}[shift={(6cm,0cm)}]
      \begin{scope}[even odd rule]
        \clip \secondcircle (-3,-3) rectangle (3,3);
        \fill[cyan] \firstcircle;
      \end{scope}
      \begin{scope}[even odd rule]
        \clip \thirdcircle (-3,-3) rectangle (3,3);
        \fill[cyan] \secondcircle;
      \end{scope}
      \begin{scope}[even odd rule]
        \clip \firstcircle (-3,-3) rectangle (5,5);
        \fill[cyan] \thirdcircle;
      \end{scope}
      \begin{scope}
        \clip \secondcircle;
        \fill[white] \thirdcircle;
      \end{scope}
      \begin{scope}
        \clip \firstcircle;
        \fill[white] \secondcircle;
      \end{scope}
      \begin{scope}
        \clip \thirdcircle;
        \fill[white] \firstcircle;
      \end{scope}
      \begin{scope}
        \clip \firstcircle;
        \clip \secondcircle;
        \fill[cyan] \thirdcircle;
      \end{scope}
      \draw \firstcircle node {$A$};
      \draw \secondcircle node {$B$};
      \draw \thirdcircle node {$C$};
    \end{scope}
  \end{tikzpicture}}
\vspace{30pt}

13. Use any method you wish to verify the following identities:

\hspace{12pt}(a) $(A \triangle B) \cup C = (A \cup C) \triangle (B \setminus C)$.

\hspace{12pt}(b) $(A \triangle B) \cap C = (A \cap C) \triangle (B \cap C)$.

\hspace{12pt}(c) $(A \triangle B) \setminus C = (A \setminus C) \triangle (B \setminus C)$.
\vspace{30pt}

(a) $$(A \triangle B) \cup C = (A \cup C) \triangle (B \setminus C)$$

$$A \triangle B$$

\centerline{\begin{tikzpicture}
    \begin{scope}[shift={(6cm,0cm)}]
      \begin{scope}[even odd rule]
        \clip \firstcircle (-3,-3) rectangle (3,3);
        \fill[cyan] \secondcircle;
      \end{scope}
      \begin{scope}[even odd rule]
        \clip \secondcircle (-3,-3) rectangle (5,5);
        \fill[cyan] \firstcircle;
      \end{scope}
      \draw \firstcircle node {$A$};
      \draw \secondcircle node {$B$};
      \draw \thirdcircle node {$C$};
    \end{scope}
  \end{tikzpicture}}

$$(A \triangle B) \cup C$$

\centerline{\begin{tikzpicture}
    \begin{scope}[shift={(6cm,0cm)}]
      \begin{scope}[even odd rule]
        \fill[cyan] \thirdcircle;
      \end{scope}
      \begin{scope}[even odd rule]
        \clip \firstcircle (-3,-3) rectangle (3,3);
        \fill[cyan] \secondcircle;
      \end{scope}
      \begin{scope}[even odd rule]
        \clip \secondcircle (-3,-3) rectangle (5,5);
        \fill[cyan] \firstcircle;
      \end{scope}
      \draw \firstcircle node {$A$};
      \draw \secondcircle node {$B$};
      \draw \thirdcircle node {$C$};
    \end{scope}
  \end{tikzpicture}}

$$P = A \cup C$$

\centerline{\begin{tikzpicture}
    \begin{scope}
      \fill[cyan] \firstcircle;
      \fill[cyan] \thirdcircle;
    \end{scope}
    \draw \firstcircle node[text=black,above] {$A$};
    \draw \secondcircle node [text=black,below left] {$B$};
    \draw \thirdcircle node [text=black,below right] {$C$};
  \end{tikzpicture}}

$$Q = B \setminus C$$

\centerline{\begin{tikzpicture}
    \begin{scope}[shift={(6cm,0cm)}]
      \begin{scope}[even odd rule]
        \clip \thirdcircle (-3,-3) rectangle (3,3);
        \fill[cyan] \secondcircle;
      \end{scope}
      \draw \firstcircle node {$A$};
      \draw \secondcircle node {$B$};
      \draw \thirdcircle node {$C$};
    \end{scope}
  \end{tikzpicture}}

$$P \cap Q$$
\centerline{\begin{tikzpicture}
    \begin{scope}[shift={(6cm,0cm)}]
      \begin{scope}
        \clip \firstcircle;
        \fill[cyan] \secondcircle;
      \end{scope}
      \begin{scope}
        \clip \firstcircle;
        \clip \secondcircle;
        \fill[white] \thirdcircle;
      \end{scope}
      \draw \firstcircle node {$A$};
      \draw \secondcircle node {$B$};
      \draw \thirdcircle node {$C$};
    \end{scope}
  \end{tikzpicture}}

$$P \cup Q$$

\centerline{\begin{tikzpicture}
    \begin{scope}
      \fill[cyan] \firstcircle;
      \fill[cyan] \secondcircle;
      \fill[cyan] \thirdcircle;
    \end{scope}
    \draw \firstcircle node[text=black,above] {$A$};
    \draw \secondcircle node [text=black,below left] {$B$};
    \draw \thirdcircle node [text=black,below right] {$C$};
  \end{tikzpicture}}

$$(P \cup Q) \setminus (P \cap Q) = (A \cup C) \triangle (B \setminus C)$$

\centerline{\begin{tikzpicture}
    \begin{scope}[shift={(6cm,0cm)}]
      \begin{scope}[even odd rule]
        \fill[cyan] \thirdcircle;
      \end{scope}
      \begin{scope}[even odd rule]
        \clip \firstcircle (-3,-3) rectangle (3,3);
        \fill[cyan] \secondcircle;
      \end{scope}
      \begin{scope}[even odd rule]
        \clip \secondcircle (-3,-3) rectangle (5,5);
        \fill[cyan] \firstcircle;
      \end{scope}
      \draw \firstcircle node {$A$};
      \draw \secondcircle node {$B$};
      \draw \thirdcircle node {$C$};
    \end{scope}
  \end{tikzpicture}}
\vspace{30pt}

(b) $$(A \triangle B) \cap C = (A \cap C) \triangle (B \cap C)$$

$$(A \triangle B) \cap C$$

\centerline{\begin{tikzpicture}
  \begin{scope}[shift={(6cm,0cm)}]
    \begin{scope}
      \clip \secondcircle;
      \fill[cyan] \thirdcircle;
    \end{scope}
    \begin{scope}
      \clip \firstcircle;
      \fill[cyan] \thirdcircle;
    \end{scope}
    \begin{scope}
      \clip \firstcircle;
      \clip \secondcircle;
      \fill[white] \thirdcircle;
    \end{scope}
    \draw \firstcircle node {$A$};
    \draw \secondcircle node {$B$};
    \draw \thirdcircle node {$C$};
  \end{scope}
\end{tikzpicture}}

$$P = B \cap C$$

\centerline{\begin{tikzpicture}
    \begin{scope}
      \clip \secondcircle;
      \fill[cyan] \thirdcircle;
    \end{scope}
    \begin{scope}
      \clip \firstcircle;
    \end{scope}
    \draw \firstcircle node[text=black,above] {$A$};
    \draw \secondcircle node [text=black,below left] {$B$};
    \draw \thirdcircle node [text=black,below right] {$C$};
  \end{tikzpicture}}

$$Q = A \cap C$$

\centerline{\begin{tikzpicture}
    \begin{scope}
      \clip \firstcircle;
      \fill[cyan] \thirdcircle;
    \end{scope}
    \begin{scope}
      \clip \firstcircle;
    \end{scope}
    \draw \firstcircle node[text=black,above] {$A$};
    \draw \secondcircle node [text=black,below left] {$B$};
    \draw \thirdcircle node [text=black,below right] {$C$};
  \end{tikzpicture}}

$$P \cap Q$$

\centerline{\begin{tikzpicture}
    \begin{scope}[shift={(6cm,0cm)}]
      \begin{scope}
        \clip \secondcircle;
        \fill[white] \thirdcircle;
      \end{scope}
      \begin{scope}
        \clip \firstcircle;
        \fill[white] \secondcircle;
      \end{scope}
      \begin{scope}
        \clip \thirdcircle;
        \fill[white] \firstcircle;
      \end{scope}
      \begin{scope}
        \clip \firstcircle;
        \clip \secondcircle;
        \fill[cyan] \thirdcircle;
      \end{scope}
      \draw \firstcircle node {$A$};
      \draw \secondcircle node {$B$};
      \draw \thirdcircle node {$C$};
    \end{scope}
  \end{tikzpicture}}
\vspace{30pt}

$$P \cup Q$$

\centerline{\begin{tikzpicture}
    \begin{scope}
      \clip \secondcircle;
      \fill[cyan] \thirdcircle;
    \end{scope}
    \begin{scope}
      \clip \firstcircle;
      \fill[cyan] \thirdcircle;
    \end{scope}
    \begin{scope}
      \clip \firstcircle;
    \end{scope}
    \draw \firstcircle node[text=black,above] {$A$};
    \draw \secondcircle node [text=black,below left] {$B$};
    \draw \thirdcircle node [text=black,below right] {$C$};
  \end{tikzpicture}}

$$(P \cup Q) \setminus (P \cap Q) = (A \cap C) \triangle (B \cap C)$$

\centerline{\begin{tikzpicture}
  \begin{scope}[shift={(6cm,0cm)}]
    \begin{scope}
      \clip \secondcircle;
      \fill[cyan] \thirdcircle;
    \end{scope}
    \begin{scope}
      \clip \firstcircle;
      \fill[cyan] \thirdcircle;
    \end{scope}
    \begin{scope}
      \clip \firstcircle;
      \clip \secondcircle;
      \fill[white] \thirdcircle;
    \end{scope}
    \draw \firstcircle node {$A$};
    \draw \secondcircle node {$B$};
    \draw \thirdcircle node {$C$};
  \end{scope}
\end{tikzpicture}}

\vspace{30pt}

(c) $$(A \triangle B) \setminus C = (A \setminus C) \triangle (B \setminus C)$$

$$(A \triangle B) \setminus C$$

\centerline{\begin{tikzpicture}
    \begin{scope}[shift={(6cm,0cm)}]
      \begin{scope}[even odd rule]
        \clip \firstcircle (-3,-3) rectangle (3,3);
        \fill[cyan] \secondcircle;
      \end{scope}
      \begin{scope}[even odd rule]
        \clip \secondcircle (-3,-3) rectangle (5,5);
        \fill[cyan] \firstcircle;
      \end{scope}
      \begin{scope}
        \fill[white] \thirdcircle;
      \end{scope}
      \draw \firstcircle node {$A$};
      \draw \secondcircle node {$B$};
      \draw \thirdcircle node {$C$};
    \end{scope}
  \end{tikzpicture}}

$$(A \setminus C) \triangle (B \setminus C)$$

$$P = A \setminus C$$

\centerline{\begin{tikzpicture}
    \begin{scope}[shift={(6cm,0cm)}]
      \begin{scope}[even odd rule]
        \clip \thirdcircle (-3,-3) rectangle (3,3);
        \fill[cyan] \firstcircle;
      \end{scope}
      \draw \firstcircle node {$A$};
      \draw \secondcircle node {$B$};
      \draw \thirdcircle node {$C$};
    \end{scope}
  \end{tikzpicture}}

$$Q = B \setminus C$$

\centerline{\begin{tikzpicture}
    \begin{scope}[shift={(6cm,0cm)}]
      \begin{scope}[even odd rule]
        \clip \thirdcircle (-3,-3) rectangle (3,3);
        \fill[cyan] \secondcircle;
      \end{scope}
      \draw \firstcircle node {$A$};
      \draw \secondcircle node {$B$};
      \draw \thirdcircle node {$C$};
    \end{scope}
  \end{tikzpicture}}

$$P \cap Q$$
\centerline{\begin{tikzpicture}
    \begin{scope}[shift={(6cm,0cm)}]
      \begin{scope}
        \clip \firstcircle;
        \fill[cyan] \secondcircle;
      \end{scope}
      \begin{scope}
        \clip \firstcircle;
        \clip \secondcircle;
        \fill[white] \thirdcircle;
      \end{scope}
      \draw \firstcircle node {$A$};
      \draw \secondcircle node {$B$};
      \draw \thirdcircle node {$C$};
    \end{scope}
  \end{tikzpicture}}

$$P \cup Q$$
\centerline{\begin{tikzpicture}
    \begin{scope}[shift={(6cm,0cm)}]
      \begin{scope}[even odd rule]
        \clip \thirdcircle (-3,-3) rectangle (3,3);
        \fill[cyan] \firstcircle;
      \end{scope}
      \begin{scope}[even odd rule]
        \clip \thirdcircle (-3,-3) rectangle (3,3);
        \fill[cyan] \secondcircle;
      \end{scope}
      \draw \firstcircle node {$A$};
      \draw \secondcircle node {$B$};
      \draw \thirdcircle node {$C$};
    \end{scope}
  \end{tikzpicture}}

$$(P \cup Q) \setminus (P \cap Q) = (A \setminus C) \triangle (B \setminus C)$$

\centerline{\begin{tikzpicture}
    \begin{scope}[shift={(6cm,0cm)}]
      \begin{scope}[even odd rule]
        \clip \firstcircle (-3,-3) rectangle (3,3);
        \fill[cyan] \secondcircle;
      \end{scope}
      \begin{scope}[even odd rule]
        \clip \secondcircle (-3,-3) rectangle (5,5);
        \fill[cyan] \firstcircle;
      \end{scope}
      \begin{scope}
        \fill[white] \thirdcircle;
      \end{scope}
      \draw \firstcircle node {$A$};
      \draw \secondcircle node {$B$};
      \draw \thirdcircle node {$C$};
    \end{scope}
  \end{tikzpicture}}
\vspace{30pt}

14. Use any method you wish to verify the following identities:

\hspace{12pt}(a) $(A \cup B) \triangle C = (A \triangle C) \triangle (B \setminus A)$.

\hspace{12pt}(b) $(A \cap B) \triangle C = (A \triangle C) \triangle (A \setminus B)$.

\hspace{12pt}(c) $(A \setminus B) \triangle C = (A \triangle C) \triangle (A \cap B)$.

\vspace{30pt}

(a) $$(A \cup B) \triangle C$$

$$A \cup B$$

\centerline{\begin{tikzpicture}
    \begin{scope}
      \fill[cyan] \firstcircle;
      \fill[cyan] \secondcircle;
    \end{scope}
    \draw \firstcircle node[text=black,above] {$A$};
    \draw \secondcircle node [text=black,below left] {$B$};
    \draw \thirdcircle node [text=black,below right] {$C$};
  \end{tikzpicture}}

$$(A \cup B) \triangle C$$

\centerline{\begin{tikzpicture}
    \begin{scope}[shift={(6cm,0cm)}]
      \begin{scope}[even odd rule]
        \clip \secondcircle (-3,-3) rectangle (3,3);
        \fill[cyan] \firstcircle;
      \end{scope}
      \begin{scope}[even odd rule]
        \clip \thirdcircle (-3,-3) rectangle (3,3);
        \fill[cyan] \secondcircle;
      \end{scope}
      \begin{scope}[even odd rule]
        \clip \firstcircle (-3,-3) rectangle (5,5);
        \fill[cyan] \thirdcircle;
      \end{scope}
      \begin{scope}
        \clip \secondcircle;
        \fill[white] \thirdcircle;
      \end{scope}
      \begin{scope}
        \clip \thirdcircle;
        \fill[white] \firstcircle;
      \end{scope}
      \draw \firstcircle node {$A$};
      \draw \secondcircle node {$B$};
      \draw \thirdcircle node {$C$};
    \end{scope}
  \end{tikzpicture}}
\vspace{30pt}

$$(A \triangle C) \triangle (B \setminus A)$$

$$P = A \triangle C$$

\centerline{\begin{tikzpicture}
    \begin{scope}[shift={(6cm,0cm)}]
      \begin{scope}[even odd rule]
        \clip \firstcircle (-3,-3) rectangle (5,5);
        \fill[cyan] \thirdcircle;
      \end{scope}
      \begin{scope}[even odd rule]
        \clip \thirdcircle (-3,-3) rectangle (5,5);
        \fill[cyan] \firstcircle;
      \end{scope}
      \draw \firstcircle node {$A$};
      \draw \secondcircle node {$B$};
      \draw \thirdcircle node {$C$};
    \end{scope}
  \end{tikzpicture}}

$$Q = B \setminus A$$

\centerline{\begin{tikzpicture}
    \begin{scope}[shift={(6cm,0cm)}]
      \begin{scope}[even odd rule]
        \clip \firstcircle (-3,-3) rectangle (3,3);
        \fill[cyan] \secondcircle;
      \end{scope}
      \draw \firstcircle node {$A$};
      \draw \secondcircle node {$B$};
      \draw \thirdcircle node {$C$};
    \end{scope}
  \end{tikzpicture}}

$$P \cap Q$$

\centerline{\begin{tikzpicture}
    \begin{scope}[shift={(6cm,0cm)}]
      \begin{scope}
        \clip \secondcircle;
        \fill[cyan] \thirdcircle;
      \end{scope}
      \begin{scope}
        \clip \secondcircle;
        \clip \thirdcircle;
        \fill[white] \firstcircle;
      \end{scope}
      \draw \firstcircle node {$A$};
      \draw \secondcircle node {$B$};
      \draw \thirdcircle node {$C$};
    \end{scope}
  \end{tikzpicture}}

$$P \cup Q$$

\centerline{\begin{tikzpicture}
    \begin{scope}[shift={(6cm,0cm)}]
      \begin{scope}[even odd rule]
        \clip \firstcircle (-3,-3) rectangle (3,3);
        \fill[cyan] \secondcircle;
      \end{scope}
      \begin{scope}[even odd rule]
        \clip \firstcircle (-3,-3) rectangle (5,5);
        \fill[cyan] \thirdcircle;
      \end{scope}
      \begin{scope}[even odd rule]
        \clip \thirdcircle (-3,-3) rectangle (5,5);
        \fill[cyan] \firstcircle;
      \end{scope}
      \draw \firstcircle node {$A$};
      \draw \secondcircle node {$B$};
      \draw \thirdcircle node {$C$};
    \end{scope}
  \end{tikzpicture}}

$$P \triangle Q$$

\centerline{\begin{tikzpicture}
    \begin{scope}[shift={(6cm,0cm)}]
      \begin{scope}[even odd rule]
        \clip \secondcircle (-3,-3) rectangle (3,3);
        \fill[cyan] \firstcircle;
      \end{scope}
      \begin{scope}[even odd rule]
        \clip \thirdcircle (-3,-3) rectangle (3,3);
        \fill[cyan] \secondcircle;
      \end{scope}
      \begin{scope}[even odd rule]
        \clip \firstcircle (-3,-3) rectangle (5,5);
        \fill[cyan] \thirdcircle;
      \end{scope}
      \begin{scope}
        \clip \secondcircle;
        \fill[white] \thirdcircle;
      \end{scope}
      \begin{scope}
        \clip \thirdcircle;
        \fill[white] \firstcircle;
      \end{scope}
      \draw \firstcircle node {$A$};
      \draw \secondcircle node {$B$};
      \draw \thirdcircle node {$C$};
    \end{scope}
  \end{tikzpicture}}
\vspace{30pt}

(c) $(A \setminus B) \triangle C = (A \triangle C) \triangle (A \cap B)$

$$A \setminus B$$

\centerline{\begin{tikzpicture}
    \begin{scope}[shift={(6cm,0cm)}]
      \begin{scope}[even odd rule]
        \clip \secondcircle (-3,-3) rectangle (3,3);
        \fill[cyan] \firstcircle;
      \end{scope}
      \draw \firstcircle node {$A$};
      \draw \secondcircle node {$B$};
      \draw \thirdcircle node {$C$};
    \end{scope}
  \end{tikzpicture}}

$$(A \setminus B) \triangle C$$

\centerline{\begin{tikzpicture}
    \begin{scope}[shift={(6cm,0cm)}]
      \begin{scope}[even odd rule]
        \clip \secondcircle (-3,-3) rectangle (3,3);
        \fill[cyan] \firstcircle;
      \end{scope}
      \begin{scope}[even odd rule]
        \clip \firstcircle (-3,-3) rectangle (5,5);
        \fill[cyan] \thirdcircle;
      \end{scope}
      \begin{scope}
        \clip \thirdcircle;
        \fill[white] \firstcircle;
      \end{scope}
      \begin{scope}
        \clip \firstcircle;
        \clip \secondcircle;
        \fill[cyan] \thirdcircle;
      \end{scope}
      \draw \firstcircle node {$A$};
      \draw \secondcircle node {$B$};
      \draw \thirdcircle node {$C$};
    \end{scope}
  \end{tikzpicture}}

\vspace{30pt}

$$(A \triangle C) \triangle (A \cap B)$$

$$P = A \triangle C$$

\centerline{\begin{tikzpicture}
    \begin{scope}[shift={(6cm,0cm)}]
      \begin{scope}[even odd rule]
        \clip \firstcircle (-3,-3) rectangle (5,5);
        \fill[cyan] \thirdcircle;
      \end{scope}
      \begin{scope}[even odd rule]
        \clip \thirdcircle (-3,-3) rectangle (5,5);
        \fill[cyan] \firstcircle;
      \end{scope}
      \draw \firstcircle node {$A$};
      \draw \secondcircle node {$B$};
      \draw \thirdcircle node {$C$};
    \end{scope}
  \end{tikzpicture}}

$$Q = A \cap B$$

\centerline{\begin{tikzpicture}
    \begin{scope}
      \clip \firstcircle;
      \fill[cyan] \secondcircle;
    \end{scope}
    \begin{scope}
      \clip \firstcircle;
    \end{scope}
    \draw \firstcircle node[text=black,above] {$A$};
    \draw \secondcircle node [text=black,below left] {$B$};
    \draw \thirdcircle node [text=black,below left] {$C$};
  \end{tikzpicture}}

$$P \cup Q$$

\centerline{\begin{tikzpicture}
    \begin{scope}[shift={(6cm,0cm)}]
      \begin{scope}
        \clip \firstcircle;
        \fill[cyan] \secondcircle;
      \end{scope}
      \begin{scope}
        \clip \firstcircle;
      \end{scope}
      \begin{scope}[even odd rule]
        \clip \firstcircle (-3,-3) rectangle (5,5);
        \fill[cyan] \thirdcircle;
      \end{scope}
      \begin{scope}[even odd rule]
        \clip \thirdcircle (-3,-3) rectangle (5,5);
        \fill[cyan] \firstcircle;
      \end{scope}
      \draw \firstcircle node {$A$};
      \draw \secondcircle node {$B$};
      \draw \thirdcircle node {$C$};
    \end{scope}
  \end{tikzpicture}}

$$P \cap Q$$

\centerline{\begin{tikzpicture}
    \begin{scope}[shift={(6cm,0cm)}]
      \begin{scope}
        \clip \firstcircle;
        \fill[cyan] \secondcircle;
      \end{scope}
      \begin{scope}
        \clip \firstcircle;
        \clip \secondcircle;
        \fill[white] \thirdcircle;
      \end{scope}
      \draw \firstcircle node {$A$};
      \draw \secondcircle node {$B$};
      \draw \thirdcircle node {$C$};
    \end{scope}
  \end{tikzpicture}}

\vspace{30pt}

$$P \triangle Q$$

\centerline{\begin{tikzpicture}
    \begin{scope}[shift={(6cm,0cm)}]
      \begin{scope}[even odd rule]
        \clip \secondcircle (-3,-3) rectangle (3,3);
        \fill[cyan] \firstcircle;
      \end{scope}
      \begin{scope}[even odd rule]
        \clip \firstcircle (-3,-3) rectangle (5,5);
        \fill[cyan] \thirdcircle;
      \end{scope}
      \begin{scope}
        \clip \thirdcircle;
        \fill[white] \firstcircle;
      \end{scope}
      \begin{scope}
        \clip \firstcircle;
        \clip \secondcircle;
        \fill[cyan] \thirdcircle;
      \end{scope}
      \draw \firstcircle node {$A$};
      \draw \secondcircle node {$B$};
      \draw \thirdcircle node {$C$};
    \end{scope}
  \end{tikzpicture}}
\vspace{30pt}

15. Fill in the blanks to make true identities:

\hspace{12pt}(a) $(A \triangle B) \cap C = (C \setminus A) \triangle \dots$

\hspace{12pt}(b) $C \setminus (A \triangle B) = (A \cap C) \triangle \dots$

\hspace{12pt}(c) $(B \setminus A) \triangle C = (A \triangle C) \triangle \dots$

\vspace{30pt}

(a) $(A \triangle B) \cap C = (C \setminus A) \triangle \dots$

$$(A \triangle B) \cap C$$

\centerline{\begin{tikzpicture}
  \begin{scope}[shift={(6cm,0cm)}]
    \begin{scope}
      \clip \secondcircle;
      \fill[cyan] \thirdcircle;
    \end{scope}
    \begin{scope}
      \clip \firstcircle;
      \fill[cyan] \thirdcircle;
    \end{scope}
    \begin{scope}
      \clip \firstcircle;
      \clip \secondcircle;
      \fill[white] \thirdcircle;
    \end{scope}
    \draw \firstcircle node {$A$};
    \draw \secondcircle node {$B$};
    \draw \thirdcircle node {$C$};
  \end{scope}
\end{tikzpicture}}


$$C \setminus A$$

\centerline{\begin{tikzpicture}
    \begin{scope}[shift={(6cm,0cm)}]
      \begin{scope}[even odd rule]
        \clip \firstcircle (-5,-5) rectangle (5,5);
        \fill[cyan] \thirdcircle;
      \end{scope}
      \draw \firstcircle node {$A$};
      \draw \secondcircle node {$B$};
      \draw \thirdcircle node {$C$};
    \end{scope}
  \end{tikzpicture}}

$$C \setminus B$$

\centerline{\begin{tikzpicture}
    \begin{scope}[shift={(6cm,0cm)}]
      \begin{scope}[even odd rule]
        \clip \secondcircle (-5,-5) rectangle (5,5);
        \fill[cyan] \thirdcircle;
      \end{scope}
      \draw \firstcircle node {$A$};
      \draw \secondcircle node {$B$};
      \draw \thirdcircle node {$C$};
    \end{scope}
  \end{tikzpicture}}

$$(A \triangle B) \cap C = (C \setminus A) \triangle (C \setminus B)$$
\vspace{10pt}

(b) $C \setminus (A \triangle B) = (A \cap C) \triangle \dots$



\centerline{\begin{tikzpicture}
    \begin{scope}[shift={(6cm,0cm)}]
      \begin{scope}[even odd rule]
        \clip \firstcircle (-3,-3) rectangle (5,5);
        \fill[cyan] \thirdcircle;
      \end{scope}
      \begin{scope}
        \clip \secondcircle;
        \fill[white] \thirdcircle;
      \end{scope}
      \begin{scope}
        \clip \thirdcircle;
        \fill[white] \firstcircle;
      \end{scope}
      \begin{scope}
        \clip \firstcircle;
        \clip \secondcircle;
        \fill[cyan] \thirdcircle;
      \end{scope}
      \draw \firstcircle node {$A$};
      \draw \secondcircle node {$B$};
      \draw \thirdcircle node {$C$};
    \end{scope}
  \end{tikzpicture}}

$$A \cap C$$

\centerline{\begin{tikzpicture}
    \begin{scope}
      \clip \firstcircle;
      \fill[cyan] \thirdcircle;
    \end{scope}
    \begin{scope}
      \clip \firstcircle;
    \end{scope}
    \draw \firstcircle node[text=black,above] {$A$};
    \draw \secondcircle node [text=black,below left] {$B$};
    \draw \thirdcircle node [text=black,below right] {$C$};
  \end{tikzpicture}}

$$B \cap C$$

\centerline{\begin{tikzpicture}
    \begin{scope}
      \clip \secondcircle;
      \fill[cyan] \thirdcircle;
    \end{scope}
    \begin{scope}
      \clip \firstcircle;
    \end{scope}
    \draw \firstcircle node[text=black,above] {$A$};
    \draw \secondcircle node [text=black,below left] {$B$};
    \draw \thirdcircle node [text=black,below right] {$C$};
  \end{tikzpicture}}

$$(A \cap C) \triangle (B \cap C)$$

\centerline{\begin{tikzpicture}
  \begin{scope}[shift={(6cm,0cm)}]
    \begin{scope}
      \clip \secondcircle;
      \fill[cyan] \thirdcircle;
    \end{scope}
    \begin{scope}
      \clip \firstcircle;
      \fill[cyan] \thirdcircle;
    \end{scope}
    \begin{scope}
      \clip \firstcircle;
      \clip \secondcircle;
      \fill[white] \thirdcircle;
    \end{scope}
    \draw \firstcircle node {$A$};
    \draw \secondcircle node {$B$};
    \draw \thirdcircle node {$C$};
  \end{scope}
\end{tikzpicture}}


$$C \setminus (A \triangle B) = (A \cap C) \triangle (B \cap C) \triangle C$$
\vspace{30pt}

(c) $$(B \setminus A) \triangle C = (A \triangle C) \triangle \dots$$

$$B \setminus A$$

\centerline{\begin{tikzpicture}
    \begin{scope}[shift={(6cm,0cm)}]
      \begin{scope}[even odd rule]
        \clip \firstcircle (-3,-3) rectangle (3,3);
        \fill[cyan] \secondcircle;
      \end{scope}
      \draw \firstcircle node {$A$};
      \draw \secondcircle node {$B$};
      \draw \thirdcircle node {$C$};
    \end{scope}
  \end{tikzpicture}}

$$(B \setminus A) \triangle C$$

\centerline{\begin{tikzpicture}
    \begin{scope}[shift={(6cm,0cm)}]
      \begin{scope}[even odd rule]
        \clip \thirdcircle (-3,-3) rectangle (3,3);
        \fill[cyan] \secondcircle;
      \end{scope}
      \begin{scope}[even odd rule]
        \clip \firstcircle (-3,-3) rectangle (5,5);
        \fill[cyan] \thirdcircle;
      \end{scope}
      \begin{scope}
        \clip \secondcircle;
        \fill[white] \thirdcircle;
      \end{scope}
      \begin{scope}
        \clip \firstcircle;
        \fill[white] \secondcircle;
      \end{scope}
      \begin{scope}
        \clip \firstcircle;
        \clip \secondcircle;
        \fill[cyan] \thirdcircle;
      \end{scope}
      \begin{scope}
        \clip \firstcircle;
        \fill[cyan] \thirdcircle;
      \end{scope}
      \draw \firstcircle node {$A$};
      \draw \secondcircle node {$B$};
      \draw \thirdcircle node {$C$};
    \end{scope}
  \end{tikzpicture}}
\vspace{30pt}

$$A \triangle C$$

\centerline{\begin{tikzpicture}
    \begin{scope}[shift={(6cm,0cm)}]
      \begin{scope}[even odd rule]
        \clip \firstcircle (-3,-3) rectangle (5,5);
        \fill[cyan] \thirdcircle;
      \end{scope}
      \begin{scope}[even odd rule]
        \clip \thirdcircle (-3,-3) rectangle (5,5);
        \fill[cyan] \firstcircle;
      \end{scope}
      \draw \firstcircle node {$A$};
      \draw \secondcircle node {$B$};
      \draw \thirdcircle node {$C$};
    \end{scope}
  \end{tikzpicture}}

$$A \cup B$$

\centerline{\begin{tikzpicture}
    \begin{scope}[shift={(6cm,0cm)}]
      \begin{scope}[even odd rule]
        \fill[cyan] \firstcircle;
        \fill[cyan] \secondcircle;
      \end{scope}
      \draw \firstcircle node {$A$};
      \draw \secondcircle node {$B$};
      \draw \thirdcircle node {$C$};
    \end{scope}
  \end{tikzpicture}}

$$(B \setminus A) \triangle C = (A \triangle C) \triangle (A \cup B)$$


\vspace{50pt}

\textbf{1.5. The Conditional and Biconditional Connectives}
\vspace{30pt}

$$P \to Q = \neg P \lor Q$$
Converse of $P \to Q$:
$$Q \to P = \neg Q \lor P$$
Contrapositive of $P \to Q$:
$$\neg Q \to \neg P = \neg \neg Q \lor \neg P = Q \lor \neg P =  P \to Q$$

\centerline{
  \begin{tabular}{c c c c}
  P & Q & $\neg P \lor Q$ & $\neg Q \lor P$ \\
  \hline
  F & F & T & T \\
  F & T & T & F \\
  T & F & F & T \\
  T & T & T & T \\
  \end{tabular}}
\vspace{20pt}

Biconditional statement: $$P \leftrightarrow Q = (P \to Q) \land (Q \to P)$$
\vspace{30pt}

Exercises:

1. Analyze the logical forms of the following statements:

\hspace{12pt}(a) If this gas either has an unpleasant smell or is not explosive, then it
isn’t hydrogen.

\hspace{12pt}(b) Having both a fever and a headache is a sufficient condition for George
to go to the doctor.

\hspace{12pt}(c) Both having a fever and having a headache are sufficient conditions
for George to go to the doctor.

\hspace{12pt}(d) If $x \neq 2$, then a necessary condition for x to be prime is that x be odd.
\vspace{30pt}

(a) P stand for "Gas has pleasant smell"

G stand for "Gas is explosive"

Q stand for "Gas is hydrogen"

$$(\neg P \lor \neg G) \to \neg Q$$
DeMorgan's law: $$\neg (P \land G) \to \neg Q$$
Contrapositive law: $$Q \to (P \land G)$$
\vspace{20pt}

(b) P stand for "To have a fever"

G stand for "To have a headache"

Q stand for "George goes to the doctor"

$$(P \land G) \to Q$$
\vspace{20pt}

(c) P stand for "To have a fever"

G stand for "To have a headache"

Q stand for "George goes to the doctor"

$$(P \lor G) \to Q$$
$$\neg (P \lor G) \lor Q$$
$$(\neg P \land \neg G) \lor Q$$
$$(\neg P \lor Q) \land (\neg G \lor Q)$$
$$(P \to Q) \land (G \to Q)$$

\vspace{20pt}

(d) P stand for "$x = 2$"

R stand for "x is odd"

Q stand for "x is prime"

$$\neg P \to (Q \to R)$$

or

$$(x \neq 2) \to (Q(x) \to R(x))$$

2. Analyze the logical forms of the following statements:

\hspace{12pt}(a) Mary will sell her house only if she can get a good price and find a
nice apartment.

\hspace{12pt}(b) Having both a good credit history and an adequate down payment is a
necessary condition for getting a mortgage.

\hspace{12pt}(c) John will kill himself, unless someone stops him. (Hint: First try to
rephrase this using the words if and then instead of unless.)

\hspace{12pt}(d) If x is divisible by either 4 or 6, then it isn't prime
\vspace{30pt}

(a) P is "Mary will sell her house"

Q is "Mary can get a good price"

R is "Mary can find a nice apartment"

$$P \to (Q \land R)$$
\vspace{20pt}

(b) Q is "to have a good credit history"

R is "to have an adequate down payment"

P is "to get a mortgage"

$$(Q \land R) \to P$$
\vspace{20pt}

(c) If someone doesn't stop John then he will kill himself.

P is "Someone stops John"

Q is "John will kill himself"

$$\neg P \to Q$$
\vspace{20pt}

(d) P(x) is "x is divisible by 4"

R(x) is "x is divisible by 6"

Q(x) is "x is prime"

$$(P(x) \lor R(x)) \to \neg Q(x)$$
\vspace{20pt}

3. Analyze the logical form of the following statement:

\hspace{12pt}(a) If it is raining, then it is windy and the sun is not shining.

Now analyze the following statements. Also, for each statement determine
whether the statement is equivalent to either statement (a) or its converse.

\hspace{12pt}(b) It is windy and not sunny only if it is raining.

\hspace{12pt}(c) Rain is a sufficient condition for wind with no sunshine.

\hspace{12pt}(d) Rain is a necessary condition for wind with no sunshine.

\hspace{12pt}(e) It’s not raining, if either the sun is shining or it’s not windy.

\hspace{12pt}(f) Wind is a necessary condition for it to be rainy, and so is a lack of
sunshine.

\hspace{12pt}(g) Either it is windy only if it is raining, or it is not sunny only if it is
raining.
\vspace{30pt}

(a) R: "It's raining"

W: "It's windy"

S: "Sun is shining"

$$R \to (W \land \neg S)$$
\vspace{20pt}

(b) $$(W \land \neg S) \to R$$

Equivalent to converse of (a).

(c) $$R \to (W \land \neg S)$$

Equivalent to (a).
\vspace{20pt}

(d) $$(W \land \neg S) \to R$$

Equivalent to converse of (a).
\vspace{20pt}

(e) $$(S \lor \neg W) \to \neg R$$

$$\neg (W \land \neg S) \to \neg R$$

$$R \to (W \land \neg S)$$

Equivalent to (a).
\vspace{20pt}

(f) $$R \to (W \land \neg S)$$

Equivalent to (a).
\vspace{20pt}

(g) $$(W \to R) \lor (\neg S \to R)$$

$$(\neg W \lor R) \lor (S \lor R)$$
$$(\neg W \lor S) \lor R$$
$$\neg (W \land \neg S) \lor R$$
$$(W \land \neg S) \to R$$

Equivalent to converse of (a).
\vspace{30pt}

4. Use truth tables to determine whether or not the following arguments are
valid:

\hspace{12pt}(a) Either sales or expenses will go up. If sales go up, then the boss will
be happy. If expenses go up, then the boss will be unhappy. Therefore,
sales and expenses will not both go up.

\hspace{12pt}(b) If the tax rate and the unemployment rate both go up, then there will
be a recession. If the GNP goes up, then there will not be a recession.
The GNP and taxes are both going up. Therefore, the unemployment
rate is not going up.

\hspace{12pt}(c) The warning light will come on if and only if the pressure is too high and
the relief valve is clogged. The relief valve is not clogged. Therefore,
the warning light will come on if and only if the pressure is too high.
\vspace{30pt}

(a)
$$S \lor E$$
$$S \to B$$
$$E \to \neg B$$
$$\therefore (\neg S \land E) \lor (S \land \neg E)$$

\centerline{
  \begin{tabular}{c c c c c c c}
  S & E & B & $S \lor E$ & $\neg S \lor B$ & $\neg E \lor \neg B$ & $(\neg S \land E) \lor (S \land \neg E)$ \\
  \hline
  F & F & F & F & T & T & F \\
  F & F & T & F & T & T & F \\
  F & T & F & T & T & T & T \\
  F & T & T & T & T & F & T \\
  T & F & F & T & F & T & T \\
  T & F & T & T & T & T & T \\
  T & T & F & T & F & T & F \\
  T & T & T & T & T & F & F \\
  \end{tabular}}
\vspace{20pt}

3 and 6 row: all arguments are true and conclusion is true as well, therefore arguments are valid.
\vspace{20pt}

(b) T: "Tax rate goes up"

U: "Unemployment rate goes up"

R: "Recession will be"

G: "GNP goes up"

$$(T \land U) \to R$$
$$G \to \neg R$$
$$G \land T$$
$$\therefore \neg U$$

\centerline{
  \begin{tabular}{c c c c l l l l}
  T & U & R & G & $\neg (T \land U) \lor R$ & $\neg (G \land R)$ &  $G \land T$ & $\neg U$ \\
  \hline
  F & F & F & F & \textbf{T} F F F & \textbf{T} F F F & F \textbf{F} F & \textbf{T} F \\
  F & F & F & T & \textbf{T} F F F & \textbf{T} T F F & T \textbf{F} F & \textbf{T} F \\
  F & F & T & F & \textbf{T} F F F & \textbf{T} F F T & F \textbf{F} F & \textbf{T} F \\
  F & F & T & T & \textbf{T} F F F & \textbf{F} T T T & T \textbf{F} F & \textbf{T} F \\
  F & T & F & F & \textbf{T} F F T & \textbf{T} F F F & F \textbf{F} F & \textbf{F} T \\
  F & T & F & T & \textbf{T} F F T & \textbf{T} T F F & T \textbf{F} F & \textbf{F} T \\
  F & T & T & F & \textbf{T} F F T & \textbf{T} F F T & F \textbf{F} F & \textbf{F} T \\
  F & T & T & T & \textbf{T} F F T & \textbf{F} T T T & T \textbf{F} F & \textbf{F} T \\
  T & F & F & F & \textbf{T} T F F & \textbf{T} F F F & F \textbf{F} T & \textbf{T} F \\
  T & F & F & T & \textbf{T} T F F & \textbf{T} T F F & T \textbf{T} T & \textbf{T} F \\
  T & F & T & F & \textbf{T} T F F & \textbf{T} F F T & F \textbf{F} T & \textbf{T} F \\
  T & F & T & T & \textbf{T} T F F & \textbf{F} T T T & T \textbf{T} T & \textbf{T} F \\
  T & T & F & F & \textbf{F} T T T & \textbf{T} F F F & F \textbf{F} T & \textbf{F} T \\
  T & T & F & T & \textbf{F} T T T & \textbf{T} T F F & T \textbf{T} T & \textbf{F} T \\
  T & T & T & F & \textbf{F} T T T & \textbf{T} F F T & F \textbf{F} T & \textbf{F} T \\
  T & T & T & T & \textbf{F} T T T & \textbf{F} T T T & T \textbf{T} T & \textbf{F} T \\
  \end{tabular}}
\vspace{20pt}

All arguments are true only in 10th row and conclusion is true as well, therefore arguments are valid.
\vspace{20pt}

(c) L: "Warning light will come on"

P: "Pressure is too high"

R: "Relief valve is clogged"

$$L \leftrightarrow (P \land R) = (L \to (P \land R)) \land ((P \land R) \to L) = (\neg L \lor (P \land R)) \land (\neg (P \land R) \lor L)$$
$$\neg R$$
$$\therefore L \leftrightarrow P = (\neg L \lor P) \land (\neg P \lor L)$$

\centerline{
  \begin{tabular}{c c c l l l}
  P & R & L & $(\neg L \lor (P \land R)) \land (\neg (P \land R) \lor L)$ & $\neg R$ & $(\neg L \lor P) \land (\neg P \lor L)$ \\
  \hline
  F & F & F & T F  T  F F F \textbf{T} T F F F  T  F & \textbf{T} F & T F T F \textbf{T} T F T F \\
  F & F & T & F T  F  F F F \textbf{F} T F F F  T  T & \textbf{T} F & F T F F \textbf{F} T F T T \\
  F & T & F & T F  T  F F T \textbf{T} T F F T  T  F & \textbf{F} T & T F T F \textbf{T} T F T F \\
  F & T & T & F T  F  F F T \textbf{F} T F F T  T  T & \textbf{F} T & F T F F \textbf{F} T F T T \\
  T & F & F & T F  T  T F F \textbf{T} T T F F  T  F & \textbf{T} F & T F T T \textbf{F} F T F F \\
  T & F & T & F T  F  T F F \textbf{F} T T F F  T  T & \textbf{T} F & F T T T \textbf{T} F T T T \\
  T & T & F & T F  T  T T T \textbf{F} F T T T  F  F & \textbf{F} T & T F T T \textbf{F} F T F F \\
  T & T & T & F T  T  T T T \textbf{T} F T T T  T  T & \textbf{F} T & F T T T \textbf{T} F T T T \\
  \end{tabular}}
\vspace{20pt}

All arguments are true only in 5th row but conclusion is false, therefore arguments are invalid.
\vspace{30pt}

5. (a) Show that $P \leftrightarrow Q$ is equivalent to $(P \land Q) \lor (\neg P \land \neg Q)$.

\hspace{12pt}(b) Show that $(P \to Q) \lor (P \to R)$ is equivalent to $P \to (Q \lor R)$.
\vspace{30pt}

(a) Definition of biconditional statement: $P \leftrightarrow Q = (P \to Q) \land (Q \to P)$

Contrapositive law: $(P \to Q) \land (\neg P \to \neg Q)$

Conditional law: $(\neg P \lor Q) \land (\neg \neg P \lor \neg Q)$

Distributive law: $((\neg P \lor Q ) \land \neg \neg P) \lor ((\neg P \lor Q) \land \neg Q)$

Distributive law: $((\neg P \land P) \lor (P \land Q)) \lor ((\neg P \land \neg Q) \lor (Q \land \neg Q))$

$((contradiction) \lor (P \land Q)) \lor ((\neg P \land \neg Q) \lor (contradiction))$

Contradiction law: $(P \land Q) \lor (\neg P \land \neg Q)$
\vspace{30pt}

(b) $(P \to Q) \lor (P \to R)$

Conditional law: $(\neg P \lor Q) \lor (\neg P \lor R)$

Associative law: $(\neg P \lor \neg P) \lor (Q \lor R)$

Idemponent law: $\neg P \lor (Q \lor R)$

Conditional law: $P \to (Q \lor R)$
\vspace{30pt}

6. (a) Show that $(P \to R) \land (Q \to R)$ is equivalent to $(P \lor Q) \to R$.

\hspace{12pt}(b) Formulate and verify a similar equivalence involving $(P \to R) \lor (Q \to R)$.
\vspace{20pt}

(a) $(P \to R) \land (Q \to R)$

Conditional law $(\neg P \lor R) \land (\neg Q \lor R)$

Distributive law: $(\neg P \land \neg Q) \lor R$

DeMorgan's law: $\neg (P \lor Q) \lor R$

Conditional law: $(P \lor Q) \to R$
\vspace{10pt}

(b) $(P \to R) \lor (Q \to R)$

Conditional law: $(\neg P \lor R) \lor (\neg Q \lor R)$

Associative law: $(\neg P \lor \neg Q) \lor (R \lor R)$

Idemponent law: $(\neg P \lor \neg Q) \lor R$

DeMorgan's law: $\neg (P \land Q) \lor R$

Conditional law: $(P \land Q) \to R$

\vspace{30pt}

7. (a) Show that $(P \to Q) \land (Q \to R)$ is equivalent to $(P \to R) \land [(P \leftrightarrow Q) \lor (R \leftrightarrow Q)]$.

\hspace{12pt}(b) Show that $(P \to Q) \lor (Q \to R)$ is a tautology.
\vspace{20pt}

(a)

$$(P \to Q) \land (Q \to R)$$

\centerline{
  \begin{tabular}{c c c l l c}
  P & Q & R & $(P \to Q)$ & $(Q \to R)$ & $(P \to Q) \land (Q \to R)$ \\
  \hline
  F & F & F & T T F & T T F & T \\
  F & F & T & T T F & T T T & T \\
  F & T & F & T T T & F F F & F \\
  F & T & T & T T T & F T T & T \\
  T & F & F & F F F & T T F & F \\
  T & F & T & F F F & T T T & F \\
  T & T & F & F T T & F F F & F \\
  T & T & T & F T T & F T T & T \\
  \end{tabular}}
\vspace{20pt}

$$N = P \to R$$

\centerline{
  \begin{tabular}{c c c l}
  P & Q & R & $P \to R$ \\
  \hline
  F & F & F & T \textbf{T} F \\
  F & F & T & T \textbf{T} T \\
  F & T & F & T \textbf{T} F \\
  F & T & T & T \textbf{T} T \\
  T & F & F & F \textbf{F} F \\
  T & F & T & F \textbf{T} T \\
  T & T & F & F \textbf{F} F \\
  T & T & T & F \textbf{T} T \\
  \end{tabular}}
\vspace{20pt}

$$M = P \leftrightarrow Q = (P \to Q) \land (Q \to P)$$

\centerline{
  \begin{tabular}{c c c l l c}
  P & Q & R & $(P \to Q)$ & $(Q \to P)$ & $(P \to Q) \land (Q \to P)$ \\
  \hline
  F & F & F & T T F & T T F & T \\
  F & F & T & T T F & T T F & T \\
  F & T & F & T T T & F F F & F \\
  F & T & T & T T T & F F F & F \\
  T & F & F & F F F & T T T & F \\
  T & F & T & F F F & T T T & F \\
  T & T & F & F T T & F T T & T \\
  T & T & T & F T T & F T T & T \\
  \end{tabular}}
\vspace{20pt}

$$Y = R \leftrightarrow Q = (R \to Q) \land (Q \to R)$$

\centerline{
  \begin{tabular}{c c c l l c}
  P & Q & R & $(R \to Q)$ & $(Q \to R)$ & $(R \to Q) \land (Q \to R)$ \\
  \hline
  F & F & F & T T F & T T F & T \\
  F & F & T & F F F & T T T & F \\
  F & T & F & T T T & F F F & F \\
  F & T & T & F T T & F T T & T \\
  T & F & F & T T F & T T F & T \\
  T & F & T & F F F & T T T & F \\
  T & T & F & T T T & F F F & F \\
  T & T & T & F T T & F T T & T \\
  \end{tabular}}
\vspace{20pt}

$$(P \to R) \land [(P \leftrightarrow Q) \lor (R \leftrightarrow Q)] = N \land (M \lor Y)$$

\centerline{
  \begin{tabular}{c c c c}
  N & M & Y & $(P \to R) \land [(P \leftrightarrow Q) \lor (R \leftrightarrow Q)]$ \\
  \hline
  T & T & T & T \\
  T & T & F & T \\
  T & F & F & F \\
  T & F & T & T \\
  F & F & T & F \\
  T & F & F & F \\
  F & T & F & F \\
  T & T & T & T \\
  \end{tabular}}
\vspace{20pt}

Truth table for $(P \to Q) \land (Q \to R)$ is equivalent to $(P \to R) \land [(P \leftrightarrow Q) \lor (R \leftrightarrow Q)]$ truth table.

Therefore $(P \to Q) \land (Q \to R)$ is equivalent to $(P \to R) \land [(P \leftrightarrow Q) \lor (R \leftrightarrow Q)]$.

\vspace{20pt}

(b) $(P \to Q) \lor (Q \to R)$

Conditional law: $(\neg P \lor Q) \lor (\neg Q \lor R)$

Associative law: $(\neg P \lor R) \lor (Q \lor \neg Q)$

$(\neg P \lor R) \lor (tautology)$

Tautology law: $(tautology)$

\vspace{30pt}

8. Find a formula involving only the connectives $\neg$ and $\to$ that is equivalent to $P \land Q$.
\vspace{20pt}

$$P \land Q$$
$$\neg \neg P \land \neg \neg Q$$
$$\neg (\neg P \lor \neg Q)$$
$$\neg (P \to \neg Q)$$

\vspace{30pt}

9. Find a formula involving only the connectives $\neg$ and $\to$ that is equivalent to $P \leftrightarrow Q$.
\vspace{20pt}

$$P \leftrightarrow Q$$
$$(P \to Q) \land (Q \to P)$$

Using results from ex. 8:

$$\neg ((P \to Q) \to \neg (Q \to P))$$


\vspace{30pt}

10. Which of the following formulas are equivalent?

\hspace{12pt}(a) $P \to (Q \to R)$.

\hspace{12pt}(b) $Q \to (P \to R)$.

\hspace{12pt}(c) $(P \to Q) \land (P \to R)$.

\hspace{12pt}(d) $(P \land Q) \to R$.

\hspace{12pt}(e) $P \to (Q \land R)$.
\vspace{20pt}

(a) $$P \to (Q \to R)$$
$$\neg P \lor (Q \to R)$$
$$\neg P \lor \neg Q \lor R$$
\vspace{10pt}

(b) $$Q \to (P \to R)$$
$$\neg Q \lor (P \to R)$$
$$\neg Q \lor \neg P \lor R$$
$$\neg P \lor \neg Q \lor R$$

(c) $$(P \to Q) \land (P \to R)$$
$$(\neg P \lor Q) \land (\neg P \lor R)$$
$$\neg P \lor (Q \land R)$$

(d) $$(P \land Q) \to R$$
$$\neg (P \land Q) \lor R$$
$$\neg P \lor \neg Q \lor R$$

(e) $$P \to (Q \land R)$$
$$\neg P \lor (Q \land R)$$
\vspace{20pt}

(c) is equivalent to (e)

(a) is equivalent to (b) and (d)































































































\end{document}
