\documentclass{article}
\usepackage{mathtools}
\usepackage{xcolor}
\usepackage{listings}
\usepackage{amssymb}
\usepackage{tikz}
\usepackage{soul}
\usetikzlibrary{shapes,backgrounds}
\renewcommand{\baselinestretch}{1.5}
\lstset{
  frame=none,
  xleftmargin=2pt,
  stepnumber=1,
  numbers=left,
  numbersep=5pt,
  numberstyle=\ttfamily\tiny\color[gray]{0.3},
  belowcaptionskip=\bigskipamount,
  captionpos=b,
  escapeinside={*'}{'*},
  language=haskell,
  tabsize=2,
  emphstyle={\bf},
  commentstyle=\it,
  stringstyle=\mdseries\rmfamily,
  showspaces=false,
  keywordstyle=\bfseries\rmfamily,
  columns=flexible,
  basicstyle=\small\sffamily,
  showstringspaces=false,
  morecomment=[l]\%,
}
\begin{document}
\topskip0pt
\vspace*{\fill}
\centerline{\sc \large Solutions of the exercises for "How to prove it" book }
\centerline{by drets}
\centerline{\textit{(may contain various errors)}}
\vspace*{\fill}
%
\pagebreak
\centerline{\sc \large 3. Proofs}
\vspace{50pt}

\textbf{3.1. Proof Strategies}

\vspace{40pt}

To prove a goal of the form $P \to Q$

Assume $P$ is true and then prove $Q$.
\vspace{20pt}

To prove a goal of the form $P \to Q$

Assume $Q$ is false and prove that $P$ is false

\vspace{30pt}

Exercises:

\vspace{30pt}

∗1. Consider the following theorem. (This theorem was proven in the introduction.)

\textbf{Theorem}. \textit{Suppose n is an integer larger than 1 and n is not prime. Then
$2^n - 1$ is not prime.}

\hspace{12pt}(a) Identify the hypotheses and conclusion of the theorem. Are the hypotheses
true when $n = 6$? What does the theorem tell you in this instance? Is it right?

\hspace{12pt}(b) What can you conclude from the theorem in the case $n = 15$? Check
directly that this conclusion is correct.

\hspace{12pt}(c) What can you conclude from the theorem in the case $n = 11$?

\vspace{20pt}

(a) Hypotheses: $n \in \mathbb{Q}$ and $n > 1$, and n is not prime.

Conclusion: $2^n - 1$ is not prime

When $n = 6$ hypotheses are true.

$2^6-1 = 63$ is not prime, theorem is right.

\vspace{20pt}

(b) $2^{15}-1 = 32767 = 7*4681$

$5*3 = 15$

\vspace{20pt}

(c) The theorem tells us nothing since 11 is prime, so hypotheses are not satisfied.

\vspace{30pt}

2. Consider the following theorem. (The theorem is correct, but we will not
ask you to prove it here.)

\textbf{Theorem}. \textit{Suppose that $b^2 > 4ac$. Then the quadratic equation $ax^2 +
bx + c = 0$ has exactly two real solutions.}

\hspace{12pt}(a) Identify the hypotheses and conclusion of the theorem.

\hspace{12pt}(b) To give an instance of the theorem, you must specify values for a, b,
and c, but not x. Why?

\hspace{12pt}(c) What can you conclude from the theorem in the case a = 2, b = -5,
c = 3? Check directly that this conclusion is correct.

\hspace{12pt}(d) What can you conclude from the theorem in the case a = 2, b = 4,
c = 3?

\vspace{20pt}

(a) Hypotheses: $b^2 > 4ac$

Conclusion: $ax^2 + bx + c = 0$ has exactly two real solutions.

\vspace{20pt}

(b) Because the values of $x$ are the solutions, we need to calculate them.

\vspace{20pt}

(c) $2x^2 - 5x + 3 = 0$

$D = b^2 - 4ac = 25 - 24 = 1$

$x_1 = 1.5$ $x_2 = 1$

\vspace{20pt}

(d) $2x^2 + 4x + 3 = 0$

The theorem tells us nothing, since hypothese is not satisfied $16 \ngtr 24$


\vspace{30pt}

3. Consider the following incorrect theorem:

\textbf{Incorrect Theorem}. \textit{Suppose n is a natural number larger than 2, and
n is not a prime number. Then $2n + 13$ is not a prime number.}

What are the hypotheses and conclusion of this theorem? Show that
the theorem is incorrect by finding a counterexample.

\vspace{30pt}

Hypotheses: n is a natural number larger than 2, and n is not a prime number.

Conclusion: $2n + 13$ is not a prime number.

Counterexample: n = 9 is a natural number larger than 2, and n is not a prime number since $3*3=9$

$2 * 9 + 13 = 18 + 13 = 31$ is prime number.

\vspace{30pt}

∗4. Complete the following alternative proof of the theorem in Example 3.1.2.

\textit{Proof}. Suppose $0 < a < b$. Then $b - a > 0$.
Multiplying both sides by the positive number $b + a$, we get $(b+a)(b-a)>(b+a)*0$, or in other words $b^2 - a^2 > 0$.
Since $b^2 - a^2 > 0$, it follows that $a^2 < b^2$. Therefore if $0 < a < b$ then
$a^2 < b^2$.

\vspace{20pt}

5. Suppose a and b are real numbers. Prove that if $a < b < 0$ then $a^2 > b^2$.

\vspace{20pt}

\textit{Proof}. Suppose $a < b < 0$. Then $b - a > 0$.
Multiplying both sides by the negative number $b + a$, we get $(b+a)(b-a)<(b+a)*0$, or in other words $b^2 - a^2 < 0$.
Since $b^2 - a^2 < 0$, it follows that $a^2 > b^2$. Therefore if $a < b < 0$ then $a^2 > b^2$

\vspace{30pt}

6. Suppose a and b are real numbers. Prove that if $0 < a < b$ then $\frac{1}{b} < \frac{1}{a}$.

\vspace{20pt}

\textit{Proof}. Suppose $0 < a < b$. Then $b - a > 0$.
Dividing both sides by the positive number a, we get $\frac{b}{a} - 1 > 0$.
Then dividing both sides by the positive number b, we get $\frac{b}{a*b} - \frac{1}{b} > 0$, or in other words
$\frac{1}{b} < \frac{1}{a}$. Therefore if $0 < a < b$ then $\frac{1}{b} < \frac{1}{a}$

\vspace{30pt}

7. Suppose that a is a real number. Prove that if $a^3 > a$ then $a^5 > a$. (Hint:
One approach is to start by completing the following equation: $a^5 - a = (a^3 - a) * ?$ .)

\vspace{20pt}

\textit{Proof}. Suppose $a^3 > a$. Then $a^3 - a > 0$.
Multiplying both sides by the positive number $a^2$, we get $a^5 - a^3 > 0$, or in other words $a^5 > a^3$.
Since $a^5 > a^3$ and $a^3 > a$, it follows that $a^5 > a$. Therefore if $a^3 > a$ then $a^5 > a$.

\vspace{30pt}

8. Suppose $A \setminus B \subseteq C \cap D$ and $x \in A$. Prove that if $x \notin D$ then $x \in B$.

\vspace{20pt}

$\forall x (x \in (A \setminus B) \to x \in (C \cap D))$

$\forall x (\neg (x \in A \land x \notin B) \lor (x \in C \land x \in D))$

$\forall x ((x \notin A \lor x \in B) \lor (x \in C \land x \in D))$
\vspace{20pt}

\textit{Proof}. Suppose $A \setminus B \subseteq C \cap D$ and $x \in A$ and $x \notin D$.
Then $\forall x ((x \notin A \lor x \in B) \lor (x \in C \land x \in D))$, or in other words
$(false \lor x \in B) \lor (x \in C \land false)$ should be equal to true.
Therefore if $A \setminus B \subseteq C \cap D$ and $x \in A$ and $x \notin D$ then $x \in B$.


\vspace{30pt}

∗9. Suppose a and b are real numbers. Prove that if $a < b$ then $\frac{a+b}{2} < b$.

\vspace{20pt}

\textit{Proof}. Suppose $a < b$. Adding the number $b$ to both sides, we get $a + b < b + b$, or in other words
$a + b < 2b$. Since $a + b < 2b$, it follows that $\frac{a+b}{2} < b$. Therefore if $a < b$ then $\frac{a+b}{2} < b$

\vspace{30pt}

10. Suppose x is a real number and $x \neq 0$. Prove that if $\frac{\sqrt[3]{x} + 5}{x^2 + 6} = \frac{1}{x}$
then $x \neq 8$.

\vspace{20pt}

\textit{Proof}. Suppose $x \neq 0$ and $\frac{\sqrt[3]{x} + 5}{x^2 + 6} = \frac{1}{x}$.
Then $\frac{x^2 + 6}{\sqrt[3]{x} + 5} = {x}$. Let x to be equal to 8.
$\frac{64 + 6}{2 + 5} \neq 8$, or in other words $10 \neq 8$.
Therefore if $x \neq 0$ and $\frac{\sqrt[3]{x} + 5}{x^2 + 6} = \frac{1}{x}$ then $x \neq 8$.

\vspace{30pt}

∗11. Suppose a, b, c, and d are real numbers, $0 < a < b$, and $d > 0$.
Prove that if $ac \geq bd$ then $c > d$.

\vspace{20pt}

\textbf{Theorem}. \textit{Suppose a, b, c, and d are real numbers, $0 < a < b$, and $d > 0$. If $ac \geq bd$ then $c > d$}

\textit{Proof}. We will prove the contrapositive. Suppose $c \leq d$. Multiplying both sides of this inequality by the positive number a,
we get $ac \leq ad$. Also, multiplying both sides of the given inequality $a < b$ by the positive number d gives us $ad < bd$.
Combining $ac \leq ad$ and $ad < bd$, we can conclude that $ac < bd$. Thus, if $ac \geq bd$ then $c > d$.

\vspace{30pt}

12. Suppose x and y are real numbers, and $3x + 2y \leq 5$. Prove that if $x > 1$ then $y < 1$.

\vspace{20pt}

\textbf{Theorem}. \textit{Suppose x and y are real numbers, and $3x + 2y \leq 5$. If $x > 1$ then $y < 1$}.

\textit{Proof}. We will prove the contrapositive. Suppose $y \geq 1$. Then $2y \geq 2$.
By substacting 5 from the both sides and multiplying by -1, we get $5 - 2y \leq 3$.
By moving 2y to another side of inequality and dividing by 3, we get $x \leq \frac{5 - 2y}{3}$.
Using $5 - 2y \leq 3$ and $x \leq \frac{5 - 2y}{3}$ we can conclude that $x \leq 1$. Thus, if $x > 1$ then $y < 1$.

\vspace{30pt}

13. Suppose that x and y are real numbers. Prove that if $x^2 + y = -3$ and $2x - y = 2$ then $x = -1$.

\vspace{20pt}

\textbf{Theorem}. \textit{Suppose x and y are real numbers. If $x^2 + y = -3$ and $2x - y = 2$ then $x = -1$.}

\textit{Proof}. Suppose $x^2 + y = -3$ and $2x - y = 2$. Then $y = 2x - 2$.
Combining the given inequality $x^2 + y = -3$ and $y = 2x - 2$, we get $x^2 + 2x + 1 = 0$.
Solving the inequality using Vieta's formula, we get $x = -1$. Therefore, if $x^2 + y = -3$ and $2x - y = 2$ then $x = -1$.

\vspace{30pt}

∗14. Prove the first theorem in Example 3.1.1. (Hint: You might find it useful to apply the theorem from Example 3.1.2.)

\vspace{20pt}

\textbf{Theorem}. \textit{Suppose $x > 3$ and $y < 2$. Then $x^2 - 2y > 5$}

\textit{Proof}. We will prove the contrapositive. Suppose $x^2 - 2y \leq 5$.
Then $y \geq \frac{x^2 - 5}{2}$. We can transform the given inequality $x > 3$ to $\frac{x^2 - 5}{2} > \frac{3x - 5}{2}$.
Again, using the given inequality $x > 3$ we can get $\frac{3x - 5}{2} > \frac{9 - 5}{2}$.
$2 < \frac{3x - 5}{2} < \frac{x^2 - 5}{2} \leq y$, it follows $y > 2$.
Thus, if $x > 3$ and $y < 2$ then $x^2 - 2y > 5$.

\vspace{30pt}

15. Consider the following theorem.

\textbf{Theorem}. \textit{Suppose x is a real number and $x \neq 4$. If $\frac{2x-5}{x-4} = 3$ then $x = 7$}.

\hspace{12pt}(a) What’s wrong with the following proof of the theorem?

\textit{Proof}. Suppose x = 7. Then $\frac{2x-5}{x-4} = \frac{2(7)-5}{7-4}=\frac{9}{3}=3$.

Therefore if $\frac{2x-5}{x-4} = 3$ then $x = 7$


\hspace{12pt}(b) Give a correct proof of the theorem.

\vspace{20pt}

(a) The proof strategy is wrong (it doesn't correspond to either $P \to Q$ or $\neg Q \to \neg P$). It may be the case that there is more than one value of x which satisfy the hypothese.

\vspace{20pt}

(b) Suppose $\frac{2x-5}{x-4} = 3$. Then $2x - 5 = 3x - 12$, or in other word $x = 7$.
Therefore, if $\frac{2x-5}{x-4} = 3$ then $x = 7$

\vspace{30pt}

16. Consider the following incorrect theorem:

\textbf{Incorrect Theorem}. \textit{Suppose that x and y are real numbers and $x \neq 3$.
If $x^2y = 9y$ then $y = 0$.}

\hspace{12pt}(a) What’s wrong with the following proof of the theorem?

\textit{Proof}. Suppose that $x^2y = 9y$. Then $(x^2 - 9)y = 0$. Since $x \neq 3$,
$x^2 \neq 9$, so $x^2 - 9 \neq 0$. Therefore we can divide both sides of the
equation $(x^2 - 9)y = 0$ by $x^2 - 9$, which leads to the conclusion
that $y = 0$. Thus, if $x^2y = 9y$ then $y = 0$.

\hspace{12pt}(b) Show that the theorem is incorrect by finding a counterexample.

\vspace{20pt}

(a) It's not allowed to divide by $x^2 - 9$ since $x$ may be equal to $-3$.

\vspace{20pt}

(b) $x = -3$

$9y = 9y$

$0y = 0$

$\therefore$ y has undefined value

\vspace{50pt}

\textbf{3.2. Proofs Involving Negations and Conditionals}

Exercises:

∗1. This problem could be solved by using truth tables, but don't do it that
way. Instead, use the methods for writing proofs discussed so far in this
chapter. (See Example 3.2.4.)

\hspace{12pt}(a) Suppose $P \to Q$ and $Q \to R$ are both true. Prove that $P \to R$ is
true.

\hspace{12pt}(b) Suppose $\neg R \to (P \to \neg Q)$ is true. Prove that $P \to (Q \to R)$ is
true.
\vspace{20pt}

(a) Scratch work:

\centerline{
  \begin{tabular}{l l}
  \textit{Givens} & \textit{Goal} \\
  $P \to Q$ & $P \to R$ \\
  $Q \to R$ &           \\
  \end{tabular}}
\vspace{20pt}

\centerline{
  \begin{tabular}{l l}
  \textit{Givens} & \textit{Goal} \\
  $P \to Q$ & $R$ \\
  $Q \to R$ &     \\
  $P$       &     \\
  \end{tabular}}
\vspace{20pt}

Suppose $P$

$\quad$ [Proof of R goes here.]

Therefore $P \to R$

Since we know $P$ and $P \to Q$ by modus ponens we can infer $Q$.

\centerline{
  \begin{tabular}{l l}
  \textit{Givens} & \textit{Goal} \\
  $P \to Q$ & $R$ \\
  $Q \to R$ &     \\
  $P$       &     \\
  $Q$       &     \\
  \end{tabular}}
\vspace{20pt}

Since we know $Q$ and $Q \to R$ by modus ponens we can infer $R$.
This is our goal, so the proof is done.
\vspace{20pt}

\textbf{Theorem}. \textit{Suppose $P \to Q$ and $Q \to R$ are both true then $P \to R$ is
true}

\textit{Proof}. Suppose P. Since $P$ and $P \to Q$, it follows that $Q$. But then,
since $Q$ and $Q \to R$, it follow that $R$. Therefore if $P \to Q$ and $Q \to R$ then $P \to R$.

(b) Scratch work:

\centerline{
  \begin{tabular}{l l}
  \textit{Givens} & \textit{Goal} \\
  $\neg R \to (P \to \neg Q)$ & $P \to (Q \to R)$ \\
  \end{tabular}}
\vspace{20pt}

\centerline{
  \begin{tabular}{l l}
  \textit{Givens} & \textit{Goal} \\
    $\neg R \to (P \to \neg Q)$ & $Q \to R$ \\
    $P$ & \\
  \end{tabular}}
\vspace{20pt}

\centerline{
  \begin{tabular}{l l}
  \textit{Givens} & \textit{Goal} \\
    $\neg R \to (P \to \neg Q)$ & $\neg Q$ \\
    $P$ & \\
    $\neg R$ & \\
  \end{tabular}}
\vspace{20pt}

\textbf{Theorem} \textit{Suppose $\neg R \to (P \to \neg Q)$ is true then $P \to (Q \to R)$ is
  true.}

\textit{Prove}. Suppose $P$. To prove that $Q \to R$, we will prove the contrapositive, so suppose $\neg R$. Since $\neg R$ and $\neg R \to \neg Q$, it follows that $P \to \neg Q$. But then, since $P$ and $P \to \neg Q$, it follows that $\neg Q$. Thus, $Q \to R$, so $P \to (Q \to R)$.

\vspace{30pt}

2. This problem could be solved by using truth tables, but don't do it that
way. Instead, use the methods for writing proofs discussed so far in this
chapter. (See Example 3.2.4.)

\hspace{12pt}(a) Suppose $P \to Q$ and $R \to \neg Q$ are both true. Prove that $P \to \neg R$
is true.

\hspace{12pt}(b) Suppose that $P$ is true. Prove that $Q \to \neg (Q \to \neg P)$ is true.
\vspace{20pt}

(a) Scratch work

\centerline{
  \begin{tabular}{l l}
  \textit{Givens} & \textit{Goal} \\
  $P \to Q$      & $P \to \neg R$ \\
  $R \to \neg Q$ & \\
  \end{tabular}}
\vspace{20pt}

\centerline{
  \begin{tabular}{l l}
  \textit{Givens} & \textit{Goal} \\
  $P \to Q$      & $\neg R$ \\
  $R \to \neg Q$ & \\
  $P$ & \\
  \end{tabular}}
\vspace{20pt}

\centerline{
  \begin{tabular}{l l}
  \textit{Givens} & \textit{Goal} \\
  $P \to Q$      & $\neg R$ \\
  $R \to \neg Q$ & \\
  $P$ & \\
  $Q$ & \\
  \end{tabular}}
\vspace{20pt}

$$R \to \neg Q = Q \to \neg R$$

\centerline{
  \begin{tabular}{l l}
  \textit{Givens} & \textit{Goal} \\
  $P \to Q$      & $\neg R$ \\
  $Q \to \neg R$ & \\
  $P$ & \\
  $Q$ & \\
  \end{tabular}}
\vspace{20pt}

\textbf{Theorem}. \textit{Suppose $P \to Q$ and $R \to \neg Q$ are both true then $P \to \neg R$ is true.}.

\textit{Proof}. Suppose $P$. Since $P$ and $P \to Q$, it follows $Q$. Since $Q$ and $R \to \neg Q$, it follows $\neg R$. Thus $P \to \neg R$.

\vspace{20pt}

(b)

\centerline{
  \begin{tabular}{l l}
  \textit{Givens} & \textit{Goal} \\
  $P$      & $Q \to \neg (Q \to \neg P)$ \\
  \end{tabular}}
\vspace{20pt}

\centerline{
  \begin{tabular}{l l}
  \textit{Givens} & \textit{Goal} \\
  $P$            & $\neg Q$ \\
  $Q \to \neg P$ &          \\
  \end{tabular}}
\vspace{20pt}

\textbf{Theorem}. \textit{Suppose that $P$ is true then prove that $Q \to \neg (Q \to \neg P)$ is true.}

\textit{Proof}. Suppose $Q \to \neg P$. Since $P$ and $Q \to \neg P$, it follows $\neg Q$.
Thus $Q \to \neg (Q \to \neg P)$

\vspace{30pt}

3. Suppose $A \subseteq C$, and $B$ and $C$ are disjoint. Prove that if $x \in A$ then
$x \notin B$.
\vspace{20pt}

Scratch work

\centerline{
  \begin{tabular}{l l}
  \textit{Givens} & \textit{Goal} \\
  $A \subseteq C$ & $x \notin B$ \\
  $B \cap C = \varnothing$ & \\
  $x \in A$ & \\
  \end{tabular}}
\vspace{20pt}

\textbf{Theorem}. \textit{Suppose $A \subseteq C$, and $B$ and $C$ are disjoint. If $x \in A$ then
$x \notin B$.}

\textit{Proof}. Suppose $x \in A$. Since $A \subseteq C$ and $B \cap C = \varnothing$, it follows
$B \cap A = \varnothing$. But then, since $B \cap A = \varnothing$ and $x \in A$, it follows $x \notin B$.
Thus if $x \in A$ then $x \notin B$.

\vspace{30pt}

4. Suppose that $A \setminus B$ is disjoint from $C$ and $x \in A$. Prove that if $x \in C$
then $x \in B$.
\vspace{20pt}

Scratch work

\centerline{
  \begin{tabular}{l l}
  \textit{Givens} & \textit{Goal} \\
    $A \setminus B \cap C = \varnothing$ & $x \in B$ \\
    $x \in C$ & \\
    $x \in A$ & \\
  \end{tabular}}
\vspace{20pt}

\centerline{
  \begin{tabular}{l l}
  \textit{Givens} & \textit{Goal} \\
    $A \setminus B \cap C = \varnothing$ & $x \in B$ \\
    $x \in C$ & \\
    $x \in B$ & \\
  \end{tabular}}
\vspace{20pt}

\textbf{Theorem}. \textit{Suppose that $A \setminus B$ is disjoint from $C$ and $x \in A$. If $x \in C$
then $x \in B$.}

\textit{Proof}. Suppose $x \in C$. Since $A \setminus B \cap C = \varnothing$, it follows $\neg (x \in A \land x \notin B \land x \in C)$. Since $x \notin A \lor x \in B \lor x \notin C$ and $x \in C$ and $x \in A$, it follows $x \in B$. Thus, if $x \in C$ then $x \in B$.

\vspace{30pt}

∗5. Use the method of proof by contradiction to prove the theorem in Example 3.2.1.
\vspace{20pt}

\centerline{
  \begin{tabular}{l l}
  \textit{Givens} & \textit{Goal} \\
    $A \cap C \subseteq B$ & $a \notin A \setminus B$ \\
    $a \in C$ & \\
  \end{tabular}}
\vspace{20pt}

\centerline{
  \begin{tabular}{l l}
  \textit{Givens} & \textit{Goal} \\
    $A \cap C \subseteq B$ & Contradiction \\
    $a \in C$ & \\
    $a \in A \setminus B$ & \\
  \end{tabular}}
\vspace{20pt}

\centerline{
  \begin{tabular}{l l}
  \textit{Givens} & \textit{Goal} \\
    $A \cap C \subseteq B$ & Contradiction \\
    $a \in C$ & \\
    $a \in A$ & \\
    $a \notin B$ & \\
  \end{tabular}}
\vspace{20pt}

\centerline{
  \begin{tabular}{l l}
  \textit{Givens} & \textit{Goal} \\
    $A \cap C \subseteq B$ & $a \in B$ \\
    $a \in C$ & \\
    $a \in A$ & \\
    $a \notin B$ & \\
  \end{tabular}}
\vspace{20pt}

\textbf{Theorem}. \textit{Suppose $A \cap C \subseteq B$ and $a \in C$. Prove that $a \notin A \setminus B$}

\textit{Proof}. Suppose $a \in A \setminus B$. This means that $a \in A$ and $a \notin B$.
Since $A \cap C \subseteq B$ and $a \in C$ and $a \in A$, it follows that $a \in B$. But this
contradicts the fact that $a \notin B$. Therefore, if $A \cap C \subseteq B$ and $a \in C$ then $a \notin A \setminus B$.

\vspace{30pt}

6. Use the method of proof by contradiction to prove the theorem in Example 3.2.5.
\vspace{20pt}

\centerline{
  \begin{tabular}{l l}
  \textit{Givens} & \textit{Goal} \\
    $A \subseteq B$ & $a \in C$ \\
    $a \in A$ & \\
    $a \notin B \setminus C$ & \\
  \end{tabular}}
\vspace{20pt}

\centerline{
  \begin{tabular}{l l}
  \textit{Givens} & \textit{Goal} \\
    $A \subseteq B$ & Contradiction \\
    $a \in A$ & \\
    $a \notin B \setminus C$ & \\
    $a \notin C$ & \\
  \end{tabular}}
\vspace{20pt}

\centerline{
  \begin{tabular}{l l}
  \textit{Givens} & \textit{Goal} \\
    $A \subseteq B$ & Contradiction \\
    $a \in A$ & \\
    $a \notin B$ & \\
    $a \notin C$ & \\
  \end{tabular}}
\vspace{20pt}

\centerline{
  \begin{tabular}{l l}
  \textit{Givens} & \textit{Goal} \\
    $A \subseteq B$ & $a \in B$ \\
    $a \in A$ & \\
    $a \notin B$ & \\
    $a \notin C$ & \\
  \end{tabular}}
\vspace{20pt}

\textbf{Theorem}. \textit{Suppose that $A \subseteq B$, $a \in A$, and $a \notin B \setminus C$.
  Then $a \in C$.}

\textit{Proof}. Suppose $a \notin C$. Since $A \notin B \setminus C$ means $a \notin B \land a \notin C$
and $a \notin C$, it follows $a \notin B$. Since $a \in A$ and $A \subseteq B$, it follows $a \in B$.
But this contradicts the fact that $a \notin B$, Thus, $a \in C$

\vspace{30pt}

7. Suppose that $y + x = 2y - x$, and $x$ and $y$ are not both zero. Prove that $y \neq 0$.
\vspace{20pt}

Scratch work

\centerline{
  \begin{tabular}{l l}
  \textit{Givens} & \textit{Goal} \\
    $y + x = 2y - x$ & $y \neq 0$ \\
    $y = 0 \to x \neq 0$ & \\
  \end{tabular}}
\vspace{20pt}

\centerline{
  \begin{tabular}{l l}
  \textit{Givens} & \textit{Goal} \\
    $y + x = 2y - x$ & $Contradiction$ \\
    $y = 0 \to x \neq 0$ & \\
    $y = 0$ & \\
  \end{tabular}}
\vspace{20pt}

\centerline{
  \begin{tabular}{l l}
  \textit{Givens} & \textit{Goal} \\
    $y + x = 2y - x$ & $Contradiction$ \\
    $y = 0 \to x \neq 0$ & \\
    $y = 0$ & \\
    $x \neq 0$ & \\
  \end{tabular}}
\vspace{20pt}

\centerline{
  \begin{tabular}{l l}
  \textit{Givens} & \textit{Goal} \\
    $y + x = 2y - x$ & $x = 0$ \\
    $y = 0 \to x \neq 0$ & \\
    $y = 0$ & \\
    $x \neq 0$ & \\
  \end{tabular}}
\vspace{20pt}

\textbf{Theorem}. \textit{Suppose that $y + x = 2y - x$, and $x$ and $y$ are not both zero. Then $y \neq 0$}

\textit{Proof}. Suppose $y = 0$. Since $y = 0 \to x \neq 0$ and $y = 0$, it follows that $x \neq 0$.
Since $y + x = 2y - x$ and $y = 0$, it follows that $x = 0$. But this contradicts the fact that $x \neq 0$.
Thus, $y \neq 0$.

\vspace{30pt}

∗8. Suppose that $a$ and $b$ are nonzero real numbers. Prove that if $a < 1/a < b < 1/b$ then $a < -1$.
\vspace{20pt}

Scratch work

\centerline{
  \begin{tabular}{l l}
  \textit{Givens} & \textit{Goal} \\
    $a < 1/a < b < 1/b$ & $a < -1$ \\
  \end{tabular}}
\vspace{20pt}

$a + 1/a < 0$

$\frac{a^2 + 1}{a} < 0$

$a < 0$

$a^2 > 1$

$a < - 1$

$a > 1$

\textbf{Theorem}. \textit{Suppose that $a$ and $b$ are nonzero real numbers. If $a < 1/a < b < 1/b$ then $a < -1$.}

\textit{Proof}. Suppose $a < 1/a < b < 1/b$. Then $\frac{a^2 + a}{a} < 0$, it follows that $a < 0$.
Since $a < 0$ and $a < 1/a$ we get $a^2 > 1$, which means that $a < -1$ and $a > 1$. But then,
since $a < - 1$ and $a > 1$ and $a < 0$, we get $a < -1$. Thus, $a < -1$.

\vspace{30pt}

9. Suppose that $x$ and $y$ are real numbers. Prove that if $x^2 y = 2x + y$, then if $y \neq 0$ then $x \neq 0$.
\vspace{20pt}

Scratch work

\centerline{
  \begin{tabular}{l l}
  \textit{Givens} & \textit{Goal} \\
    $x^2y = 2x + y$ & $y \neq 0 \to x \neq 0$ \\
  \end{tabular}}
\vspace{20pt}

\centerline{
  \begin{tabular}{l l}
  \textit{Givens} & \textit{Goal} \\
    $x^2y = 2x + y$ & $y = 0$ \\
    $x = 0$ & \\
  \end{tabular}}
\vspace{20pt}

\textbf{Theorem}. \textit{Suppose that $x$ and $y$ are real numbers. If $x^2 y = 2x + y$, then if $y \neq 0$ then $x \neq 0$.}

\textit{Proof}. We will prove the contrapositive. Suppose $x = 0$. Since $x^2 y = 2x + y$ and $x = 0$,
if follows that $y = 0$. Thus, if $x^2 y = 2x + y$, then if $y \neq 0$ then $x \neq 0$.

\vspace{30pt}

10. Suppose that $x$ and $y$ are real numbers. Prove that if $x \neq 0$, then if $y = \frac{3x^2+2y}{x^2+2}$ then $y = 3$.
\vspace{20pt}

\centerline{
  \begin{tabular}{l l}
  \textit{Givens} & \textit{Goal} \\
    $x \neq 0$ & $y = \frac{3x^2+2y}{x^2+2} \to y = 3$ \\
  \end{tabular}}
\vspace{20pt}

\centerline{
  \begin{tabular}{l l}
  \textit{Givens} & \textit{Goal} \\
    $x \neq 0$ & $y = 3$ \\
    $y = \frac{3x^2+2y}{x^2+2}$ & \\
  \end{tabular}}
\vspace{20pt}

\centerline{
  \begin{tabular}{l l}
  \textit{Givens} & \textit{Goal} \\
    $x \neq 0$ & $y = 3$ \\
    $y = \frac{3x^2+2y}{x^2+2}$ & \\
 \end{tabular}}
\vspace{20pt}

$$y = \frac{3x^2+2y}{x^2+2}$$
$$yx^2 + 2y = 3x^2 + 2y$$
$$(y - 3)x^2 = 0$$
$$y - 3 = 0$$
$$y = 3$$

\textbf{Theorem}. \textit{Suppose that $x$ and $y$ are real numbers. If $x \neq 0$, then if $y = \frac{3x^2+2y}{x^2+2}$ then $y = 3$.}

\textit{Proof}. Suppose $y = \frac{3x^2+2y}{x^2+2}$. Then $(y - 3)x^2 = 0$. Since $x \neq 0$ and $(y - 3)x^2 = 0$, it follows that $y = 3$. Thus, if $x \neq 0$, then if $y = \frac{3x^2+2y}{x^2+2}$ then $y = 3$.

\vspace{30pt}

∗11. Consider the following incorrect theorem:

\textbf{Incorrect Theorem}. \textit{Suppose $x$ and $y$ are real numbers and $x + y = 10$.
Then $x \neq 3$ and $y \neq 8$.}

\hspace{12pt}(a) What's wrong with the following proof of the theorem?

\textit{Proof}. Suppose the conclusion of the theorem is false. Then $x = 3$ and
$y = 8$. But then $x + y = 11$, which contradicts the given
information that $x + y = 10$. Therefore the conclusion must be
true.

\hspace{12pt}(b) Show that the theorem is incorrect by finding a counterexample.
\vspace{20pt}

(a)

\centerline{
  \begin{tabular}{l l}
  \textit{Givens} & \textit{Goal} \\
    $x + y = 10$ & $(x \neq 3) \land (y \neq 8)$ \\
  \end{tabular}}

\centerline{
  \begin{tabular}{l l}
  \textit{Givens} & \textit{Goal} \\
    $x + y = 10$ & $\neg x = 3 \land \neg y = 8$ \\
  \end{tabular}}

\centerline{
  \begin{tabular}{l l}
  \textit{Givens} & \textit{Goal} \\
    $x + y = 10$ & $\neg (x = 3 \lor y = 8)$ \\
  \end{tabular}}

\centerline{
  \begin{tabular}{l l}
  \textit{Givens} & \textit{Goal} \\
    $x + y = 10$ & $Contradiction$ \\
    $x = 3 \lor y = 8$ & \\
  \end{tabular}}

\vspace{20pt}

(b) x = 3 and y = 7.

x + y = 10.

\vspace{30pt}

12. Consider the following incorrect theorem:

\textbf{Incorrect Theorem}. \textit{Suppose that $A \subseteq C$, $B \subseteq C$, and $x \in A$.
Then $x \in B$.}

\hspace{12pt}(a) What's wrong with the following proof of the theorem?

\textit{Proof}. Suppose that $x \notin B$. Since $x \in A$ and $A \subseteq C$, $x \in C$.
Since $x \notin B$ and $B \subseteq C$, $x \notin C$. But now we have proven both
$x \in C$ and $x \notin C$, so we have reached a contradiction. Therefore $x \in B$.

\hspace{12pt}(b) Show that the theorem is incorrect by finding a counterexample.
\vspace{20pt}

(a)

\centerline{
  \begin{tabular}{l l}
  \textit{Givens} & \textit{Goal} \\
    $A \subseteq C$ & $x \in B$ \\
    $B \subseteq C$ & \\
    $x \in A$ & \\
  \end{tabular}}

\centerline{
  \begin{tabular}{l l}
  \textit{Givens} & \textit{Goal} \\
    $A \subseteq C$ & Contradiction \\
    $B \subseteq C$ & \\
    $x \in A$ & \\
    $x \notin B$ & \\
    $x \in C$ & \\
  \end{tabular}}

\centerline{
  \begin{tabular}{l l}
  \textit{Givens} & \textit{Goal} \\
    $A \subseteq C$ & $x \notin C$ \\
    $B \subseteq C$ & \\
    $x \in A$ & \\
    $x \notin B$ & \\
    $x \in C$ & \\
  \end{tabular}}

The following sentence is wrong: "Since $x \notin B$ and $B \subseteq C$, $x \notin C$".

$B \subseteq C$. Using $x \notin B$ we can't conclude that $x \notin C$.

\vspace{20pt}

(b)

\def\firstcircle{(0,0) circle (5cm)}
\def\secondcircle{(180:3cm) circle (1.5cm)}
\def\thirdcircle{(0:2cm) circle (1.5cm)}

\centerline{\begin{tikzpicture}
    \begin{scope}
      \fill[cyan] \thirdcircle;
    \end{scope}
    \draw \firstcircle node[text=black,above] {$C$};
    \draw \secondcircle node [text=black,below left] {$B$};
    \draw \thirdcircle node [text=black,below right] {$A$};
  \end{tikzpicture}}

\vspace{30pt}

13. Use truth tables to show that modus tollens is a valid rule of inference.
\vspace{20pt}

\centerline{Modus tollens}:

$$P \to Q$$
$$\neg Q$$
$$\therefore \neg P$$

\centerline{
  \begin{tabular}{c c l c c}
  P & Q & $P \to Q$ & $\neg Q$ & $\neg P$ \\
  \hline
  F & F & T \textbf{T} F & T & T \\
  F & T & T \textbf{T} T & F & T \\
  T & F & F \textbf{F} F & T & F \\
  T & T & F \textbf{T} T & F & F \\
  \end{tabular}}
\vspace{10pt}

All premises are true only in the 1st row and conclusion is also true. Therefore, the argument is valid.

\vspace{30pt}

∗14. Use truth tables to check the correctness of the theorem in Example 3.2.4.
\vspace{20pt}

$$P \to (Q \to R)$$
$$\neg P \lor \neg Q \lor R$$
$$\neg (P \land Q) \lor R$$
\vspace{10pt}

$$\neg R \to (P \to \neg Q)$$
$$R \lor (\neg P \lor \neq Q)$$
$$\neg (P \land Q) \lor R$$

\centerline{
  \begin{tabular}{c c c l l}
  P & Q & R & $\neg P \lor \neg Q \lor R$ & $R \lor (\neg P \lor \neq Q)$ \\
  \hline
  F & F & F & T F T T F \textbf{T} F & F \textbf{T} T F T T F \\
  F & T & F & T F T F T \textbf{T} F & F \textbf{T} T F T F T \\
  F & F & T & T F T T F \textbf{T} T & T \textbf{T} T F T T F \\
  F & T & T & T F T F T \textbf{T} T & T \textbf{T} T F T F T \\
  T & F & F & F T T T F \textbf{T} F & F \textbf{T} F T T T F \\
  T & T & F & F T F F T \textbf{F} F & F \textbf{F} F T F F T \\
  T & F & T & F T T T F \textbf{T} T & T \textbf{T} F T T T F \\
  T & T & T & F T F F T \textbf{T} T & T \textbf{T} F T F F T \\
  \end{tabular}}
\vspace{10pt}

\vspace{30pt}

15. Use truth tables to check the correctness of the statements in exercise 1.
\vspace{20pt}

(a)

\centerline{
  \begin{tabular}{c c l l l l l}
  P & Q & R & $P \to Q$ & $Q \to R$ & $P \to Q \land Q \to R$ & $P \to R$ \\
  \hline
  F & F & F & T F \textbf{T} F & T F \textbf{T} F & T \textbf{T} T & T F \textbf{T} F \\
  F & T & F & T F \textbf{T} T & F T \textbf{F} F & T \textbf{F} F & T F \textbf{T} F \\
  F & F & T & T F \textbf{T} F & T F \textbf{T} T & T \textbf{T} T & T F \textbf{T} T \\
  F & T & T & T F \textbf{T} T & F T \textbf{T} T & T \textbf{T} T & T F \textbf{T} T \\
  T & F & F & F T \textbf{F} F & T F \textbf{T} F & F \textbf{F} T & F T \textbf{F} F \\
  T & T & F & F T \textbf{T} T & F T \textbf{F} F & T \textbf{F} F & F T \textbf{F} F \\
  T & F & T & F T \textbf{F} F & T F \textbf{T} T & F \textbf{F} T & F T \textbf{T} T \\
  T & T & T & F T \textbf{T} T & F T \textbf{T} T & T \textbf{T} T & F T \textbf{T} T \\
  \end{tabular}}
\vspace{10pt}

All premises are true in the 1,3,4,8 rows and the conclusions are also true, so the argument is valid.
\vspace{20pt}

(b)

\centerline{
  \begin{tabular}{c c l l l l}
  P & Q & R & $P \to \neg Q$ & $\neg R \to (P \to \neg Q)$ & $P \to (Q \to R)$ \\
  \hline
  F & F & F & T F T T F & F \textbf{T} T & T F \textbf{T} T F T F \\
  F & T & F & T F T F T & F \textbf{T} T & T F \textbf{T} F T F F \\
  F & F & T & T F T T F & T \textbf{T} T & T F \textbf{T} T F T T \\
  F & T & T & T F T F T & T \textbf{T} T & T F \textbf{T} F T T T \\
  T & F & F & F T T T F & F \textbf{T} T & F T \textbf{T} T F T F \\
  T & T & F & F T F F T & F \textbf{F} F & F T \textbf{F} F T F F \\
  T & F & T & F T T T F & T \textbf{T} T & F T \textbf{T} T F T T \\
  T & T & T & F T F F T & T \textbf{T} F & F T \textbf{T} F T T T \\
  \end{tabular}}

\vspace{30pt}

16. Use truth tables to check the correctness of the statements in exercise 2.
\vspace{20pt}

(a)

\centerline{
  \begin{tabular}{c c l l l l l}
  P & Q & R & $P \to Q$ & $R \to \neg Q$ & $P \to Q \land Q \to R$ & $P \to \neg R$ \\
  \hline
  F & F & F & T F \textbf{T} F & T F \textbf{T} T F & T \textbf{T} T & T F \textbf{T} T F \\
  F & T & F & T F \textbf{T} T & T F \textbf{T} F T & T \textbf{T} T & T F \textbf{T} T F \\
  F & F & T & T F \textbf{T} F & F T \textbf{T} T F & T \textbf{T} T & T F \textbf{T} F T \\
  F & T & T & T F \textbf{T} T & F T \textbf{F} F T & T \textbf{F} F & T F \textbf{T} F T \\
  T & F & F & F T \textbf{F} F & T F \textbf{T} T F & F \textbf{F} T & F T \textbf{T} T F \\
  T & T & F & F T \textbf{T} T & T F \textbf{T} F T & T \textbf{T} T & F T \textbf{T} T F \\
  T & F & T & F T \textbf{F} F & F T \textbf{T} T F & F \textbf{F} T & F T \textbf{F} F T \\
  T & T & T & F T \textbf{T} T & F T \textbf{F} F T & T \textbf{F} F & F T \textbf{F} F T \\
  \end{tabular}}
\vspace{10pt}

All premises are true in the 1,2,3,6 rows and the conclusions are also true, so the argument is valid.

(b)

\centerline{
  \begin{tabular}{c c l c c}
  P & Q & P & $Q \to \neg P$ & $Q \to \neg (Q \to \neg P)$ \\
  \hline
  F & F & \textbf{F} & T F \textbf{T} T F & T F \textbf{T} F T \\
  F & T & \textbf{F} & F T \textbf{T} T F & F T \textbf{F} F T \\
  T & F & \textbf{T} & T F \textbf{T} F T & T F \textbf{T} F T \\
  T & T & \textbf{T} & F T \textbf{F} F T & F T \textbf{T} T F \\
  \end{tabular}}
\vspace{10pt}

All premises are true only in the 3 row and the conclusion is also true, so the argument is valid.

\vspace{30pt}

17. Can the proof in Example 3.2.2 be modified to prove that if $x^2 + y = 13$
and $x \neq 3$ then $y \neq 4$? Explain.

Scratch work

\centerline{
  \begin{tabular}{l l}
  \textit{Givens} & \textit{Goal} \\
    $x^2 + y = 13$ & $y \neq 4$ \\
    $x \neq 3$ & \\
  \end{tabular}}

\centerline{
  \begin{tabular}{l l}
  \textit{Givens} & \textit{Goal} \\
    $x^2 + y = 13$ & Contradiction \\
    $x \neq 3$ & \\
    $y = 4$ & \\
  \end{tabular}}

\centerline{
  \begin{tabular}{l l}
  \textit{Givens} & \textit{Goal} \\
    $x^2 + y = 13$ & $x = 3$ \\
    $x \neq 3$ & \\
    $y = 4$ & \\
  \end{tabular}}

$$x^2 = 9$$

$$x_1 = 3$$

$$x_2 = -3$$

No, it can't. Since having only $x^2 = 9$ it's not possible to prove that $x = 3$, because $x$ may be equal to -3 as well.
\vspace{50pt}

\textbf{3.3. Proofs Involving Quantifiers}
















































\end{document}
